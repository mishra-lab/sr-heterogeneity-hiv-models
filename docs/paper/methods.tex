We searched \textsc{medline} and \textsc{embase} via Ovid  %SM: suggest including a statement about using/following guidelines to conduct the scoping review. We conducted and reported the scoping review according to XXX recommendations[e.g. PRISMA, etc.].
using search terms related to \hiv, \ssa, and transmission modelling
(full search terms and yield are given in Appendix~\ref{a:search}).
After de-duplication of studies,
potentially relevant studies were identified by title and abstract screening, 
followed by full text review and any available supplementary material using the inclusion/exclusion criteria.
One reviewer (JK) conducted the search and data extraction.
% ==============================================================================
\subsection{Inclusion/Exclusion Criteria} %suggest following format used in other papers re: how inclusion/exclusion section are detailed/organized, etc. with more precision - exact dates for inclusion (X date to Jan 1, 2019; how is SSA defined? [provide list of countries in appendix, etc.]); English language only?,  etc. I found this paragraph a bit hard to know how exactly someone else would reproduce the inclusion/exclusion. e.g. consider listing and structuring it a bit more and then providing definitions? can see how other papers did it to help find a format that you like? the methods need to show they can be reproduced [e.g. with details in appendix if needed]. 
% date/years of studies (using publication date); language; what is defined as SSA (using UNAIDS definition/WHO definition of countries - and list countries in appendix)? compartmental, deterministic models only?
\label{ss:meth:inex}
Studies published after 2019 were included if
they applied a dynamical model of sexual \hiv transmission
to any context within \ssa.
A \hiv transmission model projects the number of new \hiv infections  %feel like many of these statements/definitons would benefit from citations
in a population based on initial conditions.
In a \emph{dynamical} transmission model,
the number of infections projected at time $t$ is influenced by
the number of infections previously projected by the model before time $t$.
Thus dynamical models can capture
higher-order (nonlinear) population-level transmission dynamics,
such as indirect benefits of prevention interventions.
Models were only included if they considered sexual transmission of \hiv,
possibly alongside other routes of transmission,
such as mother-to-child and parenteral,
and possibly alongside transmission of other infectious diseases
such as other \sti, tuberculosis, or malaria.
We excluded studies without primary modelling results,
such as reviews and commentaries,
as well as conference publications.
% ==============================================================================
\subsection{Data Extraction}
\label{ss:meth:data}
% ------------------------------------------------------------------------------
\paragraph{Epidemic Context}
Studies were categorized by the geographic scale of the simulated epidemic
(city, sub-national, national, super-national)   %super-national? 
and by whether multiple geographic contexts were considered.
Whenever the impact of interventions was assessed,  % did not understand the sentence, as unclear re: what 'whenever impact of interventions was assessed.." refers to
the epidemic scale was classified according to overall \hiv prevalence:
low ($<1\%$), medium ($1-5\%$), or high ($>5\%$).
% ------------------------------------------------------------------------------
\paragraph{Interventions}
We examined which of the following interventions were included in the model.
We classified each simulated intervention as either
part of \emph{historical} interventions
(typically constituting a ``status-quo'' scenario,
and possibly including historical behaviour change),  %did not follow the statement in the parentheses / could it be clarified?
or part of \emph{counterfactual} interventions
(typically to assess intervention impact in roll-out/scale-up scenarios).
\begin{itemize}
  \item \art\cdf: any increase in \art coverage according to a \cdf threshold
%  \item \art\who: any scale-up of \art coverage according to \who clinical criteria
  \item \art\utt: any increase in \art coverage without initiation condition
        (``universal test \& treat'')
  \item \vmmc: voluntary medical male circumcision
  \item \prep: pre-exposure prophylaxis, by any route of administration
  \item Condoms: any intervention which was simulated to increase condom use
  \item Partners: any intervention which reduces the rate of partnership formation
  \item \sti: any intervention reducing \hiv transmission by reducing \sti symptom burden
  \item Generic: an unspecified generic \hiv prevention interventions
%  \item Vaccine: a hypothetical \hiv vaccine
%  \item Cure: a hypothetical \hiv cure
%  \item Structural: any intervention addressing social or economic risk factors for transmission,
%        such as decriminalization of sex work, violence reduction, cash transfer, etc.
\end{itemize}
For each simulated intervention scenario,
we extracted information to define each combination of interventions, including
combinations of historical interventions,
counterfactual interventions on top of historical interventions, and  %was not super clear on differnece between countefactual interventions on top of historical interventions vs. combination of counterfactual interventions. might be somewhat unclear to JIAS/AIDS/JAIDS readers... might want to run it by a non-modeler friend re: if we can simplify/explain a bit more for clarity?
combinations of counterfactual interventions.
We categorized each counterfactual simulated intervention as focusing one of the following risk groups:
all women; all men; all people; young women; young men; all young people;
\fsw; clients; \msm; other generic high or low risk group.
For studies that simulated historical or counterfactual interventions
reaching multiple risk groups concurrently,
such as ``80\% \art coverage overall'',
we extracted information on whether intervention coverage was assumed to be equal across modelled risk groups,  %instead of 'noted' - consider using language typically used in SRs around 'extracted information on..." etc.
possibly ignoring historical gaps/future challenges in reaching higher risk groups. %making a judgement here vs. objective methods i think this is more of a results/interpretation vs.so would remove this part of the sentence or rephrase to sound less harsh :). 
\par
We noted whether each study quantified the impact of counterfactual interventions
on each of the following outcomes:
\hiv epidemic (incidence, prevalence, total infections, and/or mortality);
reproduction number;
transmitted drug resistance; and
any economic analysis.
% ------------------------------------------------------------------------------
\paragraph{Risk Heterogeneity}
We defined risk heterogeneity in compartmental models
as population stratifications other than age, health state, or intervention involvement
that conferred differential risk of \hiv acquisition and/or transmission. %cite
We documented the defining characteristics of modelled risk groups, including:
sex, %do we mean sex or gender or both?
different rates of sexual partnership formation, and different types of sexual partnerships. %only sexual partnerships yes?
We noted whether any of the following key populations was included in the model: %citation for key population from literature / WHO, UNAIDS, etc.
female sex workers (\fsw);
male clients of \fsw (\cli);
adolescent girls and young women (\agyw); %give citation for AGYW as KP
men who have sex with men (\msm);  %including MSM who sell sex?
and people who inject drugs (\pwid).
\par
We noted which of the following characteristics were used to define  %not sure i followed...
simulated sexual partnerships:
the risk groups involved;
different volumes of sex (total number coital acts per partnership); and
different levels of condom use.
We noted whether simulated partnerships represented
any of the following identifiable types:
main/spousal;
casual/extramarital;  %can we say casual (and then say could be defined as... etc.) - the term 'extramariital' is a whilte-person/colonilali appraoch to defining relationships in Africa (my colleagues have told me - and the models just call them as such and based on data that define as such, but mechanistically - it just reflects a differnet number of sex acts, type of sex acts, duration, condom use, etc. etc. so I think using casual alone is better [and then in appendix or in parenthesis, can say how these could have been defined in the models). i.e. lets try to avoid using value-based terminology in the methods. 
formal sex work; %based on feedback from communities, there is a general move to say use term paid sex in the context of formal sex work (instead of commercial sex)
transactional (exchange of gifts/favours for sex,
outside the context of formal sex work). % TODO: this definition? %see papers on transactional sex? (there is a systematic review on transactional sex in SSA)
Finally, we noted whether models simulated any degree of assortative vs proportionate %cite & define 
partnership formation (mixing) between risk groups.
If some partnership types were only formed by certain risk groups,
mixing was automatically considered assortative.
\par
We additionally noted the number of unique age groups, the nature of age group mixing---%
proportionate, assortative without age differences, or assortative with age differences
(e.g. younger females with older males)---%
and whether age conferred any additional risk differences beyond mixing
(e.g. higher rates of partnership formation).
Finally, we counted the number \hiv infection states modelled (excluding treatment),
and noted which characteristic was used to define the states, including:
early infection (must include increased infectivity),
\who clinical criteria, \cdf count, and viral load.
