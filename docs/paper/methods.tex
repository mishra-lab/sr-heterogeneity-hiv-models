We searched \textsc{medline} and \textsc{embase} via Ovid
using search terms related to \hiv, \ssa, and transmission modelling
(full search terms and hits are given in Appendix~\ref{a:search}).
Duplicate studies were automatically removed.
Potentially relevant studies was identified by title and abstract screening.
The final selection of studies were identified by review of
the full text and any available supplementary material.
Data extraction also considered the full text and supplementary material.
One reviewer (JK) conducted the search and data extraction.
% ==============================================================================
\subsection{Inclusion/Exclusion Criteria}
\label{ss:meth:inex}
Studies published after 2019 were included if
they applied a dynamical model of sexual \hiv transmission
to any context within \ssa.
A \hiv transmission model projects the number of new \hiv infections
in a population based on initial conditions.
In a \emph{dynamical} transmission model,
the number of infections projected at time $t$ is influenced by
the number of infections previously projected by the model before time $t$.
Thus dynamical models can capture
higher-order (nonlinear) population-level transmission dynamics,
such as indirect benefits of prevention interventions.
Models were only included if they considered sexual transmission of \hiv,
possibly alongside other routes of transmission,
such as mother-to-child and parenteral,
and possibly alongside transmission of other infectious diseases
such as other \sti, tuberculosis, or malaria.
We excluded studies without primary modelling results,
such as reviews and commentaries,
as well as conference publications.
% ==============================================================================
\subsection{Data Extraction}
\label{ss:meth:data}
% ------------------------------------------------------------------------------
\paragraph{Epidemic Context}
Studies were categorized by the geographic scale of the simulated epidemic
(city, sub-national, national, super-national)
and by whether multiple geographic contexts were considered.
Whenever the impact of interventions was assessed,
the epidemic scale was classified according to overall \hiv prevalence:
low ($<1\%$), medium ($1-5\%$), or high ($>5\%$).
% ------------------------------------------------------------------------------
\paragraph{Interventions}
We examined which of the following interventions were included in the model.
We classified each simulated intervention as either
part of \emph{historical} interventions
(typically constituting a ``status-quo'' scenario,
and possibly including historical behaviour change),
or part of \emph{counterfactual} interventions
(typically to assess intervention impact in roll-out/scale-up scenarios).
\begin{itemize}
  \item \art\cdf: any increase in \art coverage according to a \cdf threshold
%  \item \art\who: any scale-up of \art coverage according to \who clinical criteria
  \item \art\utt: any increase in \art coverage without initiation condition
        (``universal test \& treat'')
  \item \vmmc: voluntary medical male circumcision
  \item \prep: pre-exposure prophylaxis, by any route of administration
  \item Condoms: any intervention which was simulated to increase condom use
  \item Partners: any intervention which reduces the rate of partnership formation
  \item \sti: any intervention reducing \hiv transmission by reducing \sti symptom burden
  \item Generic: an unspecified generic \hiv prevention interventions
%  \item Vaccine: a hypothetical \hiv vaccine
%  \item Cure: a hypothetical \hiv cure
%  \item Structural: any intervention addressing social or economic risk factors for transmission,
%        such as decriminalization of sex work, violence reduction, cash transfer, etc.
\end{itemize}
For each simulated intervention scenario,
we noted which combinations of interventions were considered, including
combinations of historical interventions,
counterfactual interventions on top of historical interventions, and
combinations of counterfactual interventions.
We categorized each counterfactual simulated intervention as focusing one of the following risk groups:
all women; all men; all people; young women; young men; all young people;
\fsw; clients; \msm; other generic high or low risk group.
For studies that simulated historical or counterfactual interventions
reaching multiple risk groups concurrently,
such as ``80\% \art coverage overall'',
we noted whether intervention coverage was assumed to be equal across modelled risk groups,
possibly ignoring historical gaps/future challenges in reaching higher risk groups.
\par
We noted whether each study quantified the impact of counterfactual interventions
on each of the following outcomes:
\hiv epidemic (incidence, prevalence, total infections, and/or mortality);
reproduction number;
transmitted drug resistance; and
any economic analysis.
% ------------------------------------------------------------------------------
\paragraph{Risk Heterogeneity}
Risk heterogeneity was identified in compartmental models
as population stratifications other than age, health state, or intervention involvement
that conferred differential risk of \hiv acquisition and/or transmission.
We documented the defining characteristics of modelled risk groups, including:
sex, different rates of partnership formation, and different types of partnerships formed.
We noted whether each of the following key populations was included in the model:
female sex workers (\fsw);
male clients of \fsw (\cli);
adolescent girls and young women (\agyw);
men who have sex with men (\msm);
and people who inject drugs (\pwid).
\par
We noted which of the following characteristics were used to define
simulated sexual partnerships:
the risk groups involved;
different volumes of sex (total number coital acts per partnership); and
different levels of condom use.
We noted whether simulated partnerships represented
any of the following identifiable types:
main/spousal;
casual/extramarital;
commercial/sex work;
transactional (exchange of gifts/favours for sex,
outside the context of formal sex work). % TODO: this definition?
Finally, we noted whether models simulated any degree of assortative vs proportionate
partnership formation (mixing) between risk groups.
If some partnership types were only formed by certain risk groups,
mixing was automatically considered assortative.
\par
We additionally noted the number of unique age groups, the nature of age group mixing---%
proportionate, assortative without age differences, or assortative with age differences
(e.g. younger females with older males)---%
and whether age conferred any additional risk differences beyond mixing
(e.g. higher rates of partnership formation).
Finally, we counted the number \hiv infection states modelled (excluding treatment),
and noted which characteristic was used to define the states, including:
early infection (must include increased infectivity),
\who clinical criteria, \cdf count, and viral load.
