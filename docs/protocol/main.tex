% !TeX TS-program = pdflatex
% !BIB TS-program = bibtex
\documentclass{article}
\usepackage{newpxmath}
\usepackage[osf]{ETbb}
\renewcommand*{\ttdefault}{cmtt}
\addbibresource{../refs/refs.bib}
\newcommand{\reviewtitle}{%
  Heterogeneity and mixing in
  dynamical models of HIV transmission:\linebreak[1]
  a scoping review of parameterizations}
\title{\Large\textsc{\large{Scoping Review Protocol:}}\\\reviewtitle}
\author{Jesse Knight}
%%%%%%%%%%%%%%%%%%%%%%%%%%%%%%%%%%%%%%%%%%%%%%%%%%%%%%%%%%%%%%%%%%%%%%%%%%%%%%%%
\begin{document}
%%%%%%%%%%%%%%%%%%%%%%%%%%%%%%%%%%%%%%%%%%%%%%%%%%%%%%%%%%%%%%%%%%%%%%%%%%%%%%%%
\maketitle
\tableofcontents
\clearpage
%%%%%%%%%%%%%%%%%%%%%%%%%%%%%%%%%%%%%%%%%%%%%%%%%%%%%%%%%%%%%%%%%%%%%%%%%%%%%%%%
\section{General Information}
% ==============================================================================
\subsection{Identifying Information}
\paragraph{Date} 2020 February
\paragraph{Principle Investigators \& Affiliations}\n
Jesse Knight\aff{1};
Sharmistha Mishra\aff{1}
\par
\aff{1}Institute of Medical Sciences, University of Toronto
\paragraph{Co-Investigators}
N/A
\paragraph{Review Title}\n
\textit{\reviewtitle}
\paragraph{Research Funding}
NSERC CGS-D
% ==============================================================================
\subsection{Background \& Rationale}
% --------------------------------------------------------------------------------------------------
\subsubsection*{Different Populations need Different \hiv Interventions}
Advances in \hiv antiretroviral treatment (\art)
have produced highly effective drug regimens,
whereby circulating levels of \hiv virus
in adherent patients are reduced to undetectable levels%
~\cite{TBD}.
Viral suppression by \art has clear individual-level benefits
for health and quality of life~\cite{TBD}.
Moreover, recent trials have suggested that
virally suppressed individuals cannot transmit \hiv,
a finding described as:
``undetectable = untransmittable'' (U=U)~\cite{Eisinger2019}.
Inspired by U=U, researchers and policymakers have called for
rapid scale-up of \art coverage
as the main intervention by which to reduce \hiv incidence (``treatment as prevention'')
in the widespread epidemics of Sub-Saharan Africa~\cite{TBD}.
Global ambition to scale up \art coverage is further motivated by
the \textsc{unaids} \mbox{90-90-90} targets, defined as:
90\% of people living with \hiv are diagnosed;
90\% of those diagnosed are on \art; and
90\% of those on \art are virally suppressed.
\par
Unfortunately, several large-scale trials aiming to demonstrate
population-level impact of treatment as prevention in Sub-Saharan Africa
have failed to show a significant reduction in new infections%
~\cite{Iwuji2018,Havlir2018,Hayes2019}.
As suggested by \textcite{Baral2019} and others, these unexpected results
might be attributable to implementation challenges at scale.
Such challenges can emerge at several steps along the treatment cascade, including:
testing for \hiv,
linkage to care after a positive test,
starting \art after linkage to care,
achieving viral suppression after starting \art~\cite{TBD}.
Moreover, individuals who are most likely to experience challenges in \hiv care
are often at the highest risk of \hiv acquisition and onward transmission~\cite{TBD}.
Groups of vulnerable individuals in the epidemic
are sometimes described as ``key populations''.
\par
Several key populations have been identified, including:
adolescent girls and young women;
sex workers;
men who have sex with men;
transgender people;
prisoners;
and people who inject drugs~\cite{WHO2014HIVKP}.
Key populations often experience
several risk factors for \hiv transmission and barriers to care, such as:
violence and coercion into unsafe sex;
criminalization of lifestyle;
stigma related to lifestyle or \hiv status;
housing and financial instability;
and substance abuse~\cite{Mountain2014, WHO2014HIVKP}.
To meet the unique \hiv prevention and treatment needs of key populations,
specific interventions are needed which
address known vulnerabilities~\cite{WHO2014HIVKP}.
For example, risk of \hiv transmission can be reduced through
needle exchange programs~\cite{Kurth2011},
increased access to condoms~\cite{TBD},
and financial support to reduce transactional sex~\cite{Pettifor2012}.
Similarly, outreach and support by community peers
can increase engagement of key populations in \hiv care~\cite{WHO2014HIVKP}.
\par
Despite considerable evidence supporting the need for diversified \hiv interventions,
recent large-scale studies of treatment as prevention
have not considered the unique needs of key populations%
~\cite{Iwuji2018,Havlir2018,Hayes2019}.
Failure to deliver appropriate interventions to key populations
has left these groups far behind global progress toward
the \mbox{90-90-90} targets~\cite{Gupta2017},
threatening to undermine the expected benefits of treatment as prevention.
% --------------------------------------------------------------------------------------------------
\subsubsection*{Mathematical Modelling of \hiv Transmission}
Population-level models of \hiv transmission have long been used to
project \hiv epidemic trajectories
(e.g.\ incidence over time)
and predict intervention impacts
(e.g.\ reduction in incidence after X years)%
~\cite{Eaton2012}.
In popular compartmental models,
overall populations are stratified by disease state and risk group,
while differential equations are used to govern movement of individuals between compartments.
% TODO: a figure could be helpful here.
Many different compartmental model structures have been used,
from a 3-compartment model, representing
3 disease states in a homogeneous population~\cite{Moghadas2003},
to a 294-compartment model, representing
21 disease states and 14 risk groups~\cite{Kerr2015}.
\par
Unfortunately, differences in model structure and assumptions
have been shown to substantially influence projections of
epidemic trajectory and intervention impact~\cite{Eaton2012,Hontelez2013}.
Most importantly, failure to model heterogeneity in risk
results in lower basic reproduction number $R_0$~\cite{Anderson1986},
which could lead to overestimated ease of epidemic control
through universal treatment as prevention~\cite{Hontelez2013}.
And yet, several mathematical models that were used to support treatment as prevention
did not consider heterogeneity in risk of \hiv acquisition or transmission%
~\cite{Granich2009,Eaton2012,TBD}.
Even models that did consider risk heterogeneity
rarely acknowledged known differences in the treatment cascade across risk groups%
~\cite{Eaton2012,TBD},
such as among key populations~\cite{Gupta2017}.
\textcite{Knight2019} showed that
the modelled impact of achieving \mbox{90-90-90} in a population overall
was highly dependent on which risk groups were left behind in the remaining \mbox{``10-10-10''},
emphasizing that differences in treatment cascade cannot be ignored.
Finally, simulated sexual mixing between risk groups
has generally been simpler than observed in reality~\cite{TBD},
with potential implications for validity of modelling results.
For example, \textcite{Wang2019} have shown that
failure to model assortative mixing by \hiv status among men who have sex with men
may result in underestimated impact of pre-exposure prophylaxis.
\par
One major reason why risk groups and mixing may be missing
from \hiv transmission models is lack of data.
Despite best efforts, key populations are often not captured by
large-scale demographic and health surveys, such as those by \textcite{DHS},
due to several barriers:
household-based sampling methodologies,
criminalization of lifestyle,
social desirability bias,
and stigma~\cite{Abdul-Quader2014}.
For example, in the 2006-07 Eswatini demographic and health survey~\cite{DHS-SWZ}
just 0.2\% male respondents reported paying for sex, while estimates of
commercial sex client populations in similar regions were as high as 8\%~\cite{Carael2006}.
% TODO: FSW too/instead?
In many cases, parallel surveys with specific sampling methodologies and community involvement
can overcome these barriers, facilitating data collection on key populations~\cite{SwaziKP2014}.
Moreover, collection of key populations data can and should be integrated with
modelling work and evaluation of tailored interventions.
% --------------------------------------------------------------------------------------------------
\subsubsection*{Future Work}
This review aims to identify parameterizations of risk heterogeneity and mixing
used in previous transmission models of \hiv in Sub-Saharan Africa.
Identified parameterizations will then be considered in a systematic model comparison study,
similar to that by \textcite{Hontelez2013}.
For example, the projected impact of universal treatment as prevention will be compared
in models with versus without female sex workers, or
in models with versus without mixing by risk group.
In comparing parameterizations, potential biases and uncertainties associated with
simpler models can be estimated.
Furthermore, considering the importance of data to inform complex models,
the model comparison study will identify key pieces of information
which are necessary to construct accurate models,
so that these data may be prioritized for collection going forward.

\clearpage % TEMP
% ==============================================================================
\subsection{Review Questions}
\label{ss:questions}
\begin{enumerate}
  \item\label{q:context}
  In which contexts (geographies, populations, time periods)
  within \ssa, and
  for what applications (research questions)
  have \hiv transmission models been used?
  \item\label{q:hetero}
  What parameterizations have been used
  to represent risk heterogeneity and mixing
  in \emph{deterministic compartmental} \hiv transmission models
  applied to \ssa?
  \item\label{q:trends}
  How and why are particular parameterizations of risk heterogeneity and mixing
  in \emph{deterministic compartmental} \hiv transmission models
  associated with specific contexts and applications within \ssa?
\end{enumerate}
%%%%%%%%%%%%%%%%%%%%%%%%%%%%%%%%%%%%%%%%%%%%%%%%%%%%%%%%%%%%%%%%%%%%%%%%%%%%%%%%
\section{Methods}
\label{s:methods}
% <<< SM_REVIEW >>>
% ==============================================================================
\subsection{Eligibility Criteria}
\label{ss:eligibility}
Note that for Question~\ref{q:context},
our analysis is not limited to
\emph{deterministic compartmental} transmission models.
So, let \ss{c} denote criteria for
\emph{deterministic compartmental} transmission models
applied only to Questions~\ref{q:hetero}~and~\ref{q:trends}.
Our criteria are:
% ------------------------------------------------------------------------------
\paragraph{Publication Details}\n
\begin{criteria}{Include}
  \item English language
  \item published before 2020
  \item peer-reviewed journal article (not review)
\end{criteria}
\begin{criteria}{Exclude}
  \item non-English language
  \item published in 2020 or later
  \item non-peer reviewed journal article
  \item review article (references will be screened)
  \item textbook, grey literature, opinions, comments, conference abstracts
\end{criteria}
% ------------------------------------------------------------------------------
\paragraph{Mathematical Model of Transmission}\n
\begin{criteria}{Include}
  \item dynamical transmission model\footnotemark[1]
  \item between-host dynamics
  \item deterministic model\ss{c}
  \item compartmental model\ss{c}
\end{criteria}
\begin{criteria}{Exclude}
  \item no transmission modelling
  \item transmission model is not dynamical
  \item within-host/cellular/protein modelling
  \item stochastic (random) model\ss{c}
  \item individual-based model\ss{c}
\end{criteria}
\footnotetext[1]{We define a \emph{dynamical model} as one where
  the rates of change of system variables are
  a function of the current system state
  (e.g. a first-order ODE system),
  with fixed functional form.
}
% ------------------------------------------------------------------------------
\paragraph{Epidemic Context}\n
\begin{criteria}{Include}
  \item \hiv modelled (at least)
  \item any region in Sub-Saharan Africa (\ssa)
\end{criteria}
\begin{criteria}{Exclude}
  \item only other infections modelled
  \item theoretical context or only region(s) outside of \ssa modelled
\end{criteria}
\clearpage % TEMP
% ==============================================================================
\subsection{Information Sources}
We search the following databases: \textsc{medline}, \textsc{embase} via Ovid.
% ==============================================================================
\subsection{Search Strategy}
Our search strategy aims to identify
any type of \hiv transmission model applied in any context.
We will later manually identify
which models were deterministic and compartmental,
and which works applied the model to \ssa context.
% ------------------------------------------------------------------------------
\paragraph{Validation References}\n
Before performing the search, we identified
9 publications of \hiv modelling applied to \ssa.
We ensure that these 9 validation references (VR)
are contained in our search results,
as an indicator that the search is performing well.
\par
\valrefs{validation/ssa.bibid}
% ------------------------------------------------------------------------------
\paragraph{Search Terms}\n
We operationalize our inclusion \& exclusion criteria
using the search terms shown in Appendix~\ref{app:search}.
We implement publication year and language criteria via Ovid ``limits'',
but did not filter publication types (review, etc.),
as we found such classifications to be unreliable.
% ------------------------------------------------------------------------------
\paragraph{Results}\n
The results from \textsc{medline} \& \textsc{embase} on 2020 March 20 were:
\begin{table}[h]
  \centerline{\searchtable{summary}}
\end{table}
\par
These results (5) then form our initial database for screening.
% <<< SM_REVIEW >>>
% ==============================================================================
\subsection{Data Management}
Based on the initial search,
the bibliographic information (including abstract)
of non-duplicate matching items will be
exported from the search result in \textsc{xml} format,
and uploaded to Covidence for abstract screening.
Covidence provides tools for including and excluding items
based on a set of user-defined criteria,
tracking the results of review.
\par
The full texts of included items will then be sought
using institutional access and support.
%For each item, the data extraction form given in Appendix~\ref{app:form}
%(Google Forms) will be completed.
%Upon submission of the form, a row in the linked tabular database
%(Google Sheets) will be automatically populated with the results.
%Analysis of the tabular data will then be performed in Python,
%with the aim of generating statistics and figures to answer
%the \nameref{ss:questions}.
% ==============================================================================
\subsection{Selection Process}
Following upload of the initial search results to Covidence,
one reviewer (JK) will screen the abstracts for inclusion
using the \nameref{ss:eligibility}.
Unclear edge cases will be resolved by discussion with SM.
% ==============================================================================
\subsection{Data Extraction}
%Appendix~\ref{app:form} gives the data extraction form to be used
%for each included reference.
%The form includes a combination of
%short text inputs and pre-defined categorical variables
%in the following sections:
%\newcommand{\ibox}[1]{\parbox{0.25\linewidth}{#1}}
%\begin{itemize}
%	\item \ibox{Article meta-data:}     authors, journal, title, etc.
%	\item \ibox{Model context:}         geography, time, research questions
%	\item \ibox{Model approach:}        model type, code
%	\item \ibox{Modelled populations:}  sex, age, risk, turnover, interventions
%	\item \ibox{Modelled biology:}      \cdf, viral load, treatment
%	\item \ibox{Modelled transmission:} modes, types of acts, partnership types, modifiers
%	\item \ibox{Data:}                  calibration targets
%\end{itemize}
%Some short text inputs will capture semi-structured data,
%such as the relative sizes of population strata, like:
%\par
%  \searchtext{HSF-low: 0.35, HSF-medium: 0.125, HSF-high: 0.025, ...}
%\par
%corresponding to: heterosexual females in the low risk group representing
%$0.35$ proportion of the total population, etc.
%Provided the input are structured in the suggested way,
%the text can later be parsed in Python,
%even though the population strata definitions may change from model to model.
\par
Data extraction and subsequent analysis will be completed by one reviewer (JK).
% ==============================================================================
\subsection{Quality Assessment}
N/A
% ==============================================================================
\subsection{Data Synthesis}
% ------------------------------------------------------------------------------
\subsubsection{Question \ref{q:context}: Context}
% In which contexts (geographies, populations, time periods) within SSA,
% and for what applications (research questions) have HIV transmission models been used?
\paragraph{Geography}
We will summarize the proportion of models representing
each of the following scales of geography:
\emph{city, multiple cities, province, multiple provinces, country, multiple countries, regional}.
Additionally, we will summarize the number of times each country has been modelled at the national level.
\paragraph{Epidemic Phase}
Since the efficacy of various interventions may depend on the epidemic phase, % Boily2002
we will approximate epidemic phase using % @SM
(a) the trend of overall \hiv incidence and prevalence
at time of intervention roll-out (where applicable), using one of:
\emph{increasing, decreasing, flat ($\pm 5\%$ relative per year), varies}; or
(b) whether the time of intervention roll-out (where applicable) is:
\emph{before, after, or within 5 years of peak overall \hiv prevalence}.
\paragraph{Key Populations}
For each of the following key populations:
\begin{itemize}[topsep=0pt]
  \item adolescent girls and young women (\textsc{agyw}) % @SM
  \item female sex workers (\fsw)
  \item \fsw clients
  \item men who have sex with men (\textsc{msm})
  \item people who inject drugs (\textsc{pwid})
\end{itemize}
we will summarize the proportion of models that:
\begin{itemize}[topsep=0pt]
  \item specifically model the population
  \item use at least one population-specific behavioural data source for that country
  \item calibrate the model to reflect at least one population-specific \hiv prevalence and/or incidence estimate
\end{itemize}
\paragraph{Interventions}
For each of the following interventions:
\begin{itemize}[topsep=0pt]
  \item ART
  \item PrEP
  \item VMMC
  \item HIV vaccine
  \item TB treatment
  \item STI coinfection treatment
  \item condom use
  \item other behavioural interventions
\end{itemize}
we will summarize the proportion of models that:
\begin{itemize}[topsep=0pt]
  \item model the intervention to reflect historical events
  \item model future intervention as part of a research question
  \item model the intervention as applied to all model populations equally
        vs focused on specific populations
\end{itemize}
For those papers examining focused interventions,
we will report the distribution of interventions and target populations.
\paragraph{Research Questions}
We aim to categorize all research questions according to the following idea:
\emph{the impact of (X) on (Y)},
where \emph{X} can be:
\begin{itemize}[topsep=0pt]
  \item epidemic context
  \item intervention type
  \item intervention targeting or scale
  \item model assumptions
\end{itemize}
and \emph{Y} can be:
\begin{itemize}[topsep=0pt]
  \item projected epidemic
  \item projection accuracy
  \item intervention impact
  \item intervention cost effectiveness
\end{itemize}
Papers may have multiple \emph{X} or \emph{Y}.
% ------------------------------------------------------------------------------
\subsubsection{Question \ref{q:hetero}: Parameterizations}
% What parameterizations have been used to represent risk heterogeneity and mixing
% in deterministic compartmental HIV transmission models applied to SSA?
\paragraph{Risk}
First, we will summarize the \emph{number} of risk groups employed by each model.
Next, noting the proportionality between the basic reproduction number $R_o$
and heterogeneity in partner change rates $H_c$:
\begin{equation}
  R_o \ge \beta D \underbrace{\left(\mu_c + \frac{\sigma_c^2}{\mu_c}\right)}_{H_c}
\end{equation}
we will estimate $H_c$ using:
(a) the relative sizes of each risk group (as a proportion of total population), and
(b) the average contact rates of each risk group;
We will also note:
(c) the relative incidence rates experienced by each risk group
(relative to the lowest risk group).
We will also consider alternative measures of risk dispersion based on the same data, such as
the \emph{Gini coefficient}, and the \emph{index of dispersion}.
Where risk group sizes and/or contact rates and/or relative incidence rates
vary across time and/or model fits,
we will attempt to use any reported posterior means where available,
else any reported prior means,
else we will exclude that paper from this part of the analysis.
\paragraph{Transmission Modifiers}
We will summarize the proportion of models which consider modification of
the probability of transmission by each of the following factors:
\begin{itemize}[topsep=0pt]
  \item male circumcision
  \item condom use
  \item STI coinfection
\end{itemize}
Specifically, we will report proportions of models that consider:
\begin{itemize}[topsep=0pt]
  \item differences in the modifier by risk group
  \item changes to the modifier over time
\end{itemize}
We will also note the proportion of models which simulated transmission
at the sex-act vs partnership level.
\paragraph{Age}
\paragraph{Health}
The major health dimensions of interest are as follows:
\begin{itemize}[topsep=0pt]
  \item \hiv viral load
  \item \cdf count
  \item \textsc{sti} coinfection
\end{itemize}
For each dimension, we will report:
\begin{itemize}[topsep=0pt]
  \item the proportion of papers which stratified by the dimension
  \item the distribution of how many strata were defined along the dimension across papers
  \item the proportion of papers that calibrated to country-specific data for this dimension
\end{itemize}
% ------------------------------------------------------------------------------
\subsubsection{Question \ref{q:trends}: Trends}
% How and why are particular parameterizations of risk heterogeneity and mixing
% in deterministic compartmental HIV transmission models
% associated with specific contexts and applications within SSA?
In this research question, we are interested to know whether
systemic biases have emerged in terms of which parameterizations

\clearpage
%%%%%%%%%%%%%%%%%%%%%%%%%%%%%%%%%%%%%%%%%%%%%%%%%%%%%%%%%%%%%%%%%%%%%%%%%%%%%%%%
\printbibliography
\clearpage
\appendix
%%%%%%%%%%%%%%%%%%%%%%%%%%%%%%%%%%%%%%%%%%%%%%%%%%%%%%%%%%%%%%%%%%%%%%%%%%%%%%%%
\section{Full Search Terms \& Results}\label{app:search}
Our search strategy and component results are as follows, where
\searchtext{[section]} refers to the result from another section,
\searchtext{term/} denotes a MeSH term, and
\searchtext{.mp} searches the main text fields, including
title, abstract, and heading words.
\small
\paragraph{Combined} \n\nobreak\searchtable{summary}
\paragraph{Model}    \n\nobreak\searchtable{model}
\paragraph{HIV}      \n\nobreak\searchtable{hiv}
\paragraph{Exclude}  \n\nobreak\searchtable{exclude}
\paragraph{SSA}      \n\nobreak\searchtable{ssa}

%%%%%%%%%%%%%%%%%%%%%%%%%%%%%%%%%%%%%%%%%%%%%%%%%%%%%%%%%%%%%%%%%%%%%%%%%%%%%%%%
%\includepdf[pages=1,pagecommand={\section{Data Extraction Form}\label{app:form}}]
%{../data/extraction-form.pdf}
%\includepdf[pages=2-,pagecommand={}]
%{../data/extraction-form.pdf}
%%%%%%%%%%%%%%%%%%%%%%%%%%%%%%%%%%%%%%%%%%%%%%%%%%%%%%%%%%%%%%%%%%%%%%%%%%%%%%%%
\end{document}
%%%%%%%%%%%%%%%%%%%%%%%%%%%%%%%%%%%%%%%%%%%%%%%%%%%%%%%%%%%%%%%%%%%%%%%%%%%%%%%%
