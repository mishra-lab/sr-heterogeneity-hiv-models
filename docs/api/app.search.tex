We designed our search strategy with guidance from
an information specialist at our affiliate library.
% ==============================================================================
\subsection{Search Terms}
\label{aa:search:terms}
Our search strategy and step-wise results are as follows, where
\st{[section]} refers to the result from another section,
\st{term/} denotes a MeSH term, and
\st{.mp} searches the main text fields, including
title, abstract, and heading words.
\begin{table}[H]
  \caption{Exclusion}
  \centering
  \searchsize
\begin{tabular}{R{.02}R{.07}L{.82}}
	\toprule
	  &       Term & Hits                             \\
	\midrule
	1 & \num{2190} & \st{[model] AND [hiv] AND [ssa]} \\
	2 & \num{2160} & \st{1 NOT animal/}               \\
	3 & \num{2155} & \st{limit 2 to english language} \\
	4 & \num{2125} & \st{limit 3 to yr="1860 - 2019"} \\
	5 & \num{1384} & \st{remove duplicates from 4}    \\
	\bottomrule
\end{tabular}
  \label{tab:search-exclude}
\end{table}
\begin{table}[H]
  \caption{Search Terms related to modelling (``model'')}
  \centering
  \def\modeladjterms{(model* ADJ3 (math* OR transmission OR dynamic* OR epidemi* OR compartmental OR deterministic OR individual OR agent OR network OR infectious disease* OR markov OR dynamic* OR simulat*)).mp.}
\searchsize
\begin{tabular}{R{.01}R{.06}L{.84}}
	\toprule
	  &          Hits & Term                            \\
	\midrule
	1 &  \num{238076} & \st{model, theoretical/}        \\
	2 &  \num{334921} & \st{model, biological/}         \\
	3 &  \num{302802} & \st{computer simulation/}       \\
	4 &  \num{196814} & \st{patient-specific modeling/} \\
	5 &   \num{67459} & \st{monte carlo method/}        \\
	6 &   \num{32801} & \st{exp stochastic processes/}  \\
	7 &  \num{455312} & \st{\modeladjterms}             \\
	8 & \num{1369153} & \st{OR/ 1-7}                    \\
	\bottomrule
\end{tabular}

  \label{tab:search-model}
\end{table}
\begin{table}[H]
  \caption{Search Terms related to HIV (``HIV'')}
  \centering
  \searchsize
\begin{tabular}{R{.02}R{.07}L{.82}}
	\toprule
	  &         Term & Hits                                                                                 \\
	\midrule
	1 & \num{290863} & \st{exp HIV/}                                                                        \\
	2 & \num{651624} & \st{exp HIV infections/}                                                             \\
	3 & \num{753274} & \st{(HIV OR HIV1* OR HIV2* OR HIV-1* OR HIV-2*).mp.}                                 \\
	4 & \num{369182} & \st{hiv infect*.mp.}                                                                 \\
	5 & \num{538214} & \st{(human immun*deficiency virus OR human immun* deficiency virus).mp.}             \\
	6 & \num{216228} & \st{exp Acquired Immunodeficiency Syndrome/}                                         \\
	7 & \num{235971} & \st{(acquired immun*deficiency syndrome OR acquired immun* deficiency syndrome).mp.} \\
	8 & \num{954470} & \st{OR/ 1-7}                                                                         \\
	\bottomrule
\end{tabular}
  \label{tab:search-hiv}
\end{table}
\begin{table}[H]
  \caption{Search Terms related to SSA (``SSA'')}
  \centering
  \searchsize
\begin{tabular}{R{.02}R{.07}L{.82}}
	\toprule
	   &         Term & Hits                                                                                      \\
	\midrule
	 1 &   \num{3512} & \st{Angola/ OR Angola.mp.}                                                                \\
	 2 &   \num{9273} & \st{Benin/ OR Benin.mp.}                                                                  \\
	 3 &   \num{5809} & \st{Botswana/ OR Botswana.mp.}                                                            \\
	 4 &   \num{9983} & \st{Burkina Faso/ OR Burkina Faso.mp.}                                                    \\
	 5 &   \num{2055} & \st{Burundi/ OR Burundi.mp.}                                                              \\
	 6 &  \num{16822} & \st{Cameroon/ OR Cameroon.mp.}                                                            \\
	 7 &   \num{1196} & \st{Cape Verde/ OR Cape Verde.mp.}                                                        \\
	 8 &  \num{15416} & \st{Central African Republic/ OR Central African Republic.mp. OR CAR.ti.}                 \\
	 9 &   \num{3075} & \st{Chad/ OR Chad.mp.}                                                                    \\
	10 &    \num{995} & \st{Comoros/ OR Comoros.mp.}                                                              \\
	11 &  \num{13737} & \st{Democratic Republic of the Congo/ OR Democratic Republic of the Congo.mp. OR DRC.mp.} \\
	12 &    \num{959} & \st{Djibouti/ OR Djibouti.mp.}                                                            \\
	13 &   \num{1131} & \st{Equatorial Guinea/ OR Equatorial Guinea.mp.}                                          \\
	14 &   \num{1437} & \st{Eritrea/ OR Eritrea.mp.}                                                              \\
	15 &  \num{35959} & \st{Ethiopia/ OR Ethiopia.mp.}                                                            \\
	16 &   \num{4500} & \st{Gabon/ OR Gabon.mp.}                                                                  \\
	17 &   \num{6626} & \st{Gambia/ OR Gambia.mp.}                                                                \\
	18 &  \num{25213} & \st{Ghana/ OR Ghana.mp.}                                                                  \\
	19 & \num{360920} & \st{Guinea/ OR Guinea.mp.}                                                                \\
	20 &   \num{2625} & \st{Guinea-Bissau/ OR Guinea-Bissau.mp.}                                                  \\
	21 &   \num{9730} & \st{Cote d'Ivoire/ OR Cote d'Ivoire.mp. OR Ivory Coast.mp.}                               \\
	22 &  \num{46917} & \st{Kenya/ OR Kenya.mp.}                                                                  \\
	23 &   \num{1649} & \st{Lesotho/ OR Lesotho.mp.}                                                              \\
	24 &   \num{4239} & \st{Liberia/ OR Liberia.mp.}                                                              \\
	25 &  \num{11386} & \st{Madagascar/ OR Madagascar.mp.}                                                        \\
	26 &  \num{16367} & \st{Malawi/ OR Malawi.mp.}                                                                \\
	27 &   \num{9111} & \st{Mali/ OR Mali.mp.}                                                                    \\
	28 &   \num{1573} & \st{Mauritania/ OR Mauritania.mp.}                                                        \\
	29 &   \num{2373} & \st{Mauritius/ OR Mauritius.mp.}                                                          \\
	30 &   \num{8502} & \st{Mozambique/ OR Mozambique.mp.}                                                        \\
	31 &   \num{3818} & \st{Namibia/ OR Namibia.mp.}                                                              \\
	32 &  \num{35455} & \st{Niger/ OR Niger.mp.}                                                                  \\
	33 &  \num{82192} & \st{Nigeria/ OR Nigeria.mp.}                                                              \\
	34 &  \num{13547} & \st{Republic of the Congo/ OR Republic of the Congo.mp. OR Congo-Brazzaville.mp.}         \\
	35 &   \num{1545} & \st{Reunion/}                                                                             \\
	36 &   \num{7597} & \st{Rwanda/ OR Rwanda.mp.}                                                                \\
	37 &    \num{342} & \st{"Sao Tome AND Principe"/ OR "Sao Tome AND Principe".mp.}                              \\
	38 &  \num{16674} & \st{Senegal/ OR Senegal.mp.}                                                              \\
	39 &   \num{1566} & \st{Seychelles/ OR Seychelles.mp.}                                                        \\
	40 &   \num{5456} & \st{Sierra Leone/ OR Sierra Leone.mp.}                                                    \\
	41 &   \num{4667} & \st{Somalia/ OR Somalia.mp.}                                                              \\
	42 & \num{114536} & \st{South Africa/ OR South Africa.mp.}                                                    \\
	43 &   \num{1193} & \st{South Sudan/ OR South Sudan.mp.}                                                      \\
	44 &  \num{21680} & \st{Sudan/ OR Sudan.mp.}                                                                  \\
	45 &   \num{2409} & \st{Swaziland/ OR Swaziland.mp. OR Eswatini/ OR Eswatini.mp.}                             \\
	46 &  \num{32442} & \st{Tanzania/ OR Tanzania.mp.}                                                            \\
	47 &   \num{3749} & \st{Togo/ OR Togo.mp.}                                                                    \\
	48 &  \num{37399} & \st{Uganda/ OR Uganda.mp.}                                                                \\
	49 &  \num{13506} & \st{Zambia/ OR Zambia.mp.}                                                                \\
	50 &  \num{15755} & \st{Zimbabwe/ OR Zimbabwe.mp.}                                                            \\
	51 & \num{482060} & \st{exp africa south of the sahara/ OR sub-saharan.mp. OR south of the sahara.mp.}        \\
	52 & \num{982505} & \st{OR/ 1-51}                                                                             \\
	\bottomrule
\end{tabular}
  \label{tab:search-ssa}
\end{table}
% ==============================================================================
\subsection{Inclusion/Exclusion Criteria}
\label{aa:search:criteria}
\begin{table}[H]
  \caption{Criteria for inclusion and exclusion}
  \centering
  \footnotesize
\begin{tabular}{ll}
	\toprule
	Include                                                         & Exclude                                                                  \\
	\midrule
	\multicolumn{2}{l}{\textbf{Publication Type}}                                                                                              \\
	\midrule
	\tabitem English language                                       & \tabitem non-English language                                            \\
	\tabitem published before 2020                                  & \tabitem published in or after 2020                                      \\
	\tabitem peer-reviewed journal article                          & \tabitem non-peer-reviewed article                                       \\
	                                                                & \tabitem review article\tn{1}                                            \\
	                                                                & \tabitem textbook, grey literature                                       \\
	                                                                & \tabitem opinions, comments, correspondence                              \\
	                                                                & \tabitem conference abstracts and proceedings                            \\
	                                                                & \tabitem model comparison study                                          \\
	\midrule
	\multicolumn{2}{l}{\textbf{Mathematical Modelling of HIV Transmission}}                                                                    \\
	\midrule
	\tabitem sexual HIV transmission model                          & \tabitem no sexual HIV transmission modelled                             \\
	\tabitem non-linear HIV transmission model\tn{2}                & \tabitem HIV transmission model is linear                                \\
	\tabitem population-level dynamics                              & \tabitem only within-host/cellular/protein modelling                     \\
	\tabitem compartmental model\tn{3}                              & \tabitem individual-based model                                          \\
	\midrule
	\multicolumn{2}{l}{\textbf{Context \& Objectives}}                                                                                         \\
	\midrule
	\tabitem any region in Sub-Saharan Africa (SSA)\tn{4}           & \tabitem only regions outside SSA modelled                               \\
	\tabitem assess prevention impact of ART scale-up for all\tn{5} & \tabitem only theoretical context modelled                               \\
	                                                                & \tabitem only individual-level benefits of ART modelled                  \\
	                                                                & \tabitem only prevention benefits of other interventions                 \\
	                                                                & \tabitem no base-case scenario reflecting status quo\tn{*}               \\
	                                                                & \tabitem only ART-combination interventions\tn{*}                        \\
	                                                                & \tabitem only ART intervention targeted to some risk groups\tn{*}        \\
	                                                                & \tabitem only ART prevention impacts reported for some risk groups\tn{*} \\
	                                                                & \tabitem ART prevention impacts not reported\tn{5*}                      \\
	\bottomrule
\end{tabular}
\floatfoot{
  \tnt[1]{Review articles were included if
    they also presented new HIV transmission modelling results fitting our criteria.}
  \tnt[2]{We define a \emph{non-linear model} as one where
    the number of infections projected at time $t$ is a function of
    the number of infections previously projected by the model before time $t$.}
  \tnt[3]{We define a \emph{compartmental model} as one where
    the system variables represent the numbers of individuals in each state,
    rather than unique individuals.}
  \tnt[4]{SSA was defined based on the countries in
    the UN regions of East, South, Central, and West Africa, plus South Sudan
    (see Table~\ref{tab:search-ssa} for full country list).}
  \tnt[5]{Articles reporting HIV incidence reduction and/or
    cumulative HIV infections averted among the whole population due to
    increased coverage or initiation rate of ART for the whole population.}
  \tnt[*]{Used to define Dataset~B only.}
}

  \label{tab:search-criteria}
\end{table}
\clearpage
% ==============================================================================
\subsection{Included Papers}
\label{aa:search:dataset}
\footnotesize
% ------------------------------------------------------------------------------
\subsubsection{Dataset~B}
\foreach \bibid in \bibidB{\pseudocite{\bibid} }
% ------------------------------------------------------------------------------
\subsubsection{Dataset~A less B}
\foreach \bibid in \bibidAxB{\pseudocite{\bibid} }
\par\normalsize