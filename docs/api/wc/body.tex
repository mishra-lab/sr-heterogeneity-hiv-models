

INTRO

Sub-Saharan Africa continues to bear the largest burden of HIV.
As of 2019, two thirds (25.7 million) of all people living with HIV globally are in Sub-Saharan Africa, where
an estimated one million new HIV infections were acquired in 2019.
Data suggest that key populations, such as individuals engaged in sex work and men who have sex with men experience
disproportionate risks of HIV acquisition and onward transmission in Sub-Saharan Africa.
HIV treatment to reduce onward transmission remains a key element of combination HIV prevention.
Effective HIV treatment with antiretroviral therapy (ART) leads to viral load suppression
and has been shown to prevent HIV transmission between sex partners.

Following empirical evidence of partnership-level efficacy of ART
in preventing HIV,
and model-based evidence of "treatment as prevention",
several large-scale community-based trials of universal test-and-treat (UTT)
have recently been completed.
These trials found that over 2 to 4 years,
cumulative incidence under UTT did not significantly differ from
cumulative incidence under ART according to national guidelines.
Thus the population-level reductions in incidence anticipated from transmission modelling
were not observed in the trials.

One theme in the proposed explanations for limited population-level ART effectiveness
was heterogeneity in intervention coverage and its intersection with
heterogeneity in transmission risks.
While viral suppression improved under UTT in all three trials,
21--54\% of study participants remained unsuppressed.
It has been suggested that populations experiencing barriers to viral suppression under UTT
may be at highest risk for onward transmission, including key populations like
female sex workers, men who have sex with men, and adolescent girls and young women.
While widespread UTT scale-up may fill some of these coverage gaps,
equitable access to ART for marginalized populations remains an open challenge.

Given the upstream and complementary role of transmission modelling
in estimating the impact of ART as prevention,
we sought to critically appraise and examine the type and scope of risk heterogeneity captured
in mathematical models used to assess the prevention impacts of ART in Sub-Saharan Africa.
We conducted a scoping review with the following objectives.

RQS

Among non-linear compartmental models of sexual HIV transmission
that have been used to simulate the prevention impacts of ART in Sub-Saharan Africa:

      In which epidemic contexts (geographies, populations, epidemic phases)
      have these models been applied?

      How was the model structured to represent key factors of risk heterogeneity?

      What are the potential influences of representations of risk heterogeneity
      on the projected prevention benefits of ART for all?

METHODS

We conducted a scoping review according to the PRISMA extension for scoping reviews
(see Appendix REF for checklist).
First, we developed a conceptual framework to organize
the assumptions and representations of risk heterogeneity
in compartmental HIV transmission models.
Then, we designed and implemented the search strategy,
and extracted the data relevant to the framework
to address the objectives.

Conceptual Framework for Risk Heterogeneity

We conceptualized "factors of risk heterogeneity", meaning
epidemiological stratifications and phenomena which may/not be included in transmission models.
Such factors could include if/how populations, rates, and probabilities
are stratified along health, structural, and behavioural dimensions.
We defined the following 4 domains in which
different factors of risk heterogeneity might influence the transmission impact of ART.

   Biological Effects:
  differential transmission risk within HIV disease course
  that may coincide with differential ART coverage
   Behaviour Change Effects:
  differential transmission risk due to
  behavioural changes related to engagement in the ART cascade
   Network Effects:
  differential transmission risk within sub-populations
  that increase the challenge of epidemic control through core group dynamics
   Coverage Effects:
  differential transmission risk within sub-populations
  who also experience barriers to engaging in ART care and achieving viral suppression,
  such as youth and key populations

We then compiled a list of key factors of risk heterogeneity,
and their possible mechanisms of influence on ART prevention impact (Table REF).

Search

We searched MEDLINE and EMBASE via Ovid
using search terms related to Sub-Saharan Africa (SSA), HIV, and transmission modelling
(Appendix REF).
Search results were imported into Covidence for screening.
Duplicate studies were removed automatically by Ovid and by Covidence;
four additional duplicates that were not identified automatically,
possibly due to subtly different titles, were later identified and removed manually.
Potentially relevant studies were identified by title and abstract screening, followed 
by full-text screening using the inclusion/exclusion criteria listed below. 
One reviewer (JK) conducted the search, screening, and data extraction.

Inclusion/Exclusion Criteria

We included peer-reviewed, primarily modelling studies that used non-linear models of sexual HIV transmission
to project the prevention impacts of ART in any one or more settings within SSA.
Complete inclusion/exclusion criteria are given in Appendix REF.
We only included studies communicated in English and published anytime before Jan 1, 2020.
We excluded publications without primary modelling results and description of the methods,
such as commentaries and reviews.
We excluded conference publications.
If a model's details were provided in a separate peer-reviewed publication,
we extracted data from both publications as required.

The following criteria were used for inclusion of studies:
1) Used a non-linear compartmental model of
sexual HIV transmission at the population level.
We defined a non-linear model as one where
the number of infections projected at time t is a function of
the number of infections previously projected by the model before time t.
We defined a compartmental model as one where
the system variables represent the numbers of individuals in each state,
rather than unique individuals.
Thus, statistical models, linear models, and individual-based models without dynamic transmission were excluded.
2) The model was calibrated or parameterized to reflect at least one setting within SSA
(see Appendix REF for full country list).
3) The study simulated at least one scenario with increasing ART coverage,
possibly alongside scale-up of other interventions.
The included studies formed Dataset A,
used to complete objectives REF and REF.

A subset of Dataset A formed Dataset B,
which used to complete objective REF.
Studies in Dataset B specifically examined
scale-up of ART coverage alone (versus combination intervention)
for the whole population (versus ART prioritized to subgroups),
and reported HIV incidence reduction or cumulative HIV infections averted over time 
as compared to a base-case scenario reflecting the status quo.

Data Extraction

Data extraction used the full text and all available supplementary material.
For objectives REF and REF, data were extracted per-study.
For objective REF, data were extracted per-scenario within the study.
Variables definitions are given in Appendix REF.

Epidemic Context

For objective REF, we extracted data on
geography, epidemic phase, and key populations explicitly considered in the model.
We categorized studies by the geographic location (country and SSA region),
scale of the simulated population (city, sub-national, national, regional), and
whether the study included multiple geographic settings.
We classified epidemic size at time of ART intervention using
the overall HIV prevalence at that time (low: <1\%, medium: 1-10\%, high: >10\%),
and classified epidemic phase at the time of ART intervention
using the trend in incidence at that time
(increasing, increasing but stabilizing, stable/equilibrium, decreasing but stabilizing, and decreasing).

We extracted whether any of the following key populations were included in the model:
female sex workers (FSW);
male clients of FSW (Clients);
men who have sex with men (MSM); and
people who inject drugs (PWID).
FSW were defined as any female activity group meeting 3 criteria:
{<5\%} of the female population;
{<1/3} the size of the client population; and
having {>50x} the partners per year of
the lowest sexually active female activity group.
Likewise, clients were defined as any male activity group
described as clients of FSW and being {>3x} the size of the FSW population.
We also extracted whether any groups in the model were described as MSM or PWID.

Factors of Risk Heterogeneity

For objective REF, we examined if and how
the factors of risk heterogeneity outlined in Table REF
were simulated in each study.
Special focus was given to the factors related to Network and Coverage Effects,
due to the large variability in how these factors were simulated.
We examined the number of risk groups defined by sex and/or sexual activity, and
any turnover of individuals between activity groups and/or key populations.

We extracted whether multiple partnership types were simulated,
and classified how such partnerships were defined:
generic (all partnerships equal);
based only on the activity groups involved;
or overlapping, such that different partnership types could be formed between the same two activity groups.
We extracted whether partnerships considered different
numbers of sex acts (sex frequency and partnership duration) and levels of condom use.
We extracted whether models simulated any degree of assortative mixing by activity groups
(preference for like-with-like) versus proportionate (random mixing). 
The number of unique age groups was extracted, as well as
whether mixing by age groups was
proportionate, strictly assortative, or assortative with age differences.
We extracted whether age conferred any additional risk beyond mixing,
such as higher rates of partnership formation.

Finally, we extracted whether rates of HIV diagnosis, ART initiation, and/or ART discontinuation
differed across modelled risk strata (sex, activity, key populations, and/or age),
and if so, how they differed.

ART Prevention Impact

For objective REF, we examined the subset of studies (Dataset B)
reporting incidence reduction and/or infections averted due to
a population-wide ART scale-up intervention.
We extracted the following data for each intervention scenario within Dataset B:
the years that ART scale-up started (\tx{0}) and stopped (\tx{f});
the final overall ART coverage achieved and/or
the final ART initiation rate (per person-year among PLHIV not yet in care);
the criteria for ART initiation (e.g.\ CD4 count);
and the relative reduction in transmission probability on ART.
Then, we extracted the
reduction in incidence relative to the base-case scenario, and/or
proportion of infections averted relative to the base-case scenario
for available time horizons relative to \tx{0}.

Finally, for each factor of heterogeneity,
we compared the projected ART impacts (incidence reduction and infections averted)
across the different factor levels (whether or not, and how the factor was modelled).
We plotted the impacts versus time since \tx{0}, stratified by factor levels,
and explored whether the distribution of impacts
was the same under all factor levels.
We tested for significant differences using a Kruskal-Wallis test.

RESULTS

The search yielded 1384 publications,
of which 94 studies were included
(Figure REF).
360 studies used non-linear HIV transmission models applied to SSA, of which
255 were compartmental models. 
94 compartmental modeling studies simulated ART scale-up (Database A), of which
40 reported infections averted or incidence reduction
due to population-wide ART scale-up without additional combination interventions,
as compared to a base case reflecting status quo (Database B).
Appendix REF lists the included papers, and
Appendix REF provides additional results.

ART prevention impacts were most often modelled in
high-prevalence epidemics ({>10\%} HIV prevalence, X studies) and
medium-prevalence epidemics ({1-10\%}, X studies).
No studies reported overall HIV prevalence of {<1\%} at time of ART scale-up,
although for X studies, HIV prevalence was either
not reported or varied across independently simulated contexts/scenarios.
The \xdmdef year of scenario ART scale-up was \xdm{t0/t0}; at which time
HIV prevalence (\%) was \xdm{t0/prev}; and
incidence (per 1000 PY) was \xdm{t0/inc}.
Most contexts reporting incidence trends had decreasing or stable incidence
(45 of 48 reporting). 

Key Populations

Groups representing FSW were described in X studies.
Among these (of studies where it was possible to evaluate):
X (of X) were {<5\%} of the female population;
X (of X) were {<1/3} the size of the client population; and
X (of X) had {>50x} the partners per year of
the lowest sexually active female activity group.
In X studies at least one criteria was not met, and
in X all three were met.
Clients of FSW were modelled as a unique group in X studies,
among which X (of X reporting)
were {>3x} the size of the FSW population.
In another X, studies clients were defined as a proportion of another group,
among which X (of X)
were {>3x} the size of the FSW population.
Activity groups described as representing
men who have sex with men (MSM) were noted in X studies; and
people who inject drugs (PWID) in X.

Heterogeneity Factors

Biological Effects

The \xdmdef number of states used to represent HIV disease
(ignoring treatment-related stratifications) was \xdm{hiv/hiv.n},
and X studies represented HIV along a continuous dimension
using a partial differential equations model.
Most HIV states were defined by CD4 count
to reflect clinical progression and/or historical ART eligibility,
often with additional states to represent acute infection and/or development of AIDS.
States of increased infectiousness associated with acute infection and late stage disease
were simulated in X and X studies, respectively.

The relative risk of HIV transmission on ART was \xdm{art/rbeta},
representing an average "on-treatment" state in X studies,
versus a "virally suppressed" state specifically in X studies.
Treatment failure due to drug resistance was simulated in X studies, including:
X in which individuals experiencing treatment failure
were tracked separately from ART-naive; and
X in which such individuals
transitioned back to a generic "off-treatment" state.
Another X studies included a similar transition
that was not clearly identified as treatment failure versus ART cessation.
Transmissible drug resistance was simulated in X studies.

Behavioural Effects

Reduced sexual activity during late-stage HIV symptoms was simulated in X studies,
including at least one state with:
complete cessastion of sexual activity (X);
reduced rate or number of partnerships (X); and/or
reduced rate or number of sex acts per partnership (X).

Separate health states representing diagnosed HIV but not yet on treatment
and on treatment but not yet virally suppressed were simulated in
X and X studies, respectively.
X studies modelled changes in behaviour following awareness of HIV status among PLHIV:
increased condom use (X);
fewer partners per year (X);
fewer sex acts per partnership (X);
serosorting (X); and/or
a generic reduction in transmission probability (X).

ART cessation was simulated in X studies, including:
X in which individuals no longer on ART
were tracked separately from ART-naive; and
X in which such individuals
transitioned back to a generic "off-treatment" state.
Another X studies included a similar transition
that was not clearly identified as treatment failure versus ART cessation.

Network Effects

Representations of risk heterogeneity varied widely.
Risk groups defined at least in part by activity
(different rates and/or types of partnerships formed) were simulated in X studies,
and at least in part by sex in X studies.
The number of simulated groups defined by sex and/or activity was \xdm{act/act.n};
the number defined by activity alone (maximum number of groups among either
heterosexual women, heterosexual men, MSM, or overall if sex was not considered) was \xdm{act/act.n.z}.
The highest activity groups (including FSW and clients, where applicable) for females and males comprised
\xdm{act/hrw.p} and \xdm{act/hrm.p} \% of female and male populations, respectively.

Turnover between activity groups and/or key populations
was considered in X studies,
of which X considered turnover of only
one specific high-activity group or key population.
Another X studies simulated
movement only from lower activity groups into higher activity groups
to re-balance group sizes against disproportionate HIV mortality in higher activity groups.

Among X studies with activity groups, sexual mixing was assumed to be
assortative in X and proportionate in X.
Regarding the three approaches to partnership types:
First, partnerships were considered to have equal probability of transmission in
X studies, including all studies without activity groups.
Second, partnerships were defined by the activity groups involved (X studies).
In such partnerships, transmission was usually
lower in high-with-high activity partnerships than in low-with-low, due to a combination of
fewer sex acts (X) and
increased condom use (X).
The transmission risk in mixed high-with-low activity partnerships was defined by:
the susceptible partner (X);
the lower activity partner (X);
the higher activity partner (X); or
the unique combination of both partners' activity groups (X);
such definitions yielded indeterminate, higher, lower, or intermediate
per-partnership transmission risk, respectively.
Third, partnerships could be defined based on overlapping types
(main/spousal, casual, and sex work), such that
different partnership types could be formed between the same two activity groups (X studies).
For example, FSW could form commercial, casual, or spousal/main partnerships with clients.
All models with overlapping partnership types defined differential total sex acts and condom use between types.

Age groups were simulated in X studies.
Among studies with age groups, the number of age groups was \xdm{age/age.n},
and X studies simulated age along a continuous dimension.
Sexual mixing between age groups was assumed to be assortative
either with (X) or without (X)
average age differences between men and women;
or proportionate (X).
Differential risk behaviour by age occurred in X of these X studies.

Coverage Effects

Differential transition rates along the ART cascade was considered in
X studies, including differences between
sexes in X;
age groups in X; and
key populations in X.
No studies considered differences in cascade transition rates
among activity groups beyond key populations. 
Another X studies did not simulate differential cascade transitions,
but specifically justified the simplification using data relevant to the simulated context.
Differences between sexes included rates of
HIV diagnosis (X);
ART initiation (X); and
ART cessation (X),
with cascade engagement higher among women,
in most cases attributed to antenatal services.
Differences between age groups also affected
rates of diagnosis (X);
ART initiation (X);
but not ART cessation (X). 
Among key populations, lower rates of
diagnosis, ART initiation, and retention were simulated in
X, X, and X
studies respectively, while higher rates were simulated in
X, X, and X.

ART Prevention Impact

Dataset B comprised X studies,
including X scenarios of ART scale-up.
Relative incidence reduction with ART scale-up
as compared to a scenario without ART scale-up
was reported in X studies (X scenarios);
the proportion of cumulative infections averted due to ART scale-up
was reported in X (X);
and X (X) reported both.
Some scenarios reported these outcomes on multiple time horizons.
Reported impact on incidence ranged from 
93\% reduction over 10 years to
14\% increase over 15 years,
while impact on cumulative infections ranged from
78\% reduction over 10 years to
12\% increase over 5 years.

Figure REF summarizes each outcome versus time since ART scale-up,
stratified by a composite index of modelled risk heterogeneity.
Ecological-level analysis across scenarios by degree of risk heterogeneity
identified differences in proportions of infections averted,
but not in relative incidence reduction (Table REF).
The largest proportions of infections averted were reported from 
scenarios without risk heterogeneity 
(median [IQR]\% = \xd{api/chi/Risk.None}), followed by scenarios
with key populations prioritized for ART (\xd{api/chi/Risk.KP-(priority)}).
The smallest impact was observed in scenarios with
key populations who were not prioritized for ART (\xd{api/chi/Risk.KP-(same)})
and in models with some risk heterogeneity but without key populations
(\xd{api/chi/Risk.Activity-(No-KP)}).
Only X scenarios from X studies provided both outcomes; 
within which the pattern of incidence reduction versus modelled heterogeneity
was generally similar to the pattern of infections averted versus modelled heterogeneity
(Figure REF).

Appendix REF and Table REF summarizes
the reported ART prevention impacts (relative incidence reduction and proportion of infections averted),
stratified by: other factors of risk heterogeneity; epidemic contexts; and intervention conditions.
For example, reported incidence reduction and proportion of infections averted
were both larger with longer time horizon, greater ART eligibility, and higher ART coverage.

DISCUSSION

Via scoping review, we found that representations of risk heterogeneity varied widely across
transmission modelling studies of ART in SSA, with
stratification by sexual activity and key populations considered in approximately
two thirds and one third of models, respectively.
We also found that the projected proportions of infections averted due to ART scale-up were
larger under assumptions of homogeneous risk or prioritized ART to key populations,
as compared to heterogeneous risk or without prioritized ART to key populations.
Three notable themes emerged from our review.

First, modelling studies have an opportunity to keep pace with growing epidemiological data on risk heterogeneity.
For example, 41\% of the modelling studies reviewed included at least one key population,
such as female sex workers and men who have sex with men.
Key populations continue to experience disproportionate risk of HIV, even in high-prevalence epidemics,
and models examining the unmet needs of such key populations have suggested that
these unmet needs play an important role in the overall epidemic dynamics.
Furthermore, the we found that the number of modelled clients per female sex worker, and
the relative rate of partnership formation among female sex workers versus other women
did not always reflect the available data.
Similarly, among studies with different partnership types, only 20\% modelled
main/spousal partnerships---with more sex acts and lower condom use---between two higher risk individuals,
while 80\% modelled only casual or commercial partnerships among higher risk individuals.
However, data suggest female sex workers may form main/spousal partnerships
with regular clients and boyfriends/spouses from higher risk groups.
Thus, with increasing data available on various factors of risk heterogeneity,
there are opportunities to include these data in future models,
and to study how multiple factors might act together to influence projections of ART impact
through nested model comparison studies.

Second, most models assumed that rates ART cascade transitions,
including diagnosis, ART initiation, and retention,
are similar or equivalent across subgroups.
Recent data suggest differential ART cascade by sex, age, and key populations.
These differences may stem from the unique needs of population subgroups
and is one reason why differentiated ART services are a core component of HIV programs.
Moreover, barriers to ART may intersect with transmission risk, particularly among key populations,
due to issues of stigma, discrimination, and criminalization.
Thus, further opportunities exist to incorporate differentiated ART cascade data,
to examine the intersections of intervention and risk heterogeneity, and
to consider the impact of HIV services as they are delivered on the ground.
Similar conclusions were noted by \citet{Case2019}
in the context of modelling pre-exposure prophylaxis in SSA.
Finally, depending on the research question, the modelled treatment cascade may need expansion
to include more cascade steps and states related to treatment failure/discontinuation.

Third, based on ecological analysis of scenarios, we found that
modelling assumptions about risk and intervention heterogeneity
may influence the projected proportion of future infections averted by ART.
We did not find similar evidence for relative incidence reduction due to ART,
but studies reporting these two outcomes were largely distinct.
Among studies reporting both, the overall pattern was consistent.
These findings highlight the limitations of ecological analysis to estimate
the potential influence of modelling assumptions on projected ART prevention benefits,
and motivate additional systematic model comparison studies to better quantify this influence,
such as those by \citet{Dodd2010,Hontelez2013}.
Our ecological analysis also suggested that the anticipated ART prevention impacts from homogeneous models
may be achievable in the context of risk heterogeneity
if testing and treatment resources are prioritized to higher risk groups.

Limitations of our scoping review include our examination of only a few key populations.
In our conceptual framework for risk heterogeneity, we did not explicitly examine heterogeneity
by type of sex act (i.e. anal sex) which is associated with higher probability of HIV transmission,
nor structural factors such as violence which lead to increased HIV risks.
The large number of differences between scenarios in the scoping review context
also limited our ability to infer the influence of risk heterogeneity across scenarios.

In conclusion, representations of risk heterogeneity vary widely
among models used to project the prevention impacts of ART in SSA.
Such differences may partially explain the large variability in projected impacts.
Opportunities exist to incorporate new and existing data on
the intersections of risk heterogeneity and intervention heterogeneity.
Moving forward, systematic model comparison studies are needed to
estimate and understand the relative influence of various modelling assumptions on the prevention impacts of ART.
