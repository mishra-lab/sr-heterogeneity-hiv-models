

INTRO

As of 2019, two thirds (25.7 million) of all people living with HIV globally
were in Sub-Saharan Africa (SSA), where
an estimated one million new HIV infections were acquired in 2019.
Data suggest that key populations, including individuals engaged in sex work and men who have sex with men,
experience disproportionate risks of HIV acquisition and onward transmission in SSA.
HIV treatment to reduce onward transmission remains a key element of combination HIV prevention.
Effective HIV treatment with antiretroviral therapy (ART) leads to viral load suppression
and has been shown to prevent HIV transmission between sex partners.

Following empirical evidence of partnership-level efficacy of ART
in preventing HIV,
and model-based evidence of "treatment as prevention",
several large-scale community-based trials of universal test-and-treat (UTT)
were recently completed.
These trials found that over 2-to-4 years,
cumulative incidence under UTT did not significantly differ from
cumulative incidence under ART according to national guidelines.
Thus the population-level reductions in incidence anticipated from transmission modelling
were not observed in these trials.

One theme in the proposed explanations for limited population-level ART prevention effectiveness
was heterogeneity in intervention coverage and its intersection with
heterogeneity in transmission risks.
While viral suppression improved under UTT in all three trials,
21--54\% of study participants remained unsuppressed.
It has been suggested that populations experiencing barriers to viral suppression under UTT
may be at highest risk for onward transmission, including key populations like
sex workers, men who have sex with men, and adolescent girls and young women.
While widespread UTT scale-up may fill some coverage gaps,
equitable access to ART for marginalized populations remains an open challenge.

Given the upstream and complementary role of transmission modelling
in estimating the prevention impacts of ART,
we sought to examine and appraise representations of risk heterogeneity
in mathematical models used to assess the prevention impacts of ART in SSA.
We conducted a scoping review with the following objectives.

RQS

Among non-linear compartmental models of sexual HIV transmission
that have been used to simulate the prevention impacts of ART in SSA:

      In which epidemic contexts (geographies, populations, epidemic phases)
      have these models been applied?

      How was the model structured to represent key factors of risk heterogeneity?

      What are the potential influences of representations of risk heterogeneity
      on the projected prevention benefits of ART for all?

METHODS

We conducted a scoping review according to the PRISMA extension for scoping reviews
(Appendix REF).
First, we developed a conceptual framework to organize
assumptions and representations of risk heterogeneity
in compartmental HIV transmission models.
Then, we designed and implemented the search strategy,
and extracted data relevant to the framework to address our objectives.

Conceptual Framework for Risk Heterogeneity

We conceptualized "factors of risk heterogeneity", meaning
epidemiological stratifications and phenomena which may/not be included in transmission models.
Such factors could include if/how populations, rates, and probabilities
are stratified along various dimensions.
We defined the following 4 domains in which
different factors of risk heterogeneity might influence the transmission impact of ART.

   Biological Effects:
  differential transmission risk within HIV disease course
  that may coincide with differential ART coverage
   Behaviour Change Effects:
  differential transmission risk due to
  behavioural changes related to engagement in the ART cascade
   Network Effects:
  differential transmission risk within sub-populations
  that increases the challenge of epidemic control through core group dynamics
   Coverage Effects:
  differential transmission risk within sub-populations
  who experience barriers to ART care and achieving viral suppression,
  such as youth and key populations

We then compiled a list of key factors of risk heterogeneity,
and their possible mechanisms of influence on ART prevention impact (Table REF).

Search

We searched MEDLINE and EMBASE via Ovid
using search terms related to Sub-Saharan Africa (SSA), HIV, and transmission modelling
(Appendix REF).
Search results were imported into Covidence for screening.
Duplicate studies were removed automatically by Ovid and by Covidence;
four additional duplicates with subtly different titles were later identified and removed manually.
Potentially relevant studies were identified by title and abstract screening, followed 
by full-text screening using the inclusion/exclusion criteria below. 
One reviewer (JK) conducted the search, screening, and data extraction.

Inclusion/Exclusion Criteria

We included peer-reviewed, primarily modelling studies that used non-linear models of sexual HIV transmission
to project the prevention impacts of ART in any setting within SSA.
See Appendix REF for complete inclusion/exclusion criteria.
We only included studies published in English anytime before Jan 1, 2020.
We excluded conference publications and those without primary modelling results and description of the methods,
such as commentaries and reviews.
If a model's details were provided in another peer-reviewed publication,
we extracted data from both publications as required.

The following criteria were used for inclusion of studies:
1) Used a non-linear compartmental model of sexual HIV transmission at the population level.
We defined a non-linear model as one where
future projected infections are a function of previously projected infections,
and a compartmental model as one where
system variables represent the numbers of individuals in each state,
rather than unique individuals.
Thus, statistical models, models without dynamic transmission, and individual-based models were excluded.
2) The model was parameterized/calibrated to reflect at least one setting within SSA
(see Appendix REF for countries).
3) The study simulated at least one scenario with increasing ART coverage,
possibly alongside other interventions.
The included studies formed Dataset A,
used to complete objectives REF and REF.

A subset of Dataset A formed Dataset B,
used to complete objective REF.
Studies in Dataset B specifically examined
scale-up of ART coverage alone (versus combination intervention)
for the whole population (versus ART prioritized to subgroups),
and reported HIV incidence reduction or cumulative HIV infections averted over time 
relative to a base-case scenario reflecting status quo.

Data Extraction

Data extraction used the full text and all available supplementary material.
Data were extracted per-study for objectives REF and REF, and
per-scenario for objective REF.
Detailed variables definitions are given in Appendix REF.

Epidemic Context

For objective REF, we extracted data on
geography, epidemic phase, and key populations explicitly considered in the model.
We categorized studies by country and SSA region, and
scale of the simulated population (city, sub-national, national, regional).
We classified epidemic size at time of ART intervention using
overall HIV prevalence (low: <1\%, medium: 1-10\%, high: >10\%),
and epidemic phase using overall HIV incidence trend
(increasing, increasing-but-stabilizing, stable/equilibrium, decreasing-but-stabilizing, and decreasing).

We extracted whether any of the following key populations were modelled:
female sex workers (FSW);
male clients of FSW (Clients);
men who have sex with men (MSM); and
people who inject drugs (PWID).
FSW were defined as any female activity group meeting 3 criteria:
{<5\%} of the female population;
{<1/3} the client population size; and
having {>50x} the partners per year of
the lowest sexually active female activity group.
Clients were defined as any male activity group
described as clients of FSW, and being {>3x} the FSW population size.
We also extracted whether any groups in the model were described as MSM or PWID.

Factors of Risk Heterogeneity

For objective REF, we examined if/how
the factors of risk heterogeneity outlined in Table REF
were simulated in each study.
We examined the number of risk groups defined by sex and/or sexual activity, and
any turnover of individuals between activity groups and/or key populations.

We classified how partnership types were defined:
generic (all partnerships equal);
based only on the activity groups involved;
or overlapping, such that different partnership types could be formed between the same two activity groups.
We extracted whether partnerships considered different
numbers of sex acts (sex frequency and partnership duration) and condom use,
and whether models simulated any degree of assortative mixing by activity groups
(preference for like-with-like) versus proportionate (random) mixing.
The number of age groups was extracted, and whether mixing by age groups was
proportionate, strictly assortative, or assortative with age differences.
We extracted whether age conferred any transmission risk beyond mixing,
such as different partnership formation rates.

Finally, we extracted whether rates of HIV diagnosis, ART initiation, and/or ART discontinuation
differed across risk strata (sex, activity, key populations, and/or age),
and if so, how they differed.

ART Prevention Impact

For objective REF, we extracted
the following data for each intervention scenario within Dataset B:
the years that ART scale-up started (\tx{0}) and stopped (\tx{f});
the final overall ART coverage achieved and/or
the final ART initiation rate (per person-year among PLHIV not yet in care);
the criteria for ART initiation (e.g.\ CD4 count); and
the relative reduction in transmission probability on ART.
Then, we extracted the
relative reduction in incidence and/or proportion of infections averted
relative to the base-case scenario for available time horizons relative to \tx{0}.

Finally, for each factor of heterogeneity,
we compared projected ART impacts (incidence reduction/infections averted)
across different factor levels (whether or not, and how the factor was modelled).
We plotted the impacts versus time since \tx{0}, stratified by factor levels,
and explored whether projected impacts were the same under all factor levels,
testing for significant differences using a Kruskal-Wallis test.

RESULTS

The search yielded 1384 publications,
of which 94 studies were included (Figure REF).
These studies (Dataset A) applied non-linear compartmental modelling to simulate ART scale-up in SSA,
of which 40 reported infections averted/incidence reduction
due to population-wide ART scale-up without combination intervention,
relative to a base-case reflecting status quo (Dataset B).
Appendix REF lists the included papers, and
Appendix REF provides additional results.

Epidemic Context

Table REF summarizes key features of contexts within SSA
where the prevention impacts of ART have been modelled.
Most (X) of the 94 studies modelled HIV transmission at the national level;
studies also explored
regional (X),
sub-national (X), and
city-level (X) epidemic scales.
South Africa was the most common country simulated (X studies), and
Figure REF illustrates the number of studies by country.
East Africa was the most represented SSA region, being simulated in X studies,
followed by Southern (X), West (X), and Central Africa (X).

ART prevention impacts were most often modelled in
high-prevalence ({>10\%}) epidemics (X studies) and
medium-prevalence ({1-10\%}) epidemics, X).
No studies reported overall HIV prevalence of {<1\%} at time of intervention,
although for X studies, HIV prevalence was either
not reported or varied across simulated contexts/scenarios.
The \xdmdef year of intervention was \xdm{t0/t0}; at which time
HIV prevalence (\%) was \xdm{t0/prev}; and
incidence (per 1000 person-years) was \xdm{t0/inc}.
Most reported incidence trends were decreasing or stable (45 of 48 reporting). 

Key Populations

Groups representing FSW were described in X studies.
Among these (of studies where it was possible to evaluate):
X (of X) were {<5\%} of the female population;
X (of X) were {<1/3} the size of the client population; and
X (of X) had {>50x} partners per year versus
the lowest sexually active female activity group.
Clients of FSW were modelled as a unique group in X studies,
among which X (of X reporting)
were {>3x} the size of the FSW population.
In another X studies, clients were defined as a proportion of another group,
among which X (of X)
were {>3x} the FSW population size.
Activity groups representing
men who have sex with men (MSM) were noted in X studies; and
people who inject drugs (PWID) in X.

Heterogeneity Factors

Biological Effects

The \xdmdef number of states used to represent HIV disease
(ignoring treatment-related stratifications) was \xdm{hiv/hiv.n},
and X studies represented HIV along a continuous dimension
using partial differential equations.
States of increased infectiousness associated with acute infection and late-stage disease
were simulated in X and X studies, respectively.

The relative risk of HIV transmission on ART was \xdm{art/rbeta},
representing an average "on-treatment" state in X studies,
versus a "virally suppressed" state in X.
Treatment failure due to drug resistance was simulated in X studies, including:
X where individuals experiencing treatment failure
were tracked separately from ART-naive; and
X where such individuals
transitioned back to a generic "off-treatment" state.
Another X studies included a similar transition
that was not identified as treatment failure versus ART cessation.
Transmissible drug resistance was simulated in X studies.

Behavioural Effects

Reduced sexual activity during late-stage HIV was simulated in X studies,
including at least one state with:
complete cessastion of sexual activity (X);
reduced rate/number of partnerships (X); and/or
reduced rate/number of sex acts per partnership (X).

Separate health states representing diagnosed HIV before treatment,
and on-treatment before viral suppression were simulated in
X and X studies, respectively.
X studies modelled behaviour changes following awareness of HIV+ status, including:
increased condom use (X);
fewer partners per year (X);
fewer sex acts per partnership (X);
serosorting (X); and/or
a generic reduction in transmission probability (X).

ART cessation was simulated in X studies, including:
X where individuals no longer on ART
were tracked separately from ART-naive; and
X where such individuals
transitioned back to a generic "off-treatment" state.
Another X studies included a similar transition
that was not identified as treatment failure versus ART cessation.

Network Effects

Populations were stratified by activity (different rates and/or types of partnerships formed)
in X studies, and by sex in X.
The number of groups defined by sex and/or activity was \xdm{act/act.n};
and by activity alone (maximum number of groups among:
heterosexual women, heterosexual men, MSM, or overall if sex was not considered) was \xdm{act/act.n.z}.
The highest activity groups for females and males (possibly including FSW/clients) comprised
\xdm{act/hrw.p} and \xdm{act/hrm.p}\% of female and male populations, respectively.

Turnover between activity groups and/or key populations
was considered in X studies,
of which X considered turnover of only
one specific high-activity group or key population.
Another X studies simulated
movement only from lower to higher activity groups
to re-balance group sizes against disproportionate HIV mortality.

Among X studies with activity groups, sexual mixing was assumed to be
assortative in X and proportionate in X.
Partnerships had equal probability of transmission in X studies,
including all studies without activity groups.
Partnerships were defined by the activity groups involved in X studies,
among which transmission was usually
lower in high-with-high activity partnerships than in low-with-low, due to
fewer sex acts (X) and/or increased condom use (X).
Transmission risk in high-with-low activity partnerships was defined by:
the susceptible partner (X);
the lower activity partner (X);
the higher activity partner (X); or
both partners' activity groups (X);
yielding indeterminate, higher, lower, or intermediate
per-partnership transmission risk, respectively.
Partnerships were defined based on overlapping types, such that
different partnership types could be formed between the same two activity groups in X studies.
All models with overlapping partnership types defined differential total sex acts and condom use between types.

Age groups were simulated in X studies, among which,
the number of age groups was \xdm{age/age.n},
and X studies simulated age along a continuous dimension.
Sexual mixing between age groups was assumed to be assortative
either with (X) or without (X)
average age differences between men and women;
or proportionate (X).
Differential risk behaviour by age was modelled in X studies.

Coverage Effects

Differential transition rates along the ART cascade were considered in
X studies, including differences between
sexes in X;
age groups in X; and
key populations in X.
Another X studies did not simulate differential cascade transitions,
but specifically justified the decision using context-specific data.
Differences between sexes included rates of
HIV diagnosis (X);
ART initiation (X); and
ART cessation (X),
with cascade engagement higher among women,
in most cases attributed to antenatal services.
Differences between age groups also affected
rates of diagnosis (X);
ART initiation (X);
but not ART cessation (X). 
Among key populations, lower rates of
diagnosis, ART initiation, and retention were simulated in
X, X, and X
studies respectively, while higher rates were simulated in
X, X, and X.

ART Prevention Impact

Dataset B comprised X studies,
including X scenarios of ART scale-up.
Relative incidence reduction with ART scale-up
as compared to a scenario without ART scale-up
was reported in X studies (X scenarios);
the proportion of cumulative infections averted due to ART scale-up
was reported in X (X);
and X (X) reported both.
Some scenarios reported these outcomes on multiple time horizons.

Figure REF summarizes each outcome versus time since ART scale-up,
stratified by a composite index of modelled risk heterogeneity.
Ecological-level analysis across scenarios by degree of risk heterogeneity
identified differences in proportions of infections averted,
but not in relative incidence reduction (Table REF).
The largest proportions of infections averted were reported from 
scenarios without risk heterogeneity (median [IQR]\% = \xd{api/chi/Risk.None}), followed by
scenarios with key populations prioritized for ART (\xd{api/chi/Risk.KP-(priority)}).
The smallest impact was observed in scenarios with
key populations who were not prioritized for ART (\xd{api/chi/Risk.KP-(same)})
and in models with risk heterogeneity but without key populations
(\xd{api/chi/Risk.Activity-(No-KP)}).
Only X scenarios from X studies provided both outcomes; 
among which the pattern of incidence reduction versus modelled heterogeneity
was similar to the pattern of infections averted versus modelled heterogeneity
(Figure REF).

Appendix REF and Table REF summarizes
ART prevention impacts (relative incidence reduction/proportion of infections averted),
stratified by other factors of risk heterogeneity, epidemic contexts, and intervention conditions.
ART prevention impacts were larger with longer time horizon, greater ART eligibility, and higher ART coverage.

DISCUSSION

Via scoping review, we found that representations of risk heterogeneity varied widely across
transmission modelling studies of ART intervention in SSA, with
stratification by sexual activity and key populations considered in approximately
2/3 and 2/5 of models, respectively.
We also found that the projected proportions of infections averted due to ART scale-up were
larger under assumptions of homogeneous risk or prioritized ART to key populations,
as compared to heterogeneous risk or without prioritized ART to key populations.
Three notable themes emerged from our review.

First, modelling studies have an opportunity to keep pace with growing epidemiological data on risk heterogeneity.
For example, 41\% of the modelling studies reviewed included at least one key population, such as FSW and or MSM.
Key populations continue to experience disproportionate risk of HIV, even in high-prevalence epidemics,
and models examining the unmet needs of key populations suggest that
these unmet needs play an important role in overall epidemic dynamics.
Furthermore, the we found that the number of modelled clients per female sex worker, and
the relative rate of partnership formation among female sex workers versus other women
did not always reflect the available data.
Similarly, among studies with different partnership types, only 20\% modelled
main/spousal partnerships---with more sex acts/lower condom use---between two higher risk individuals,
while 80\% modelled only casual/commercial partnerships among higher risk individuals.
However, data suggest female sex workers may form main/spousal partnerships
with regular clients and boyfriends/spouses from higher risk groups.
Thus, future models can continue to include emerging data on these and other factors of heterogeneity,
while nested model comparison studies can study how
multiple factors might act together to influence projections of ART impact.

Second, most models assumed equal ART cascade transition rates across subgroups,
including diagnosis, ART initiation, and retention.
Recent data suggest differential ART cascade by sex, age, and key populations.
These differences may stem from the unique needs of population subgroups
and is one reason why differentiated ART services are a core component of HIV programs.
Moreover, barriers to ART may intersect with transmission risk, particularly among key populations,
due to issues of stigma, discrimination, and criminalization.
Thus, further opportunities exist to: incorporate differentiated cascade data,
examine the intersections of intervention and risk heterogeneity, and
to consider the impact of HIV services as they are delivered on the ground.
Similar opportunities were noted regarding modelling of pre-exposure prophylaxis in SSA.
Finally, depending on the research question, the modelled treatment cascade may need expansion
to include more cascade steps and states related to treatment failure/discontinuation.

Third, based on ecological analysis of scenarios, we found that
modelling assumptions about risk and intervention heterogeneity
may influence the projected proportion of infections averted by ART.
We did not find similar evidence for relative incidence reduction due to ART,
but studies reporting both outcomes were largely distinct.
Among studies reporting both, the overall pattern was consistent.
These findings highlight the limitations of ecological analysis to estimate
the potential influence of modelling assumptions on projected ART prevention benefits,
and motivate additional model comparison studies to better quantify this influence,
such as.
Our ecological analysis also suggested that the anticipated ART prevention impacts from homogeneous models
may be achievable in the context of risk heterogeneity
if testing and treatment resources are prioritized to higher risk groups.

Limitations of our scoping review include our examination of only a few key populations.
In our conceptual framework for risk heterogeneity, we did not explicitly examine heterogeneity
by type of sex act (i.e. anal sex) which is associated with higher probability of HIV transmission,
nor structural risk factors like violence.
The large number of differences between scenarios in the scoping review context
also limited our ability to infer the influence of risk heterogeneity across scenarios.

In conclusion, representations of risk heterogeneity vary widely
among models used to project the prevention impacts of ART in SSA.
Such differences may partially explain the large variability in projected impacts.
Opportunities exist to incorporate new and existing data on
the intersections of risk and intervention heterogeneity.
Moving forward, systematic model comparison studies are needed to
estimate and understand the influence of various modelling assumptions on ART prevention impacts.
