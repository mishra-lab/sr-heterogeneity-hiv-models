We searched MEDLINE and EMBASE via Ovid
using search terms related to Sub-Saharan Africa (SSA), HIV, and transmission modelling
(Appendix~\ref{aa:search:terms}).
Duplicate studies were removed automatically and also manually.
Potentially relevant studies were identified by title and abstract screening.
Further selection of studies and subsequent data extraction used
the full text and any available supplementary material.
One reviewer (JK) conducted the search and data extraction.
% ==============================================================================
\subsection{Inclusion/Exclusion Criteria}
\label{ss:meth:criteria}
A first set of criteria identified any studies applying
dynamical models of sexual HIV transmission
to any research question and context within SSA.
A dynamical transmission model projects the number of new HIV infections
based on model inputs and previous model outputs,
thereby capturing nonlinear population-level transmission dynamics,
such as indirect benefits of prevention interventions.
Models were only included if they considered sexual transmission of HIV,
possibly alongside other routes of transmission,
such as mother-to-child and parenteral,
and possibly alongside transmission of other infectious diseases
such as other STI, tuberculosis, or malaria.
We excluded studies without primary modelling results,
such as commentaries and reviews, as well as conference publications.
\par
A second set of criteria was used to select studies using
compartmental (vs individual-based) HIV transmission models,
applied to assess the any outcome related to
the prevention benefits of ART scale-up.
These studies formed the basis of the analysis to answer
our first and second research questions.
A third set of criteria was used to select a subset of the above
to answer our third research question.
The subset were selected if they specifically examined
scale-up of ART coverage alone (vs combination intervention),
and reported incidence reduction or cumulative infections averted after X years,
as compared to a base-case scenario reflecting reality.
Complete inclusion/exclusion criteria are given in
Appendix~\ref{aa:search:criteria}.
% ==============================================================================
\subsection{Data Extraction}
\label{ss:meth:data}
Most variables were extracted per-study,
vs per-scenario within the study.
We also noted whether studies used the same model,
and whether identical analyses were conducted in multiple studies
(for example, if a second study only added economic analysis to
modelling results from an earlier study).
% ------------------------------------------------------------------------------
\subsubsection{Epidemic Context}
\label{sss:meth:context}
To answer our first research question, we extracted the following data.
Studies were categorized by the geographic location (country and SSA region)
and scale of the simulated population (city, sub-national, national, super-national),
including whether multiple geographic contexts were considered.
The epidemic phase was categorized based on
the overall HIV prevalence and incidence at the time of
intervention roll-out/scenario divergence,
as well as the sign ($+/-/=$) of the first and second order derivatives of
HIV prevalence and incidence, assessed visually from plots.
Finally, we noted whether any key populations were simulated,
in addition to the so-called general population.
% ------------------------------------------------------------------------------
\subsubsection{Factors of Risk Heterogeneity}
\label{sss:meth:factors}
For our second research question, we examined if and how
the factors of risk heterogeneity outlined in Table~\ref{tab:heterogeneity}
were simulated in each study.
\par
Special focus was given to the factors related to Network and Coverage Effects,
due to the large variation in how these factors were simulated.
% JK: @SM: two questions
%     1. If so, ^ is this sufficient justification?
%     2. Since most other factors in Table 1 are extracted like "included: yes/no",
%        I didn't want to waste space here to re-define them all. Is that ok?
We examined the number and defining characteristics of
\emph{activity groups}, including
sex, different rates of partnership formation, and different types of partnerships.
We noted whether each of the following \emph{key populations} was included in the model:
female sex workers (FSW);
male clients of FSW (Cli);
adolescent girls and young women (AGYW);
men who have sex with men (MSM);
and people who inject drugs (PWID).
Any \emph{turnover} of individuals between
activity groups and/or key populations was noted.
Similarly, we noted whether ART coverage was assumed to be
equal across modelled risk groups,
possibly ignoring historical gaps/future challenges in reaching higher risk groups.
\par
We noted which of the following characteristics were used to define
different sexual \emph{partnership types}:
the risk groups involved;
different volumes of sex
(total number of coital acts per partnership,
subsuming frequency and partnership duration);
and different levels of condom use.
We noted whether simulated partnerships represented
any of the following identifiable types:
main/spousal;
casual/extramarital;
commercial/sex work;
and transactional (exchange of gifts/favours for sex, outside formal sex work).
We noted whether models simulated any degree of assortative vs proportionate
\emph{mixing} between activity groups.
% If some partnership types were only formed by certain risk groups,
% mixing was automatically considered assortative.
% JK: ^ never actually comes up I think ...
The number of unique \emph{age groups} was noted, as well as
whether \emph{mixing} by age groups was
proportionate, strictly assortative, or assortative with age differences.
Finally, we noted whether age conferred any additional risk beyond mixing,
such as higher rates of partnership formation.
% JK: [TODO] anal sex & STI co-infection ?
% ------------------------------------------------------------------------------
\subsubsection{ART Prevention Impact}
\label{sss:meth:api}
For our third research question, we examined the subset of studies
reporting incidence reduction or infections averted
due to ART scale-up alone (see \ref{ss:meth:criteria}).
We then extracted the following data for each ART scale-up scenario within each study.
We noted: the years that ART scale-up started and stopped, corresponding to
the time each scenario diverged from the base-case scenario;
the time ART coverage or initiation rates stabilized following scale-up;
the final overall ART coverage achieved and/or
the final ART initiation rate (per person-year among PLHIV);
the criteria for ART initiation (e.g. CD4 count);
and the assumed relative reduction in transmission probability on ART .
Then, we extracted two population-level ART prevention outcomes
from in-text, tabular, and figure data:
relative reduction in incidence (\%) or proportion of infections averted (\%)
reported for different time horizons relative to the start of ART scale-up.
Figure data were extracted for any of the following time horizons, if available:
5, 10, 15, 20, 30, and 40 years,
with the help of a graphical measurement tool.%
\footnote{WebPlotDigitizer: \hreftt{https://apps.automeris.io/wpd/}}
\par
Finally, to quantify the potential influence of different factors of heterogeneity
on projected incidence reduction and infections averted (outcomes),
we fit a multivariate regression model for each outcome,
and computed the partial correlation coefficients associated with each factor.
% JK: I'm still not 100% this is the best approach since the outcome is bounded...
%     I've asked the stats community here: https://stats.stackexchange.com/questions/497305
%     and what do you think about reaching out to Dr Escobar for his thoughts?
% ----
% JK: I use the word "noted" about a bajillion times here
%     since it is short and unpretentious,
%     and I personally find the meaning clear in this context. @SM hoping you agree?
% JK: Also, I use the word "study" to essentially mean article,
%     but perhaps we should use "article" throughout instead?