\section{Methods}
\label{s:meth}
We conducted a scoping review according to the PRISMA extension for scoping reviews
(Appendix~\ref{a:prisma}).
First, we developed a conceptual framework to organize
assumptions and representations of risk heterogeneity
in compartmental HIV transmission models.
Then, we designed and implemented the search strategy,
and extracted data relevant to the framework to address our objectives.
% ==============================================================================
\subsection{Conceptual Framework for Risk Heterogeneity}
\label{ss:meth:framework}
We conceptualized ``factors of risk heterogeneity'', meaning
epidemiological stratifications and phenomena which may/not be included in transmission models.
Such factors could include if/how populations, rates, and probabilities
are stratified along various dimensions.
We defined the following 4 domains in which
these factors might influence the transmission impact of ART.
\begin{itemize}
  \item \textbf{Biological Effects:}
  differential transmission risk within HIV disease course
  that may coincide with differential ART coverage
  \cite{Pilcher2004}
  \item \textbf{Behaviour Change Effects:}
  differential transmission risk due to
  behavioural changes related to engagement in the ART cascade
  \cite{Ramachandran2016,Tiwari2020}
  \item \textbf{Network Effects:}
  differential transmission risk within sub-populations
  that increases the challenge of epidemic control through core group dynamics
  \cite{Boily1997,Watts2010,Dodd2010}
  \item \textbf{Coverage Effects:}
  differential transmission risk within sub-populations
  who experience barriers to ART care and achieving viral suppression,
  such as youth and key populations
  \cite{Mountain2014,Lancaster2016,Hakim2018,Green2020}
\end{itemize}
We then compiled a list of key factors of risk heterogeneity,
and their possible mechanisms of influence on ART prevention impact (Table~\ref{tab:factors}).
\begin{sidewaystable}
  \caption{%
    Factors of heterogeneity in HIV transmission
    and their possible mechanisms of influence on the prevention impact of ART interventions}
  \footnotesize\centering
\begin{tabular}{llp{.35\linewidth}p{.4\linewidth}}
  \toprule
  \textbf{Factor}
& \textbf{MP\tn{a}}
& \textbf{Definition}
& \textbf{Possible mechanism(s) of influence on ART prevention impact}
\\
\midrule
  Acute Infection
& $\beta_i$
& Increased infectiousness immediately following infection \cite{Hollingsworth2008,Boily2009}
& \textbf{Biological}: transmissions during acute infection are unlikely to be prevented by ART
\\
  Late-Stage Infection
& $\beta_i$
& Increased infectiousness during late-stage infection \cite{Hollingsworth2008,Boily2009}
& \textbf{Biological}: transmissions during late-stage are more likely to be prevented by ART
\\
  Drug Resistance
& $\beta_i$
& Transmitted factor that requires regimen switch to achieve viral suppression \cite{DeWaal2018}
& \textbf{Biological}: transmissions during longer delay to achieving viral suppression will not be prevented by ART
\\
\midrule
  HIV Morbidity
& $c$; $\eta$
& Reduced sexual activity during late-stage disease \cite{Myer2010,McGrath2013}
& \textbf{Behaviour Change}: reduced morbidity via ART could increase HIV prevalence among the sexually active population
\\
  HIV Counselling
& $c$; $\eta$; $\kappa$
& Reduced sexual activity and/or increased condom use after HIV diagnosis \cite{Tiwari2020}
& \textbf{Behaviour Change}: increased HIV testing with ART scale up can contribute to prevention even before viral suppression is achieved
\\
\midrule
  Activity Groups
& $c$; $\kappa$
& Any stratification by rate of partnership formation \cite{Anderson1991}
& \textbf{Network}: higher transmission risk among higher activity
\\
  Age Groups
& $c$; $\kappa$
& Any stratification by age
& \textbf{Network \& Cascade}: higher transmission risk and barriers to viral suppression among youth \cite{Birdthistle2019,Green2020}
\\
  Key Populations
& $c$; $\kappa$
& Any epidemiologically defined higher risk groups \cite{WHO2016KP}
& \textbf{Network \& Cascade} higher transmission risk and barriers to viral suppression among key populations \cite{Hakim2018}
\\
  Group Turnover
& $\phi$
& Individuals move between activity groups and/or key populations reflecting sexual lifecourse \cite{Watts2010}
& \textbf{Network \& Cascade}: counteract effect of stratification due to shorter periods in higher risk \cite{Knight2020};
  viral suppression may be achieved only after periods of higher risk
\\
  Assortative Mixing
& $m$
& Any degree of assortative mixing (like-with-like) by age, activity, and/or key populations
& \textbf{Network}: assortative sexual networks compound effect of stratification \cite{Anderson1991}
\\
  Partnership Types
& $\eta$; $\kappa$
& Different partnership types are simulated, with different numbers of sex acts and/or condom usage \cite{Scorgie2012}
& \textbf{Network}: longer duration and lower condom use among main versus casual/sex work partnerships
  counteracts effect of stratification
\\
  ART Cascade Gaps
& $\tau$; $\alpha$
& Slower ART cascade transitions among higher activity groups or key populations \cite{Hakim2018,Green2020}
& \textbf{Cascade}: ART prevention benefits may be allocated differentially among risk groups
\\
\bottomrule
\end{tabular}
\floatfoot{\tnt[a]{MP: Model Parameters ---
  $\beta_i, \beta_s$: transmission probability per act (infectiousness, susceptibility);
$\eta$:      number of sex acts of each type per partnership;
$\kappa$:    proportion of sex acts unprotected by a condom;
$c$:         partnership formation rate;
$m$:         mixing matrix (probability of partnership formation);
$\mu$:       mortality rate;
$\nu$:       entry rate;
$\phi$:      internal turnover between activity groups;
$\tau$:      testing rate;
$\alpha$:    ART initiation rate (and retention-related factors).}
}

  \label{tab:factors}
\end{sidewaystable}
% ==============================================================================
\subsection{Search}
\label{ss:meth:search}
We searched MEDLINE and EMBASE via Ovid
using search terms related to Sub-Saharan Africa (SSA), HIV, and transmission modelling
(Table~\ref{tab:search}).
Search results were imported into Covidence \cite{Covidence} for screening.
Duplicate studies were removed automatically by Ovid and by Covidence;
four additional duplicates with subtly different titles were later identified and removed manually.
Potentially relevant studies were identified by title and abstract screening, followed 
by full-text screening using the inclusion/exclusion criteria below. 
One reviewer (JK) conducted the search, screening, and data extraction.
% ------------------------------------------------------------------------------
\subsubsection{Inclusion/Exclusion Criteria}
\label{sss:meth:criteria}
We included peer-reviewed, primarily modelling studies that used non-linear models of sexual HIV transmission
to project the prevention impacts of ART in any setting within SSA.
See Table~\ref{tab:criteria} for complete inclusion/exclusion criteria.
We only included studies published in English anytime before Jan~1, 2020.
We excluded conference publications, and those without primary modelling results and description of the methods,
such as commentaries and reviews.
If a model's details were provided in another peer-reviewed publication,
we extracted data from both publications as required.
\par
Inclusion criteria were as follows:
1) Used a non-linear compartmental model of sexual HIV transmission at the population level.
We defined \emph{non-linear models} as those where
future projected infections are a function of previously projected infections,
and \emph{compartmental models} as those where
system variables represent the numbers of individuals in each state,
rather than unique individuals \cite{Garnett2011}.
Thus, statistical models, models without dynamic transmission, and individual-based models were excluded.
2) The model was parameterized/calibrated to reflect at least one setting within SSA
(see Appendix~\ref{aa:defs:context} for countries).
3) The study simulated at least one scenario with increasing ART coverage,
possibly alongside other interventions.
The included studies formed Dataset~A,
used to complete objectives \ref{rq:1}~and~\ref{rq:2}.
\par
A subset of Dataset~A formed Dataset~B,
used to complete objective~\ref{rq:3}.
Studies in Dataset~B specifically examined
scale-up of ART coverage alone (versus combination intervention)
for the whole population (versus ART prioritized to subgroups),
and reported HIV incidence reduction or cumulative HIV infections averted
relative to a base-case scenario reflecting status quo.
% ==============================================================================
\subsection{Data Extraction}
\label{ss:meth:data}
Data extraction used the full text and all available supplementary material.
Data were extracted per-study for objectives \ref{rq:1}~and~\ref{rq:2}, and
per-scenario for objective~\ref{rq:3}.
Detailed variables definitions are given in Appendix~\ref{a:defs}.
% ------------------------------------------------------------------------------
\subsubsection{Epidemic Context}
\label{sss:meth:context}
For objective~\ref{rq:1}, we extracted data on
geography, epidemic phase, and key populations explicitly considered in the model.
We categorized studies by country, SSA region, and
scale of the simulated population (city, sub-national, national, regional).
We classified epidemic size at time of ART intervention using
overall HIV prevalence (low: $<$1\%, medium: 1-10\%, high: $>$10\%),
and epidemic phase using overall HIV incidence trend
(increasing, increasing-but-stabilizing, stable/equilibrium, decreasing-but-stabilizing, and decreasing).
\par
We extracted whether any of the following key populations were modelled:
female sex workers (FSW);
male clients of FSW (Clients);
men who have sex with men (MSM); and
people who inject drugs (PWID).
FSW were defined as any female activity group meeting 3 criteria:
{$<$5\%} of the female population;
{$<$1/3} the client population size; and
having {$>$50$\times$} the partners per year of
the lowest sexually active female activity group \cite{Vandepitte2006,Scorgie2012}.
Clients were defined as any male activity group
described as clients of FSW, and being {$>$3$\times$} the FSW population size.
We also extracted whether any groups in the model were described as MSM or PWID.
% ------------------------------------------------------------------------------
\subsubsection{Factors of Risk Heterogeneity}
\label{sss:meth:factors}
For objective~\ref{rq:2}, we examined if/how
the factors of risk heterogeneity outlined in Table~\ref{tab:factors}
were simulated in each study.
We examined the number of \emph{risk groups} defined by sex and/or sexual activity, and
any \emph{turnover} of individuals between activity groups and/or key populations.
\par
We classified how \emph{partnership types} were defined:
generic (all partnerships equal);
based only on the activity groups involved;
or overlapping, such that different partnership types could be formed between the same two activity groups.
We extracted whether partnerships considered different
numbers of sex acts and condom use,
and whether models simulated any degree of assortative \emph{mixing} by activity groups
(preference for like-with-like) versus proportionate (random) mixing.
The number of \emph{age groups} was extracted, and whether \emph{mixing} by age groups was
proportionate, strictly assortative, or assortative with age differences.
We extracted whether age conferred any transmission risk beyond mixing,
such as different partnership formation rates.
\par
Finally, we extracted whether rates of HIV diagnosis, ART initiation, and/or ART discontinuation
differed across risk strata (sex, activity, key populations, and/or age),
and if so, how they differed.
% ------------------------------------------------------------------------------
\subsubsection{ART Prevention Impact}
\label{sss:meth:api}
For objective~\ref{rq:3}, we extracted
the following data for each intervention scenario within Dataset~B:
the years that ART scale-up started (\tx{0}) and stopped (\tx{f});
the final overall ART coverage achieved and/or
the final ART initiation rate (per person-year among PLHIV not yet in care);
the criteria for ART initiation (e.g.\ CD4 count); and
the relative reduction in transmission probability on ART.
Then, we extracted the
relative reduction in incidence and/or proportion of infections averted
relative to the base-case scenario for available time horizons relative to \tx{0}.
\par
For each factor of heterogeneity,
we compared projected ART impacts (incidence reduction/infections averted)
across different factor levels (whether or not, and how the factor was modelled).
We estimated the effect of each factor level on ART impacts
using linear multivariate regression, with generalized estimating equations \cite{Hojsgaard2006}
to control for clustering due to multiple estimates per study.
Time since \tx{0} was included as a covariate, and two variables were removed due to missingness.
No variable selection was used to avoid biasing effect estimates \cite{Harrell2001}.
We also plotted impacts versus time since \tx{0}, stratified by factor levels.