We conducted a scoping review according to the PRISMA extension for scoping reviews
(see Appendix~\ref{a:prisma} for checklist).
% JK: might want a different word than "appraise" as appraise implies
%     judging value, which, not sure if that's what we want to do here ...  %SM: yeah, I see your point here. how about these pot'l edits? (e.g. "synthesize", or "review")?
First, we developed a framework to synthesize  %SM: or review or categorize?
the assumptions and representations of risk heterogeneity
in compartmental HIV transmission models.
Then, we designed and implemented the search strategy,
and extracted the data relevant to the framework
to address the objectives.
% ==============================================================================
\subsection{Appraisal Framework}  %SM: what do you think about using term "conceptual framework for risk heterogeneity" instead of "appraisal framework"? 
\label{ss:meth:framework}
For the review, we conceptualized ``factors of risk heterogeneity'', meaning
epidemiological stratifications and phenomena which may/not be included in transmission models.
Such factors could include if/how populations, rates, and probabilities
are stratified along health and social dimensions.   %what is meant by 'social' dimensions?
We defined the following 4 domains to help conceptualize
% JK: need better wording for "domains through which"
mechanisms related to risk heterogeneity might influence the transmission impact of ART.
\begin{itemize}
  \item \textbf{Biological Effects:}
  differential transmission risk within HIV disease course
  that coincide with differential ART coverage
  \cite{Pilcher2004}
  \item \textbf{Behaviour Change Effects:}
  differential transmission risk due to
  behavioural changes related to engagement in the ART cascade
  \cite{Ramachandran2016,Tiwari2020} % JK: TODO
  \item \textbf{Network Effects:}
  differential transmission risk within sub-populations
  that increase the challenge of epidemic control through core group dynamics
  \cite{Anderson1986,Boily1997,Watts2010,Dodd2010}
  \item \textbf{Coverage Effects:}
  differential transmission risk within sub-populations
  who also experience barriers to engaging in ART care and achieving viral suppression,
  such as youth and key populations
  \cite{Mountain2014,Lancaster2016,Hakim2018,Green2020}
\end{itemize}
We compiled a list of key factors of risk heterogeneity,   %not sure I understand this paragraph?
and their associated mechanisms of influence on ART prevention impact (Table~\ref{tab:factors}).
We did not attempt to define the magnitude or direction of each factor's influence,
since these can depend on the context, time horizon,
and which if any parameters were fitted during model calibration.\cite{Eaton2014}
\afterpage{%
  \newgeometry{margin=2cm}
  \begin{landscape}
    \captionof{table}{%
      Factors of heterogeneity in HIV transmission
      and their possible mechanisms of influence on the prevention impact of ART interventions}
    \footnotesize\centering
\begin{tabular}{llp{.35\linewidth}p{.4\linewidth}}
  \toprule
  \textbf{Factor}
& \textbf{MP\tn{a}}
& \textbf{Definition}
& \textbf{Possible mechanism(s) of influence on ART prevention impact}
\\
\midrule
  Acute Infection
& $\beta_i$
& Increased infectiousness immediately following infection \cite{Hollingsworth2008,Boily2009}
& \textbf{Biological}: transmissions during acute infection are unlikely to be prevented by ART
\\
  Late-Stage Infection
& $\beta_i$
& Increased infectiousness during late-stage infection \cite{Hollingsworth2008,Boily2009}
& \textbf{Biological}: transmissions during late-stage are more likely to be prevented by ART
\\
  Drug Resistance
& $\beta_i$
& Transmitted factor that requires regimen switch to achieve viral suppression \cite{DeWaal2018}
& \textbf{Biological}: transmissions during longer delay to achieving viral suppression will not be prevented by ART
\\
\midrule
  HIV Morbidity
& $c$; $\eta$
& Reduced sexual activity during late-stage disease \cite{Myer2010,McGrath2013}
& \textbf{Behaviour Change}: reduced morbidity via ART could increase HIV prevalence among the sexually active population
\\
  HIV Counselling
& $c$; $\eta$; $\kappa$
& Reduced sexual activity and/or increased condom use after HIV diagnosis \cite{Tiwari2020}
& \textbf{Behaviour Change}: increased HIV testing with ART scale up can contribute to prevention even before viral suppression is achieved
\\
\midrule
  Activity Groups
& $c$; $\kappa$
& Any stratification by rate of partnership formation \cite{Anderson1991}
& \textbf{Network}: higher transmission risk among higher activity
\\
  Age Groups
& $c$; $\kappa$
& Any stratification by age
& \textbf{Network \& Cascade}: higher transmission risk and barriers to viral suppression among youth \cite{Birdthistle2019,Green2020}
\\
  Key Populations
& $c$; $\kappa$
& Any epidemiologically defined higher risk groups \cite{WHO2016KP}
& \textbf{Network \& Cascade} higher transmission risk and barriers to viral suppression among key populations \cite{Hakim2018}
\\
  Group Turnover
& $\phi$
& Individuals move between activity groups and/or key populations reflecting sexual lifecourse \cite{Watts2010}
& \textbf{Network \& Cascade}: counteract effect of stratification due to shorter periods in higher risk \cite{Knight2020};
  viral suppression may be achieved only after periods of higher risk
\\
  Assortative Mixing
& $m$
& Any degree of assortative mixing (like-with-like) by age, activity, and/or key populations
& \textbf{Network}: assortative sexual networks compound effect of stratification \cite{Anderson1991}
\\
  Partnership Types
& $\eta$; $\kappa$
& Different partnership types are simulated, with different numbers of sex acts and/or condom usage \cite{Scorgie2012}
& \textbf{Network}: longer duration and lower condom use among main versus casual/sex work partnerships
  counteracts effect of stratification
\\
  ART Cascade Gaps
& $\tau$; $\alpha$
& Slower ART cascade transitions among higher activity groups or key populations \cite{Hakim2018,Green2020}
& \textbf{Cascade}: ART prevention benefits may be allocated differentially among risk groups
\\
\bottomrule
\end{tabular}
\floatfoot{\tnt[a]{MP: Model Parameters ---
  $\beta_i, \beta_s$: transmission probability per act (infectiousness, susceptibility);
$\eta$:      number of sex acts of each type per partnership;
$\kappa$:    proportion of sex acts unprotected by a condom;
$c$:         partnership formation rate;
$m$:         mixing matrix (probability of partnership formation);
$\mu$:       mortality rate;
$\nu$:       entry rate;
$\phi$:      internal turnover between activity groups;
$\tau$:      testing rate;
$\alpha$:    ART initiation rate (and retention-related factors).}
}

    \label{tab:factors}
  \end{landscape}
  \restoregeometry
\clearpage}
% ==============================================================================
\subsection{Search}
\label{ss:meth:search}
We searched MEDLINE and EMBASE via Ovid
using search terms related to Sub-Saharan Africa (SSA), HIV, and transmission modelling. %SM: can't remember but I think you got some advice re: this from the information specailist at the litbreary? make sure to acknowldge in acknowledgemetn section and note that recieved guidance on search terms by infomration specialist in the appendix...
(Appendix~\ref{aa:search:terms}).
%did you use Covidence? if yes, include that in methods... (suggest checking Christine's paper re: some methods which might help and also read over other papers/guidelines on level of detail to include to ensure reproduciblity, etc?)
Duplicate studies were removed automatically %SM: how? (within Covidence?)
and also manually. %SM: how? did you check for duplicates manually?
Potentially relevant studies were identified by title and abstract screening, followed 
by full-text screening using the inclusion/exclusion criteria listed below. 
We then conducted data extraction from the full text and available supplementary material.
One reviewer (JK) conducted the search, screening, and data extraction.  
% ------------------------------------------------------------------------------
\subsubsection{Inclusion/Exclusion Criteria}
\label{sss:meth:criteria}
We included peer-reviewed, primarily modelling studies that used dynamical models of sexual HIV transmission
to project the prevention impacts of ART in any one or more settings within SSA.
Complete inclusion/exclusion criteria are given in 
Appendix~\ref{aa:search:criteria}
We only included studies communicated in English and published between XXX and Dec 31, 2019.
We excluded publications without primary modelling results and their description of the methods,
such as commentaries and reviews. We excluded conference publications.
If a model's details were provided in a separate peer-reviewed publication, we extracted data from both publications. %SM: correct?
\par
The following criteria were used for inclusion of studies:
1) used a dynamical compartmental model of
sexual HIV transmission at the population level.
We defined a \emph{dynamical model} as one where
the number of infections projected at time $t$ is a function of
the number of infections previously projected by the model before time $t$. %SM: cite
We defined a \emph{compartmental model} as one where
the system variables represent the numbers of individuals in each state,
rather than unique individuals. %SM: cite
Thus, statistical models, non-dynamical models, and individual-based models without dynamic transmission were excluded.
2) the model was calibrated or parameterized to reflect at least one setting within SSA.
(see Table~\ref{tab:search-ssa} for full country list).
3) the study simulated at least one scenario with increasing ART coverage,
possibly alongside scale-up of other interventions.
The included articles formed Dataset~A,
used to answer research questions \ref{rq:1}~and~\ref{rq:2}.
\par
A subset of Dataset~A formed Dataset~B,
which used to answer research question~\ref{rq:3}.
Articles in Dataset~B specifically examined
scale-up of ART coverage alone (vs combination intervention)
for the whole population (vs ART prioritized to subgroups),  %SM: i believe recommendation is to use the term "priotized" vs. targeted re: terminology now?
and reported HIV incidence reduction or cumulative HIV infections averted over time 
as compared to a base-case scenario reflecting the status quo.
% ==============================================================================
\subsection{Data Extraction}
\label{ss:meth:data}
% JK: I use the word "noted" about a bajillion times here
%     since it is short and unpretentious,
%     and I personally find the meaning clear in this context. @SM hoping you agree?  %SM: extracted is a more precise verb than noted. (in reading other reviews, have not seen the verb/term' noted'  used?...but I think what you mean here is that you categorized according to.... rather than critcized, etc.?
For research questions \ref{rq:1}~and~\ref{rq:2}, data were extracted per-article.
For research question~\ref{rq:3}, data were extracted per-scenario within the article.
%We also noted whether multiple articles used the same model. % currently unused.
Additional variables definitions are given in Appendix~\ref{a:defs}.
% ------------------------------------------------------------------------------
\subsubsection{Epidemic Context} %SM: this is good, but felt like mixing up extraction from 'categorizing (using the data extracted)" in the description....tried to suggest some edits to help with organizing that so could follow the steps
\label{sss:meth:context}
To address objective 1, we extracted data on geography, epidemic phase, and subgroups explicitly considered in the model.
Specifically, we categorized studies by the geographic location (country and SSA region), 
scale of the simulated population (city, sub-national, national, regional), and 
whether the study included multiple geographic settings.
We classified epidemic size using  
the overall HIV prevalence at the time of ART intervention (low: $<1\%$, medium: $1-10\%$, high: $>10\%$),  %SM: i think this is conflating epidemic phase and has been argued against as a definition for epidemic phase by most people (Blanchard, etc). suggest you call this epidemic size instead. its more precise...
and classified epidemic phase at the time of ART intervention 
using the trend in incidence at the time of ART intervention %SM: i found 'at time of scenarios diverged' confusing... maybe simplify it for the reader a bit more...see edits?
(increasing, increasing but stabilizing, stable/equilibrium,
decreasing but stabilizing, and decreasing).  %SM: this is more in line with epidemic phase. others have used for example, ratio of HIV prevalence/incidence (one of Mike's papers - though I dont' think he has submited that yet, and a paper by the UNAIDS group recently). I think what you did is good here.
Subgroups of interest included the following key populations:
female sex workers (FSW);
male clients of FSW (Clients);
men who have sex with men (MSM); and
people who inject drugs (PWID).
%adolescent girls and young women (AGYW); and
%mobile populations;
See key population definitions in Appendix~\ref{aaa:defs:kp}
% ------------------------------------------------------------------------------
\subsubsection{Factors of Risk Heterogeneity}
\label{sss:meth:factors}
For our second research question, we examined if and how
the factors of risk heterogeneity outlined in Table~\ref{tab:factors}
were simulated in each study.
\par
Special focus was given to the factors related to Network and Coverage Effects,
due to the large variability in how these factors were simulated.
% JK: @SM: two questions
%     1. If so, ^ is this sufficient justification?
%     2. Since most other factors in Table 1 are extracted like "included: yes/no",
%        I didn't want to waste space here to re-define them all.
%        Plus, they are repeated again in the results as well. Is that ok? SM: yes agree
We examined the number and defining characteristics of
\emph{activity groups}, including
sex, different rates of partnership formation, and different types of partnerships.
We noted whether each of the \emph{key populations} noted above
was included in the model.
Any \emph{turnover} of individuals between
activity groups and/or key populations was noted.
Similarly, we noted whether ART coverage was assumed to be
equal across modelled risk groups,
possibly ignoring historical gaps/future challenges in reaching higher risk groups.
\par
We noted whether multiple \emph{partnership types} were simulated,
and how such partnerships were defined:
generic (all partnerships equal);
based on the activity groups involved;
or reflecting phenomenological types
(main/spousal; casual; commercial/sex work; and transactional).
We noted whether partnerships considered different
volumes of sex (total number of coital acts per partnership)
and levels of condom use.
We noted whether models simulated any degree of assortative vs proportionate
\emph{mixing} between activity groups.
The number of unique \emph{age groups} was noted, as well as
whether \emph{mixing} by age groups was
proportionate, strictly assortative, or assortative with age differences.
Finally, we noted whether age conferred any additional risk beyond mixing,
such as higher rates of partnership formation.
\par
Finally, we noted whether differences in rates of progression along the \emph{ART cascade}
were considered between age groups, sexes, activity groups, and/or key populations.
Specifically, we noted differences in rates of
diagnosis, ART initiation, and treatment discontinuation (due to either dropout or resistance).
% ------------------------------------------------------------------------------
\subsubsection{ART Prevention Impact}
\label{sss:meth:api}
For objective 3, we examined the subset of studies (Dataset~B)
reporting incidence reduction or infections averted due to
population-wide ART scale-up.
We extracted the following data for each scenario of ART scale-up within Dataset~B:
the years that ART scale-up started and stopped, corresponding to
the time each scenario diverged from the base-case scenario ($t_0$) and
the time ART coverage or initiation rates stabilized following scale-up ($t_f$);
the final overall ART coverage achieved and/or
the final ART initiation rate (per person-year among PLHIV not yet in care);
the criteria for ART initiation (e.g. CD4 count);
and the relative reduction in transmission probability on ART.
Then, we extracted relative reduction in incidence or proportion of infections averted
reported for different time horizons relative to $t_0$.
Figure data were extracted for any of the following time horizons, if available:
5, 10, 15, 20, 30, and 40 years,
with the help of a graphical measurement tool.%
\footnote{WebPlotDigitizer: \hreftt{https://apps.automeris.io/wpd/}}
\par
Finally, for each factor of heterogeneity,
we compared the projected ART prevention impacts across
the different factor levels (whether or not, and how the factor was modelled).
We plotted impact magnitude vs time since $t_0$, stratified by factor levels,
and explored whether the distribution in magnitude of impact
was the same under all factor levels (non-parametric Kruskal-Wallis test).