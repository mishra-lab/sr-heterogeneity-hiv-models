\textbf{Background.}
Transmission models provide complementary evidence to clinical trials about
the potential population-level incidence reduction attributable to ART (ART prevention impact).
Different modelling assumptions about risk heterogeneity may influence projected ART prevention impacts.
We sought to review representations of risk heterogeneity in compartmental HIV transmission models
applied to project ART prevention impacts in Sub-Saharan Africa.
\textbf{Methods.}
We systematically reviewed studies published before January 2020 that used
non-linear compartmental models of sexual HIV transmission
to simulate ART prevention impacts in Sub-Saharan Africa.
We summarized data on model structure/assumptions (factors) related to risk and intervention heterogeneity,
and explored multivariate ecological associations of ART prevention impact with modelled factors.
\textbf{Results.}
Of 1384 search hits, 94 studies were included,
which primarily modelled medium/high prevalence epidemics in East/Southern Africa.
64 studies considered sexual activity stratification and 39 modelled at least one key population.
21 studies modelled faster/slower ART cascade transitions (HIV diagnosis, ART initiation, or cessation) by risk group,
including 8 with faster and 4 with slower cascade transitions among key populations versus the wider population.
In ecological analysis, turnover of higher risk groups and ART cascade differences by sex/gender
both reduced projected ART prevention benefits, whereas
ART cascade prioritized to key populations improved ART prevention benefits.
\textbf{Conclusion.}
Among compartmental transmission models applied to project ART prevention impacts in Sub-Saharan Africa,
representations of risk heterogeneity and projected impacts varied considerably.
Future work should explore how assumptions about
turnover and cascade differences amongst risk groups
may influence projected ART prevention impacts.