\textbf{Background.}
Transmission models provide complementary evidence to clinical trials about
the potential population-level incidence reduction attributable to ART (ART prevention impact).
Different modelling assumptions about risk heterogeneity may influence projected ART prevention impacts.
We sought to review representations of risk heterogeneity in compartmental HIV transmission models
applied to project ART prevention impacts in Sub-Saharan Africa.
\textbf{Methods.}
We systematically reviewed studies published before January 2020 that used
non-linear compartmental models of sexual HIV transmission
to simulate ART prevention impacts in Sub-Saharan Africa.
We summarized data on model structure/assumptions (factors) related to risk and intervention heterogeneity,
and explored multivariate ecological associations of ART prevention impacts with modelled factors.
\textbf{Results.}
Of 1384 search hits, 94 studies were included.
64 studies considered sexual activity stratification and 39 modelled at least one key population.
21 studies modelled faster/slower ART cascade transitions (HIV diagnosis, ART initiation, or cessation) by risk group,
including 8 with faster and 4 with slower cascade transitions among key populations versus the wider population.
In ecological analysis of 125 scenarios from 40 studies (subset without combination intervention),
scenarios with risk heterogeneity that included turnover of higher risk groups
were associated with smaller ART prevention benefits.
Modelled differences in ART cascade across risk groups also influenced the projected ART benefits, including:
ART prioritized to key populations was associated with larger ART prevention benefits.
Of note, zero of these 125 scenarios considered lower ART coverage among key populations.
\textbf{Conclusion.}
Among compartmental transmission models applied to project ART prevention impacts in Sub-Saharan Africa,
representations of risk heterogeneity and projected impacts varied considerably.
Inclusion/exclusion of risk heterogeneity with turnover, and intervention heterogeneity across risk groups
could influence the projected impacts of ART scale-up.
These findings highlight a need to capture
risk heterogeneity with turnover and cascade heterogenetiy when projecting ART prevention impacts.