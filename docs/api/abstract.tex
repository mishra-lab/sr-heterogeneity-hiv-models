% JK: FYI, word limit: 249 / 250 %SM: great abstract. succint and clear
\par\textbf{Objective.}
Transmission models provide complementary evidence to clinical trials about
the potential population-level incidence reduction attributable to ART
(ART prevention impact). % JK: is this "definition" in brackets needed?
Different modelling assumptions about risk heterogeneity may influence
projected ART prevention impacts.
We sought to review representations of risk heterogeneity in compartmental HIV transmission models
applied to project ART prevention impacts in Sub-Saharan Africa.
\par\textbf{Design.}
% JK: FYI, this section should: "State the principal reasoning for the procedures adopted"
Scoping review to identify common modelling assumptions and applications.
\par\textbf{Methods.}
We systematically reviewed studies published before Xmonth, 2020 that used %SM: give month in 2020
dynamical compartmental models of sexual HIV transmission
to simulate the prevention impacts of ART in Sub-Saharan Africa.
We a prori conceptualized a framework for risk and intervention heterogeneity and then 
summarized data on model structure and assumptions (factors) related to risk heterogeneity.
We also explored crude associations of reported ART prevention impact with modelled factors of risk heterogeneity.
\par\textbf{Results.}
Of 1384 search hits, 94 studies were included which primarily modelled medium/high prevalence epidemics in East and Southern Africa.
64 studies considered stratification by sexual activity and 39 modelled at least one key population. %SM: give % withe the N if word count allows
21 studies modelled differential ART cascade by risk group,
including 8 with higher and 4 with lower cascade among key populations versus the wider population.
Models without activity stratification predicted the largest proportion of infections averted by ART,
followed by models with key populations that had higher ART cascade versus the general population.
\par\textbf{Conclusions.}
Among compartmental transmission models applied to project the prevention impacts of ART,
representations of risk heterogeneity and the projected impacts both varied considerably; with a tendency for models 
with relatively less heterogeneity to predict larger transmission impact of ART scale-up.
The potential influence of modelling assumptions about
differential ART cascade among risk groups should be further explored.
