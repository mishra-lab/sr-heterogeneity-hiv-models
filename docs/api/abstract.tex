\textbf{Background.}
Transmission models provide complementary evidence to clinical trials about
the potential population-level incidence reduction attributable to ART (ART prevention impact).
Different modelling assumptions about risk heterogeneity may influence projected ART prevention impacts.
We sought to review representations of risk heterogeneity in compartmental HIV transmission models
applied to project ART prevention impacts in Sub-Saharan Africa.
\textbf{Methods.}
We systematically reviewed studies published before January 2020 that used
non-linear compartmental models of sexual HIV transmission
to simulate ART prevention impacts in Sub-Saharan Africa.
We summarized data on model structure/assumptions (factors) related to risk and intervention heterogeneity,
and explored crude associations of ART prevention impact with modelled factors.
\textbf{Results.}
Of 1384 search hits, 94 studies were included,
which primarily modelled medium/high prevalence epidemics in East/Southern Africa.
64 studies considered sexual activity stratification and 39 modelled at least one key population.
21 studies modelled faster/slower ART cascade transitions (HIV diagnosis, ART initiation, or cessation) by risk group,
including 8 with faster and 4 with slower cascade transitions among key populations versus the wider population.
Models without activity stratification predicted the largest ART prevention impacts,
followed by models with key populations that had faster cascade transitions versus the wider population.
\textbf{Conclusion.}
Among compartmental transmission models applied to project ART prevention impacts,
representations of risk heterogeneity and projected impacts varied considerably,
where models with less heterogeneity tended to predict larger impacts.
The potential influence of modelling assumptions about
risk and intervention heterogeneity should be further explored.
