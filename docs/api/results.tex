The search yielded 1384 publications,
of which 94 studies were included (Figure~\ref{fig:prisma}).
These studies (Dataset~A) applied non-linear compartmental modelling to simulate ART scale-up in SSA,
of which 40 reported infections averted/incidence reduction
due to population-wide ART scale-up without combination intervention,
relative to a base-case reflecting status quo (Dataset~B).
Appendix~\ref{aa:search:dataset} lists the included papers, and
Appendix~\ref{a:results} provides additional results.
\begin{figure}
  \centering
  \includegraphics[width=0.8\textwidth]{prisma}
  \caption{PRISMA flowchart of study identification}
  \label{fig:prisma}
\end{figure}
% ==============================================================================
\subsection{Epidemic Context}
\label{ss:res:context}
Table~\ref{tab:summary} summarizes key features of contexts within SSA
where the prevention impacts of ART have been modelled.
Most (\x{geo/n.any.nat}) of the 94 studies modelled HIV transmission at the national level;
studies also explored
regional (\x{geo/n.any.sub.ssa}),
sub-national (\x{geo/n.any.sub.nat}), and
city-level (\x{geo/n.any.city}) epidemic scales.
South Africa was the most common country simulated (\x{co/n.South-Africa} studies), and
Figure~\ref{fig:map} illustrates the number of studies by country.
East Africa was the most represented SSA region, being simulated in \x{co/n.re.east} studies,
followed by Southern (\x{co/n.re.south}), West (\x{co/n.re.west}), and Central Africa (\x{co/n.re.central}).
\begin{table}
  \centering
  \caption{Summary of epidemic contexts within Sub-Saharan Africa where
    the prevention impacts of ART have been modelled}
  \label{tab:summary}
  \begin{tabular}{llr}
	\toprule
	\multicolumn{2}{l}{Study Characteristic} &               Studies \\
	\midrule
	Geographic scale & Regional              & \x{geo/n.any.sub.ssa} \\
	                 & National              &     \x{geo/n.any.nat} \\
	                 & Sub-national          & \x{geo/n.any.sub.nat} \\
	                 & City                  &    \x{geo/n.any.city} \\
	\midrule
	Modelled         & South Africa          & \x{co/n.South-Africa} \\
	countries\tn{a}  & Kenya                 &        \x{co/n.Kenya} \\
	                 & Zambia                &       \x{co/n.Zambia} \\
	                 & Other                 &        \x{co/n.Other} \\
	\midrule
	HIV prevalence   & Low ($<$1\%)          &     \x{t0/n.prev.Low} \\
	                 & Mid (1-10\%)          &     \x{t0/n.prev.Mid} \\
	                 & High ($>$10\%)        &    \x{t0/n.prev.High} \\
	                 & Unclear/Varies        &      \x{t0/n.prev.NA} \\
	\midrule
	Incidence trend  & Decreasing            &   \x{t0/n.phase.decr} \\
	at scenario      & Dec-to-stable         &    \x{t0/n.phase.dts} \\
	divergence       & Stable                &   \x{t0/n.phase.stab} \\
	                 & Inc-to-stable         &    \x{t0/n.phase.its} \\
	                 & Increasing            &   \x{t0/n.phase.incr} \\
	                 & Unclear/Varies        &     \x{t0/n.phase.NA} \\
	\midrule
	Key populations  & FSW\tn{b}             &    \x{kp/n.FSW.named} \\
	included         & Clients\tn{c}         &    \x{kp/n.Cli.named} \\
	                 & MSM                   &          \x{kp/n.MSM} \\
	                 & PWID                  &         \x{kp/n.PWID} \\
	\bottomrule
\end{tabular}
\floatfoot{
  Total studies: 94.
  FSW: female sex workers;
  Clients: clients of sex workers;
  MSM: men who have sex with men;
  PWID: people who inject drugs.
  \tnt[a]{Does not sum to 94 as some studies modelled multiple countries}.
  \tnt[b]{Groups described as FSW,
    not considering the epidemiological definitions given in Appendix~\ref{aaa:defs:kp}.}
  \tnt[c]{Likewise for clients, and excluding studies where clients were modelled
    as a proportion of another risk group.}
}

\end{table}
\begin{figure}
  \centering
  \includegraphics[width=0.8\textwidth]{map}
  \caption{Map showing number of studies (of 94 total)
    applying HIV transmission modelling in each country vs
    the number of people living with HIV (PLHIV, millions)}
  \label{fig:map}
\end{figure}
\par
ART prevention impacts were most often modelled in
high-prevalence ({$>$10\%}) epidemics (\x{t0/n.prev.High} studies) and
medium-prevalence ({1-10\%}) epidemics, \x{t0/n.prev.Mid}).
No studies reported overall HIV prevalence of {$<$1\%} at time of intervention,
although for \x{t0/n.prev.NA} studies, HIV prevalence was either
not reported or varied across simulated contexts/scenarios.
The \xdmdef year of intervention was \xdm{t0/t0}; at which time
HIV prevalence (\%) was \xdm{t0/prev}; and
incidence (per 1000 person-years) was \xdm{t0/inc}.
Most reported incidence trends were decreasing or stable (45 of 48 reporting). % MAN \x{t0/n.phase.*}
% ------------------------------------------------------------------------------
\subsubsection{Key Populations}
\label{sss:res:kp}
Groups representing FSW were described in \x{kp/n.FSW.named} studies.
Among these (of studies where it was possible to evaluate):
\x{kp/n.FSW.n.crit.p} (of \x{kp/n.FSW.n.crit.p.v}) were {$<$5\%} of the female population;
\x{kp/n.FSW.n.crit.pr} (of \x{kp/n.FSW.n.crit.pr.v}) were {$<$1/3} the size of the client population; and
\x{kp/n.FSW.n.crit.cr} (of \x{kp/n.FSW.n.crit.cr.v}) had {$>$50$\times$} partners per year versus
the lowest sexually active female activity group.
Clients of FSW were modelled as a unique group in \x{kp/n.Cli.named} studies,
among which \x{kp/n.Cli.n.crit} (of \x{kp/n.Cli.n.crit.v} reporting)
were {$>$3$\times$} the size of the FSW population.
In another \x{kp/n.Cli.named.p} studies, clients were defined as a proportion of another group,
among which \x{kp/n.Cli.n.p.crit} (of \x{kp/n.Cli.n.p.crit.v})
were {$>$3$\times$} the FSW population size.
Activity groups representing
men who have sex with men (MSM) were noted in \x{kp/n.MSM} studies; and
people who inject drugs (PWID) in \x{kp/n.PWID}.
% ==============================================================================
\subsection{Heterogeneity Factors}
\label{ss:res:factors}
% ------------------------------------------------------------------------------
\subsubsection{Biological Effects}
\label{sss:res:bio}
The \xdmdef number of states used to represent HIV disease
(ignoring treatment-related stratifications) was \xdm{hiv/hiv.n},
and \x{hiv/n.hiv.cts} studies represented HIV along a continuous dimension
using partial differential equations.
States of increased infectiousness associated with acute infection and late-stage disease
were simulated in \x{hiv/n.hiv.acute} and \x{hiv/n.hiv.late} studies, respectively.
\par
The relative risk of HIV transmission on ART was \xdm{art/rbeta},
representing an average ``on-treatment'' state in \x{art/n.rbeta.x.T} studies,
versus a ``virally suppressed'' state in \x{art/n.rbeta.x.V}.
Treatment failure due to drug resistance was simulated in \x{art/n.art.fail.any} studies, including:
\x{art/n.art.x.fail} where individuals experiencing treatment failure
were tracked separately from ART-naive; and
\x{art/n.art.r.fail} where such individuals
transitioned back to a generic ``off-treatment'' state.
Another \x{art/n.art.r.frop} studies included a similar transition
that was not identified as treatment failure versus ART cessation.
Transmissible drug resistance was simulated in \x{art/n.tdr} studies.
% ------------------------------------------------------------------------------
\subsubsection{Behavioural Effects}
\label{sss:res:beh}
Reduced sexual activity during late-stage HIV was simulated in \x{hiv/n.hiv.morb.any} studies,
including at least one state with:
complete cessastion of sexual activity (\x{hiv/n.hiv.morb.inact});
reduced rate/number of partnerships (\x{hiv/n.hiv.morb.np}); and/or
reduced rate/number of sex acts per partnership (\x{hiv/n.hiv.morb.vol}).
\par
Separate health states representing diagnosed HIV before treatment,
and on-treatment before viral suppression were simulated in
\x{art/n.art.x.dx} and \x{art/n.art.x.vlus} studies, respectively.
\x{art/n.bc.any} studies modelled behaviour changes following awareness of HIV+ status, including:
increased condom use (\x{art/n.bc.cond.any});
fewer partners per year (\x{art/n.dx.bc.np});
fewer sex acts per partnership (\x{art/n.dx.bc.vol});
serosorting (\x{art/n.dx.bc.ss}); and/or
a generic reduction in transmission probability (\x{art/n.dx.bc.gen}).
\par
ART cessation was simulated in \x{art/n.art.drop.any} studies, including:
\x{art/n.art.x.drop} where individuals no longer on ART
were tracked separately from ART-naive; and
\x{art/n.art.r.drop} where such individuals
transitioned back to a generic ``off-treatment'' state.
Another \x{art/n.art.r.frop} studies included a similar transition
that was not identified as treatment failure versus ART cessation.
% ------------------------------------------------------------------------------ 
\subsubsection{Network Effects}
\label{sss:res:network}
Populations were stratified by activity (different rates and/or types of partnerships formed)
in \x{act/n.act.def.np} studies, and by sex in \x{act/n.act.def.sex}.
The number of groups defined by sex and/or activity was \xdm{act/act.n};
and by activity alone (maximum number of groups among:
heterosexual women, heterosexual men, MSM, or overall if sex was not considered) was \xdm{act/act.n.z}.
The highest activity groups for females and males (possibly including FSW/clients) comprised
\xdm{act/hrw.p} and \xdm{act/hrm.p}\% of female and male populations, respectively.
\par
Turnover between activity groups and/or key populations
was considered in \x{act/n.turnover.any} studies,
of which \x{act/n.turnover.high} considered turnover of only
one specific high-activity group or key population.
Another \x{act/n.turnover.repl} studies simulated
movement only from lower to higher activity groups
to re-balance group sizes against disproportionate HIV mortality.
\par
Among \x{act/n.act.def.np} studies with activity groups, sexual mixing was assumed to be
assortative in \x{pt/n.mix.asso} and proportionate in \x{pt/n.mix.prop}.
Partnerships had equal probability of transmission in \x{pt/n.pt.gen} studies,
including all studies without activity groups.
Partnerships were defined by the activity groups involved in \x{pt/n.pt.grp} studies,
among which transmission was usually
lower in high-with-high activity partnerships than in low-with-low, due to
fewer sex acts (\x{pt/n.grp.vol}) and/or increased condom use (\x{pt/n.grp.condom}).
Transmission risk in high-with-low activity partnerships was defined by:
the susceptible partner (\x{pt/n.act.drive.sus});
the lower activity partner (\x{pt/n.act.drive.low});
the higher activity partner (\x{pt/n.act.drive.high}); or
both partners' activity groups (\x{pt/n.act.drive.mix});
yielding indeterminate, higher, lower, or intermediate
per-partnership transmission risk, respectively.
Partnerships were defined based on overlapping types, such that
different partnership types could be formed between the same two activity groups in \x{pt/n.pt.phen} studies.
All models with overlapping partnership types defined differential total sex acts and condom use between types.
\par
Age groups were simulated in \x{age/n.age.any} studies, among which,
the number of age groups was \xdm{age/age.n},
and \x{age/n.age.cts} studies simulated age along a continuous dimension.
Sexual mixing between age groups was assumed to be assortative
either with (\x{age/n.mix.offd}) or without (\x{age/n.mix.asso})
average age differences between men and women;
or proportionate (\x{age/n.mix.prop}).
Differential risk behaviour by age was modelled in \x{age/n.risk} studies.
% ------------------------------------------------------------------------------
\subsubsection{Coverage Effects}
\label{sss:res:cov}
Differential transition rates along the ART cascade were considered in
\x{cov/n.diff.any.any} studies, including differences between
sexes in \x{cov/n.diff.any.sex};
age groups in \x{cov/n.diff.any.age}; and
key populations in \x{cov/n.diff.any.kp}.
Another \x{cov/n.diff.any.any.j} studies did not simulate differential cascade transitions,
but specifically justified the decision using context-specific data.
Differences between sexes included rates of
HIV diagnosis (\x{cov/n.diff.dx.sex});
ART initiation (\x{cov/n.diff.art.i.sex}); and
ART cessation (\x{cov/n.diff.art.o.sex}),
with cascade engagement higher among women,
in most cases attributed to antenatal services.
Differences between age groups also affected
rates of diagnosis (\x{cov/n.diff.dx.age});
ART initiation (\x{cov/n.diff.art.i.age});
but not ART cessation (\x{cov/n.diff.art.o.age}). % MAN
Among key populations, \emph{lower} rates of
diagnosis, ART initiation, and retention were simulated in
\x{cov/n.diff.dx.kp}, \x{cov/n.diff.art.i.kp}, and \x{cov/n.diff.art.o.kp}
studies respectively, while \emph{higher} rates were simulated in
\x{cov/n.diff.dx.kp.H}, \x{cov/n.diff.art.i.kp.H}, and \x{cov/n.diff.art.o.kp.H}.
% ==============================================================================
\subsection{ART Prevention Impact}
\label{ss:res:api}
Dataset B comprised \x{n/n.a.api} studies,
including \x{n/n.s.api} scenarios of ART scale-up.
Relative incidence reduction with ART scale-up
as compared to a scenario without ART scale-up
was reported in \x{n/n.a.api.inc} studies (\x{n/n.s.api.inc} scenarios);
the proportion of cumulative infections averted due to ART scale-up
was reported in \x{n/n.a.api.chi} (\x{n/n.s.api.chi});
and \x{n/n.a.api.both} (\x{n/n.s.api.both}) reported both.
Some scenarios reported these outcomes on multiple time horizons.
\par
Figure~\ref{fig:api} summarizes each outcome versus time since ART scale-up,
stratified by a composite index of modelled risk heterogeneity.
Ecological-level analysis across scenarios by degree of risk heterogeneity
identified differences in proportions of infections averted,
but not in relative incidence reduction (Table~\ref{tab:api}).
The largest proportions of infections averted were reported from 
scenarios without risk heterogeneity (median [IQR]\% = \xd{api/chi/Risk.None}), followed by
scenarios with key populations prioritized for ART (\xd{api/chi/Risk.KP-(priority)}).
The smallest impact was observed in scenarios with
key populations who were not prioritized for ART (\xd{api/chi/Risk.KP-(same)})
and in models with risk heterogeneity but without key populations
(\xd{api/chi/Risk.Activity-(No-KP)}).
Only \x{n/n.s.api.both} scenarios from \x{n/n.a.api.both} studies provided both outcomes
\cite{Salomon2005,Abbas2006,Pretorius2010,Nichols2014,Barnighausen2016,Maheu-Giroux2017,Akudibillah2018}; % MAN
among which the pattern of incidence reduction versus modelled heterogeneity
was similar to the pattern of infections averted versus modelled heterogeneity
(Figure~\ref{fig:api:both}).
\begin{figure}
  \begin{subfigure}{0.5\textwidth}
    \centering
    \includegraphics[width=\textwidth]{{inc.s.Risk}.pdf}
    \caption{Reduction in HIV incidence}
    \label{fig:api:inc}
  \end{subfigure}%
  \begin{subfigure}{0.5\textwidth}
    \centering
    \includegraphics[width=\textwidth]{{chi.s.Risk}.pdf}
    \caption{Cumulative HIV infections averted}
    \label{fig:api:chi}
  \end{subfigure}
  \caption{Projected ART prevention benefits,
    stratified by factors of risk heterogeneity: whether models considered
    differences in sexual activity, key populations, and
    ART cascade prioritized to key populations}
  \label{fig:api}
  \floatfoot{
    The number of studies (scenarios) reporting
    incidence reduction, cumulative infections averted, both, or either was:
    \x{n/n.a.api.inc}~(\x{n/n.s.api.inc}),
    \x{n/n.a.api.chi}~(\x{n/n.s.api.chi}), 
    \x{n/n.a.api.both}~(\x{n/n.s.api.both}), and
    \x{n/n.a.api}~(\x{n/n.s.api}), respectively (Dataset~B).
    If any study included multiple scenarios of ART scale-up,
    then each scenario was included as a separate data point,
    but the size of each data point was reduced
    in proportion to the number of scenarios in the study.
    Some scenarios have multiple data points if multiple time horizons were reported.
    A small random offset was added to all data points to reduce overlap.
    KP: key populations;
    priority: cascade transitions were faster for at least one step among KP vs overall;
    same: cascade transitions were assumed the same speed in KP as overall;
    no scenarios in Dataset~B considered lower cascade among KP.}
\end{figure}
\par
Appendix~\ref{aa:res:api} and Table~\ref{tab:api} summarizes
ART prevention impacts (relative incidence reduction/proportion of infections averted),
stratified by other factors of risk heterogeneity, epidemic contexts, and intervention conditions.
ART prevention impacts were larger with longer time horizon, greater ART eligibility, and higher ART coverage.