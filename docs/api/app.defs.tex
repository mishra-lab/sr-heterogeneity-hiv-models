Data were obtained from (in order of precedence):
article text; article tables; article figures; appendix text; appendix tables; appendix figures;
and likewise for articles cited like ``the model is previously described in X''.
Data were assessed from figures with the help of a graphical measurement tool.%
\footnote{WebPlotDigitizer: \hreftt{https://apps.automeris.io/wpd/}}
\paragraph{Fitted Parameters}
For the values of fitted parameters, we used the posterior value as reported, including
the mean or median of the posterior distribution, or the best fitting value.
If the posterior was not reported, we used the mean or median of the prior distribution,
including the midpoint of uniform sampling ranges.
% ==============================================================================
\subsection{Epidemic Context}
\label{aa:defs:context}
Let $t_0$ be the time of ART scale-up/scenario divergence in the model.
\paragraph{HIV Prevalence}
As reported at $t_0$:
\emph{Low}: ${<1\%}$; \emph{Medium}: ${1-10\%}$; \emph{High}: ${>10\%}$.
\paragraph{Epidemic Phase}
Based on HIV incidence trend projected in the base case scenario between $t_0$ and roughly $t_0 + 10$ years.
Increasing (linear or exponential);
Increasing but stabilizing;
Stable;
Decreasing but stabilizing;
Decreasing (linear or exponential).
% ==============================================================================
\subsection{Risk Heterogeneity}
% ------------------------------------------------------------------------------
\subsubsection{Key Populations}
\label{aaa:defs:kp}
\paragraph{Female Sex Workers}
Any female activity group meeting 3 criteria:
representing ${< 5\%}$ of the female population; and
being ${< 1/3 \times}$ the size of client population or highest heterosexual male activity group; and
having ${> 50 \times}$ the partners of the lowest sexually active female activity group.
\cite{}
We also noted whether the authors described any activity groups as FSW.
If it was not possible to evaluate any criteria due to lack of data,
then we assumed the criteria was satisfied.
\paragraph{Clients of FSW}
Any male activity group meeting 2 criteria:
described as representing clients of FSW;
being ${> 3 \times}$ the size of the FSW population.
If group sizes were not reported,
then we assumed an activity group described as clients met the size criterion.
We also noted whether clients were described as
comprising a proportion of another male activity group.
\paragraph{Men who have Sex with Men}
Any male activity group(s) described by the authors as MSM.
\paragraph{People who Inject Drugs}
Any activity group(s) described by the authors as PWID.
\paragraph{Adolescent Girls and Young Women}
TODO
\paragraph{Mobile Populations}
TODO
% ------------------------------------------------------------------------------
\subsubsection{Activity Groups}
Activity groups were 

Activity groups were counted separately for
heterosexual women, heterosexual men, and MSM.

Age groups were counted separately, even when age influenced sexual activity.
% ------------------------------------------------------------------------------
\subsubsection{Partnership Types}
\label{aaa:defs:pt}
\paragraph{Generic}
If only one type of partnership is simulated in the model.
\paragraph{Main}
\paragraph{Casual}
\paragraph{Sex Work}
\paragraph{Transactional}
\paragraph{Defined by the Partners}
% sex workers have main partnerships: Peltzer2004, Luseno2009?
% ------------------------------------------------------------------------------
\subsubsection{Group Turnover}
\label{aaa:defs:turnover}
Turnover refers to movement of individuals between
activity groups and/or key populations reflecting sexual life course.
We defined the following five classifications of turnover:
\emph{N/A}: not applicable if no activity groups were modelled;
\emph{None}: no movement between activity groups;
\emph{High-Activity}: only movement between one high activity group or key population
  and one other activity group;
\emph{Multiple}: movement between multiple pairs of risk groups;
\emph{Replacement}: only movement from low to high activity,
  to maintain high activity group size(s) against disproportionate HIV mortality.
% ==============================================================================
\subsection{Antiretroviral Therapy}
% ------------------------------------------------------------------------------
\subsubsection{Transmission Reduction}
The reduction in HIV transmission due to ART
was defined as the relative reduction in probability of transmission
among individuals who are virally suppressed
(0 is perfect prevention, 1 is no effect).

% ------------------------------------------------------------------------------
\subsubsection{States}
\paragraph{Diagnosed}
Individuals are aware of their HIV+ status, but have not yet started ART.
\paragraph{Not Yet Virally Suppressed}
Individuals have started ART, but are not yet virally suppressed.
\paragraph{Treatment Failed Due to Resistance}
Individuals have stopped experiencing the benefits of ART due to development of resistance;
resuming ART is defined by or implies a 2+ line regimen.
\paragraph{Off ART}
Individuals are not taking ART due to reasons unrelated to resistance;
it may be possible to resume ART, possibly with the same regimen.
% ------------------------------------------------------------------------------
\subsubsection{Behaviour Change}
\paragraph{HIV Morbidity}
Any reduced sexual activity in late-stage HIV states representing impact of symptoms, including:
fewer sex acts per sex partnership; or fewer partnerships per year.
\paragraph{HIV Counselling}
Any sexual behaviour change associated with HIV testing and counselling (HTC),
applied to individuals in the diagnosed and/or on-ART states, including:
increased condom use;
fewer sex acts per sex partnership;
fewer partnerships per year;
or a generic reduction in per-act/per-partnership transmission probability due to counselling.
\paragraph{Morbidity Reduction}
Must first include HIV morbidity.
Morbidity reduction behaviour change is
any return towards normal levels of sexual activity associated with ART due to reduced symptoms.
% ------------------------------------------------------------------------------
\subsubsection{Transmitted Resistance}
Any consideration of 1+ strains of HIV which are transmitted and for which ART has reduced benefits.
We did not document the number of resistant strains,
or characteristics of resistance and transmissibility.
