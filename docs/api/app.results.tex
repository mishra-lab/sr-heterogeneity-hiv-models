% ==============================================================================
\subsection{Context}\label{aa:res:context}
\begin{figure}[h]
  \centering
  \includegraphics[width=0.8\linewidth]{map-n-vs-plhiv}
  \caption{Map showing number of studies (of \na total)
    applying HIV transmission modelling in each country vs
    the number of people living with HIV (PLHIV, millions)}
  \label{fig:map}
\end{figure}
% ==============================================================================
\subsection{Risk Heterogeneity}\label{aa:res:risk}
% ------------------------------------------------------------------------------
\subsubsection{Distributions}
\twocolumn
\filedef{\projectroot/out/tex/config/plot.list.dist}
\foreach \var/\title in \filedata{\hspace{0pt}%
  \begin{minipage}[t][.33\textheight][t]{\linewidth}
    \includegraphics[width=\linewidth]{{d.\var}.pdf}
    \captionof{figure}{\title}
    \label{fig:d:\var}
  \end{minipage}}
\onecolumn
\clearpage
% ==============================================================================
\subsection{ART Prevention Impact}\label{aa:res:api}
% ------------------------------------------------------------------------------
\subsubsection{Figures}
The following figures illustrate the projected ART prevention impact (Dataset~B),
stratified by various factors of heterogeneity and intervention conditions (colours).
The left panels show the relative reduction in HIV incidence rate;
the right panels show the relative reduction in cumulative new HIV infections;
both as compared to a base-case scenario reflecting status quo.
The number of studies (scenarios) reporting
incidence reduction, cumulative infections averted, both, or either was:
\x{n/n.a.api.inc}~(\x{n/n.s.api.inc}),
\x{n/n.a.api.chi}~(\x{n/n.s.api.chi}), 
\x{n/n.a.api.both}~(\x{n/n.s.api.both}), and
\x{n/n.a.api}~(\x{n/n.s.api}), respectively.
If any study included multiple scenarios of ART scale-up,
then each scenario was included separately,
but the size of each data point was reduced
in proportion to the number of scenarios;
so studies with only one scenario have the largest data points.
Some scenarios have multiple data points if multiple time horizons were reported.
If any factor could not be quantified due to missing data or varying values,
the data point is grey.
A small random offset has been added to the data points to reduce overlap.
\filedef{\projectroot/out/tex/config/plot.list.api}
\foreach \var/\title in \filedata{
  \begin{figure}[H]
    \begin{subfigure}{0.5\linewidth}
      \includegraphics[width=\linewidth]{{inc.s.\var}.pdf}
    \end{subfigure}%
    \begin{subfigure}{0.5\linewidth}
      \includegraphics[width=\linewidth]{{chi.s.\var}.pdf}
    \end{subfigure}
    \caption{\title}
    \label{fig:api:\var}
  \end{figure}
}
\begin{figure}[H]
  \begin{subfigure}{0.5\linewidth}
    \includegraphics[width=\linewidth]{{inc.s.Risk.both}.pdf}
    \caption{Reduction in HIV incidence (\%)}
  \end{subfigure}%
  \begin{subfigure}{0.5\linewidth}
    \includegraphics[width=\linewidth]{{chi.s.Risk.both}.pdf}
    \caption{Cumulative HIV infections averted (\%)}
  \end{subfigure}
  \caption{Projected ART prevention benefits,
    stratified by factors of risk heterogeneity: whether models considered
    differences in sexual activity, key populations, and
    ART cascade prioritized to key populations.
    Subset of studies reporting both outcomes.}
  \label{fig:api:both}
\end{figure}
% ------------------------------------------------------------------------------
\subsubsection{Table}
Table~\ref{tab:api} summarizes the median [IQR] projected ART prevention impact (Dataset~B),
stratified by various factors of heterogeneity and intervention conditions.
Reported p-values for each factor are from non-parametric Kruskal-Wallis tests
for differences in ART prevention impact under at least one of the factor levels.
\begin{table}[H]
  \caption{Projected ART prevention benefits,
    stratified by factors of risk heterogeneity and conditions of ART scale-up}
  \centering
  \footnotesize
\newlength{\ntab}\setlength{\ntab}{1ex}
\setlength{\tabcolsep}{1ex}
\newcommand{\itab}[2]{
  \x{api/#1/#2.q2} & ( \x{api/#1/#2.q1}, \x{api/#1/#2.q3} ) & \x{api/#1/n.#2}
}
\newcommand{\xtab}[1]{\itab{inc}{#1} & & \itab{chi}{#1} & }
\newcommand{\ptab}[2]{\itab{inc}{#1} & \x{api/inc/#2.pval} & \itab{chi}{#1} & \x{api/chi/#2.pval}}
\begin{tabular}{llccrrccrr}
  \toprule
                    &        & \multicolumn{4}{c}{HIR (\%)} & \multicolumn{4}{c}{HCIA (\%)} \\
  \cmidrule(rl){3-6}\cmidrule(rl){7-10}
  Factor            & Level  &  Median & (IQR) & N\tn{a} & p\tn{b} & Median & (IQR) & N\tn{a} & p\tn{b}\\
  \midrule
%\begin{tabular}{lll} % TEMP
	Time Horizon         & 0-10         & \ptab{t.cat.0}{t.cat}                            \\
	(years)              & 11-20        & \xtab{t.cat.10}                                  \\
	                     & 21-30        & \xtab{t.cat.20}                                  \\
	                     & 31+          & \xtab{t.cat.30}                                  \\[\ntab]
	HIV Prevalence       & 0-1          & \ptab{api.prev.cat.Low}{api.prev.cat}            \\
	(\%)                 & 1-10         & \xtab{api.prev.cat.Mid}                          \\
	                     & 10+          & \xtab{api.prev.cat.High}                         \\
	\midrule
	RR Transmission      & 0.0-0.039    & \ptab{art.rbeta.cat.0}{art.rbeta.cat}            \\
	on ART               & 0.4-0.099    & \xtab{art.rbeta.cat.0.04}                        \\
	                     & 0.1+         & \xtab{art.rbeta.cat.0.1}                         \\[\ntab]
	CD4 Threshold for    & 200          & \ptab{art.cd4.200}{art.cd4}                      \\
	ART Initiation       & 350          & \xtab{art.cd4.350}                               \\
	                     & 500          & \xtab{art.cd4.500}                               \\
	                     & Any          & \xtab{art.cd4.All}                               \\[\ntab]
	ART Coverage         & 0-59         & \ptab{art.cov.cat.0}{art.cov.cat}                \\
  (\%)                 & 60-84        & \xtab{art.cov.cat.0.6}                           \\
                       & 85+          & \xtab{art.cov.cat.0.85}                          \\
  \midrule
	Acute Infection      & No           & \ptab{hiv.x.acute.N}{hiv.x.acute}                \\
	                     & Yes          & \xtab{hiv.x.acute.Y}                             \\[\ntab]
	Late-Stage Infection & No           & \ptab{hiv.x.late.N}{hiv.x.late}                  \\
	                     & Yes          & \xtab{hiv.x.late.Y}                              \\[\ntab]
	Trans. Drug Resist.  & No           & \ptab{art.tdr.N}{art.tdr}                        \\
	                     & Yes          & \xtab{art.tdr.Y}                                 \\
	\midrule
	HIV Morbidity        & No           & \ptab{hiv.morb.any.N}{hiv.morb.any}              \\
	                     & Any          & \xtab{hiv.morb.any.Y}                            \\[\ntab]
	HTC Behav. Change    & No           & \ptab{bc.any.N}{bc.any}                          \\
	                     & Any          & \xtab{bc.any.Y}                                  \\
	\midrule
	Sex Stratification   & No           & \ptab{act.def.sex.N}{act.def.sex}                \\
	                     & Yes          & \xtab{act.def.sex.Y}                             \\[\ntab]
	Activity Groups      & None         & \ptab{act.kp.none}{act.kp}                       \\
	\& Key Populations   & Yes (no KP)  & \xtab{act.kp.some.no.kp}                         \\
	                     & Yes + KP     & \xtab{act.kp.any.kp}                             \\[\ntab]
	Activity Turnover    & No           & \ptab{act.turn.any.N}{act.turn.any}              \\
	                     & Yes          & \xtab{act.turn.any.Y}                            \\[\ntab]
	Partnership Types    & Generic      & \ptab{pt.def.gen}{pt.def}                        \\
	                     & by Groups    & \xtab{pt.def.grp}                                \\
	                     & Phenom.      & \xtab{pt.def.phen}                               \\
	\midrule
	Differential KP      & Priority     & \ptab{diff.any.kp.cat.priority}{diff.any.kp.cat} \\
	Cascade              & Same         & \xtab{diff.any.kp.cat.same}                      \\
	                     & Gaps         & \xtab{diff.any.kp.cat.gaps}                      \\ % MAN
	\bottomrule
\end{tabular}
\floatfoot{
  \tnt[a]{N: number of scenarios.}
  \tnt[b]{P-values from non-parametric Kruskal-Wallis test for
    whether two or more independent samples originate from the same distribution.}
  HIR: HIV incidence reduction;
  CHIA: cumulative HIV infections averted;
  RR: relative risk;
  HTC: HIV testing and counselling;
  KP: key populations.
  Factor definitions are given in Appendix~\ref{a:defs}.
}
  \label{tab:api}
\end{table}