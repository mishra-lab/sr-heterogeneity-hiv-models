\section{Discussion}
\label{s:disc}
Model-based evidence continues to support
evaluation and mechanistic understanding of ART prevention impacts.
Such evidence may be sensitive to modelling assumptions about risk heterogeneity.
Via scoping review, we found that stratification by sexual activity and key population(s)
was considered in approximately 2/3 and 2/5 of studies to date, respectively;
1/3 considered risk group turnover and 1/4 considered differential ART cascade by any risk group.
In multivariate ecological analysis, we found that
projected incidence reductions and propoportions of infections averted
were minimally affected by risk heterogeneity directly,
but were reduced by risk group turnover and differential ART cascade.
\par
Within-person variability in sexual risk has been illustrated among key populations,
including MSM, FSW, and clients of FSW \cite{Fazito2012,Romero-Severson2012,Roberts2020},
as well as in the wider population \cite{Houle2018}.
This risk variability is often reflected in compartmental models as risk group turnover.
Previous modelling suggested that
turnover could \emph{increase} the prevention benefits of treatment \cite{Henry2015};
however, the model in \cite{Henry2015} was calibrated to overall equilibrium prevalence,
allowing the reproduction number to decrease with increasing turnover.
By contrast, when calibrating to group-specific prevalence with turnover,
greater risk heterogeneity is inferred with vs turnover,
and the reproduction number may actually increase \cite{Knight2020}.
Turnover of higher risk groups can also reduce ART coverage in those groups through
net outflow of treated individuals, and net inflow of susceptible individuals,
some of whom then become infected \cite{Knight2020}.
The proportion of onward transmission prevented through ART may thus be reduced via turnover.
These findings suggest that turnover is important, and as such,
models would benefit from surveys, cohorts, and repeated population size estimates
that can provide data on individual-level trajectories of sexual risk,
such as duration in sex work \cite{Watts2010}.
\par
Most models assumed equal ART cascade transition rates across subgroups,
including diagnosis, ART initiation, and retention.
However, recent data suggest differential ART cascade by sex, age, and key populations
\cite{Lancaster2016,Stannah2019,Ma2020,Green2020}.
These differences may stem from the unique needs of subpopulations
and is one reason why differentiated ART services are a core component of HIV programs
\cite{Chikwari2018,Ehrenkranz2019}.
Moreover, barriers to ART may intersect with transmission risk, particularly among key populations,
due to issues of stigma, discrimination, and criminalization \cite{Ortblad2019,Baral2019}.
Our ecological analysis estimated that
differences in cascade by sex (lower among men) or risk (key populations prioritized)
had a large influence on projected ART prevention benefits.
Thus, opportunities exist to incorporate differentiated cascade data,
examine the intersections of intervention and risk heterogeneity, and
to consider the impact of HIV services as delivered on the ground.
Similar opportunities were noted regarding modelling of pre-exposure prophylaxis in SSA \cite{Case2019}.
Depending on the research question, the modelled treatment cascade may need
to include more cascade steps and states related to treatment failure/discontinuation.
\par
Key populations often reflect intersections of risk heterogeneity, turnover, and cascade differences.
For example, a sexual network comprising FSW with high turnover and FSW clients with low ART coverage
could remain outside the reach of ART as prevention.
Key populations continue to experience disproportionate risk of HIV,
even in high-prevalence epidemics \cite{Baral2012,Beyrer2012},
and models suggest that unmet needs of key populations
play an important role in overall epidemic dynamics \cite{Bekker2015,Stone2021}.
With the growth of data collected about communities most affected by HIV,
there are opportunities to more consistenly capture
data-driven heterogenetity in risks and intervention access among key populations in transmission models.
For example, we found that the number of modelled clients per female sex worker, and
the relative rate of partnership formation among female sex workers versus other women
did not always reflect available data syntheses for sex work \cite{Watts2010,Scorgie2012}.
Similarly, among studies with different partnership types, only 1/5 modelled
main/spousal partnerships---with more sex acts/lower condom use---between two higher risk individuals,
while 4/5 modelled only casual/commercial partnerships among higher risk individuals.
However, data suggest that female sex workers form main/spousal partnerships
with regular clients and boyfriends/spouses from higher risk groups \cite{Scorgie2012}.
Improved modelling and prioritization of sevices designed to reach key populations
will rely on continued investment in community-led data collection for hard-to-reach populations.
\par
Our scoping review has several limitations.
First, we focused on classically defined key populations,
although other priority groups like mobile populations and adolescent girls and young women
will remain important for the prevention benefits of ART \cite{Tanser2015,Dellar2015}.
Second, our conceptual framework for risk heterogeneity did not explicitly examine
heterogeneity related to anal sex, which is associated with higher probability of HIV transmission,
nor structural risk factors like violence \cite{Silverman2011,Baggaley2013}.
Third, we did not extract whether models were calibrated,
and if so, which parameters were fixed vs fitted.
If certain parameters were fitted, it could explain some counterintuitive effect estimates.
% SM: [from response letter] I don't follow ... what do we mean extract data on model fitting?
%     did not examine how models were calibrated or what the calibration targets were? or...?
% JK: We really didn't extract any of that, but as it relates to effect estimates (and so worth mentioning here),
%     I've revised above to hopfeully be more clear?
For example, modelling increased infectiousness in late-stage HIV reduced ART prevention impacts.
However, in most studies, newly ART-eligible patients via scale-up had earlier stage HIV;
therefore, such patients would have lower modelled infectiousness than late-stage HIV,
and lower infectiousness than in a model with uniform infectiousness fitted to the same data.
A similar mechanism could explain increased ART prevention impacts when including acute infection.
Finally, the strength of our multivariate analysis was limited by
the small number of studies/scenarios relative to the number of factors explored.
\par
In conclusion, model-based evidence of ART prevention impacts could likely be improved by:
1) consistenly including risk group turnover,
   as a determinant of inferred risk heterogeneity during model calibration, and
   to reflect challenges to maintaining ART coverage among risk groups with high turnover;
% SM: [from response letter] reads bit vague. what do we mean by prevention challenges because of dynamic nature?
% JK: is this better? I wonder if it's worth including the first part (inferred risk heterogeneity)
%     as it's kind of hard to summarize succinctly
2) integrating data on differences in ART cascade between sexual risk groups,
   to reflect services as delivered on the ground; and
3) capturing heterogenetiy in risks in the context of key populations,
% JK: "capturing heterogenetiy in risks in the context of key populations"
%     reads maybe a little unclear/abstract to me, can we say:
%     "capturing risk heterogeneity related to key populations" ?
   to reflect intersections of transmission risk and barriers to HIV services
   that may undermine the prevention benefits of ART.
Model comparison studies like \cite{Dodd2010,Hontelez2013} that explore
the influence of these factors in detail would also be welcome.