Via scoping review, we found that representations of risk heterogeneity varied widely across
transmission modelling studies of ART in SSA, with
stratification by sexual activity and key populations considered in approximately
two thirds and one third of models, respectively.
We also found that the projected proportions of infections averted due to ART scale-up were
larger under assumptions of homogeneous risk or prioritized ART to key populations,
as compared to heterogeneous risk or without prioritized ART to key populations.  %SM: summary paragraph very clear and succint
Three notable themes emerged from our review.
\par %SM: this paragraph needs a bit of work for clarity and to a bit less editoriliazing language (which I tend to do too!). have edited, but please clarify remaining sentence; and I'll look at this para again quickly before we submit.
First, modelling studies have an opportunity to keep pace with growing epidemiological data on risk heterogeneity.
For example, 41\% of the studies reviewed included at least one key population, %SM: modeling studies yes?
such as female sex workers and men who have sex with men.
Key populations continue to experience disproportionate risk of HIV, even in high-prevalence epidemics \cite{TODO},
and models examining the unmet needs of such key populations have suggested that
these unmet needs play an important role in the overall epidemic dynamics \cite{Stone2021,Bekker2015}.
Furthermore, the we found that the number of modelled clients per female sex worker, and
the relative rate of partnership formation among female sex workers versus other women
did not always reflect the available data on sex work \cite{Watts2010,Scorgie2012}.
Similarly, among studies with different partnership types, only 20\% modelled
main/spousal partnerships---with more sex acts and lower condom use---between two higher risk individuals,
while 80\% modelled only casual or commercial partnerships among higher risk individuals.
However, data suggest female sex workers may form main/spousal partnerships
with regular clients and boyfriends/spouses from higher risk groups \cite{Scorgie2012}.
Thus, with increasing data available on various factors of risk heterogeneity,
there are opportunities to include these data in future models,
and to study how multiple factors might act together to influence projections of ART impact
through nested model comparison studies \cite{Dodd2010,Hontelez2013}.
\par
Second, most models to date assumed that rates of transitions along the ART cascade are
similar or equivalent across subgroups. Yet, data suggest differential ART cascade by sex, age, and key populations
\cite{Mountain2014,Lancaster2016,Green2020}
Thus, a further opportunity exists to incorporate these "real-world' ART data because barriers to ART may intersect with transmission risk,
particularly among key populations, due to issues of stigma, discrimination, and criminalization.  %SM: excellent sentence and logic!
\cite{Ortblad2019,Baral2019}
A recent review of modelling of HIV pre-exposure prophylaxis in SSA\cite{Case2019} %SM: would not use PrEP acronym as trying to limit number of acroynms in paper :) 
identified similar opportunities to better leverage programmatic data.
Often, context-specific key population cascade data to support these efforts are often lacking.\cite{Mountain2014}  %SM: not sure this is justified as a statement when citing a 2014 paper. there are more up to date paper to cite (e.g. check out MC's paper on HIV cascade among MSM in SSA [lancet HIV]; and more recent reviews I think by Stef and team? ; or if not able to find - then a good reference = UNAIDS Atlas on Key populations to cite for this sentence (has ART coverage data): https://kpatlas.unaids.org/dashboard
These data suggest that different metrics of the cascade, such as the proportion of persons living with HIV who are diagnosed and aware, 
ART access/use, and ART retention or viral suppression are sometimes lower among key populations as compared to the wider population. %SM: eg. cite Sheree's study from South Africa on ART coverage among FSW? Huitng's Transitions paper on AGWY/young FSW?
Differential ART cascade often stems from the unique needs of subsets of the population by a variety of factors; and
is one of the reasons that differentiated ART services are a core component of HIV programs. %SM: cite papers on differentiated ART services for different populations (see Huiting's paper inroduciton as an example for differeniated HIV testing for example).https://www.differentiatedservicedelivery.org/
Differentiated ART services are designed to meet differential unmet needs.
Thus, there exists an opportunity to further examine the intersection of intervention and risk heterogeneity, and also 
an opportunity for models of ART impact to consider the impact of HIV services as they are being delivered on the ground.
Finally, depending on the research question, modelled representations of the treatment cascade may also need expansion
to include more cascade steps and states related to treatment failure and discontinuation.

\par %SM: great job with this implications section
Third, based on ecological analysis of scenarios, we found that
modelling assumptions about risk and intervention heterogeneity
may influence the projected proportion of future infections averted by ART.
% JK: this feels like repeated results, but I don't know how else to introduce this paragraph...%SM: I think the results of Obj 3 now lead to this, so appropriate to mention here in discussion to start off this paragraph 
We did not find similar evidence for relative incidence reduction due to ART,
but studies reporting these two outcomes were largely distinct.
Among studies reporting both, the overall pattern was consistent. %SM: cite those specific studies here
These findings highlight the limitations of ecological analysis to estimate
the potential influence of modelling assumptions on projected ART prevention benefits,
and motivate systematic model comparison studies to better answer this question. %SM: didn't Hontalez paper do that though a bit? and so should it not be cited here? what does "this question' refer to?
Our ecological analysis also suggested that the anticipated ART prevention impacts from homogeneous models
may be achievable in the context of risk heterogeneity
if testing and treatment resources are prioritized to higher risk groups. 
% JK: I remember reading a full study that did something like our eSwatini 10-10-10 analysis,
%     and I want to mention & cite it here, but for the life of me I can't find it...

\par %SM: woudl keep this short. in reading the limitations section, felt like we should have done them. the key is what would effect our key messages/ interpretation in the context of a scoping review. so brief is better here since we are not making specific inferences.
Limitations of our scoping review includes our examination of only a few key populations.
In our conceptual framework for risk heterogeneity, we did not explicitly examine heterogeneity 
by type of sex acts (i.e. anal sex) which is associated with higher probability of HIV transmission; 
nor structural factors such as violence which lead to increased HIV risks \cite{Silverman2011,Baggaley2013}.
We were also limited in drawing inference on the influence of risk heterogeneity across scenarios and models in the context of the 
scoping review and large number of differences between scenarios and models.

\par
In conclusion, representations of risk heterogeneity vary widely  %SM: i like this conclusion paragraph; clear and focused with concrete but easy to understand next steps; sets up your thesis well!
among models used to project the prevention impacts of ART in SSA.
Such differences may partially explain the large variability in the projected impacts.
Opportunities exist to incorporate new and existing data on
the intersections of risk heterogeneity and intervention heterogeneity.
Moving forward, systematic model comparison studies are needed to
estimate and understand the relative influence of various modelling assumptions on the prevention impacts of ART.
