%SM: What are you 3 key messages? I found that the discussion was more of a critique rather than a discussion. I could see this being intpreted as assessing value vs. learning from what others have done? without a benchmark, how can we assign value? how does your work compare to other reviews (e.g. the PrEP one) ? what contribution does this paper / findings make? THe discussion reads more lke your opinion or editoral piece on what should be included but we did not actually test that out.
%SM: I'd like to discuss the 3 key messages again at your PM and then you can revise the discussion section and I will re-read and edit the intro & discussion together. 

Via scoping review, we found that representations of risk heterogeneity varied widely across transmission modeling studies of ART in SSA, with [summarize main findings from Aim 1 and 2 very briefly].
We also found that patterns of population-level prevention impacts of ART scale-up projected by models were larger in the context of ... summarize main key message from Aim 3.
\paragraph{Modelled Factors of Heterogeneity}
% Overall, reporting of model structure and parameter values was good,
% though complete model inputs for Spectrum and Optima were not always provided.
% JK: ^ not sure if/where to include this. SM: but if you didn't use a reporting checklist then who determines if good or bad?
Three areas for potential improvement emerged among modelled factors of heterogeneity.
First, highest risk groups may not be modelled to reflect
sufficiently disproportionate transmission risk so as to adequately
capture core group effects.
For example, the highest activity groups among women and men were still often relatively large
(Figures~\ref{fig:d:act.HRW.p}~and~~\ref{fig:d:act.HRM.p}),
and even among models with key populations,
heterogeneity \emph{within} key populations was rarely considered.
% \cite{} % TODO - FSW heterogeneity paper
Many models also implicitly excluded the possibility of
``main/spousal'' partnerships (with more sex and lower condom use)
forming between two higher activity individuals.
% \cite{} % TODO - FSW partnerships paper
\par
Second, representations of the ART cascade tended to overlook
key steps such as diagnosis, linkage to care, and in some cases treatment failure/dropout;
\cite{Mountain2014a}
some of these simplifications might result in
overly optimistic rates of ART initiation and viral suppression.
% \cite{} % TODO - cascade leaks paper
However, equally overlooked were potential prevention benefits
due to behaviour change from HIV testing and counselling.
% \cite{} % TODO - HTC paper (e.g. Thembisa appendix)
\par
Third, intersectionality of transmission risks and cascade progression were rarely considered,
such that rates of diagnosis, linkage, viral suppression, and retention
were usually assumed equal across sexes, activity groups, age groups, and key populations.
However, evidence of differential cascade engagement by sex and age is mounting.
\cite{Witzel2017,Mayanja2018,Green2020}
In some contexts, reported cascade coverage among female sex workers
may approach or surpass that of the wider population,\cite{Mountain2014,Hakim2018}
but the same is unlikely for men who have sex with men.\cite{Mountain2014,Stannah2019}
Moreover, key population cascade data are often obtained through
prioritized research and programs that improve coverage,
suggesting that unmeasured key population cascades could be lowest.\cite{Mountain2014}
% reference figure 2 results?
\paragraph{Significance}
The prevention impacts of ART will continue to grow under
increasing adoption of universal test and treat.
Maximizing these impacts will require
continuous integration of context-specific data and assumptions
into transmission models to understand challenges and opportunities.
Priorities for such data could include detailed cascade data,
stratified by sex, age, and key populations.
In the absence of clear patterns relating
modelled factors of risk heterogeneity to projected ART prevention benefits,
questions also remain as to which factors are most influential, and in which contexts.
Such questions should be explored in step-wise model structure comparison studies,
such as in \textcite{Andrews2012,Hontelez2013,Eaton2014a}.
\paragraph{Limitations}
Our review has four main limitations.
First, the key populations considered in our analysis did not include
adolescent girls and young women, transgender people, or mobile populations,
despite the fact that such populations may face
similar risks of transmission and barriers to care as other key populations.
\cite{Tanser2015,Dellar2015} % TODO: TG ref
We also did not document representations of violence, coercion, or anal sex,
which may similarly coincide with transmission risks and barriers to care.
\cite{Silverman2011,Baggaley2013}
Future work should explore representations of
such groups and phenomena in transmission models.
% JK: might still go back and include if time...
Second, we did not document which (if any) model parameters were fitted
or to which calibration targets.
As shown by \textcite{Eaton2014a,Knight2020}, model fitting can produce
% JK: self-cite = cringe? We really did show this though :\
parameter values which compensate for differences in model structure,
and thereby underpin counterintuitive associations between model structure and modelling results.
Third, we did not compare modelled factors of heterogeneity to
context-specific epidemiological data,
which in some cases may justify model assumptions of homogeneity.
However, we did note when authors specifically justified such assumptions.
Finally, we did not estimate the effect size of
individual heterogeneity factors on the projected ART prevention impact.
Such an effect estimate could be biased by confounding factors in univariate analysis,
while exploratory work found challenges in multivariate analysis of our data,
due to the small number of scenarios and high data collinearity.
We were further discouraged from estimating effects after noting opposite trends in
incidence reduction and cumulative infections averted for several factors,
suggesting the potential for finding spurious patterns.
