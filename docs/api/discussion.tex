Via scoping review, we found that representations of risk heterogeneity varied widely across
transmission modelling studies of ART intervention in SSA, with
stratification by sexual activity and key populations considered in approximately
2/3 and 2/5 of models, respectively.
We also found that the projected proportions of infections averted due to ART scale-up were
larger under assumptions of homogeneous risk or prioritized ART to key populations,
as compared to heterogeneous risk or without prioritized ART to key populations.
Three notable themes emerged from our review.
\par
First, modelling studies have an opportunity to keep pace with growing epidemiological data on risk heterogeneity.
For example, 41\% of the modelling studies reviewed included at least one key population, such as FSW and or MSM.
Key populations continue to experience disproportionate risk of HIV, even in high-prevalence epidemics \cite{AIDSinfo},
and models examining the unmet needs of key populations suggest that
these unmet needs play an important role in overall epidemic dynamics \cite{Stone2021,Bekker2015}.
Furthermore, the we found that the number of modelled clients per female sex worker, and
the relative rate of partnership formation among female sex workers versus other women
did not always reflect the available data \cite{Watts2010,Scorgie2012}.
Similarly, among studies with different partnership types, only 20\% modelled
main/spousal partnerships---with more sex acts/lower condom use---between two higher risk individuals,
while 80\% modelled only casual/commercial partnerships among higher risk individuals.
However, data suggest female sex workers may form main/spousal partnerships
with regular clients and boyfriends/spouses from higher risk groups \cite{Scorgie2012}.
Thus, future models can continue to include emerging data on these and other factors of heterogeneity,
while nested model comparison studies can study how
multiple factors might act together to influence projections of ART impact \cite{Dodd2010,Hontelez2013}.
\par
Second, most models assumed equal ART cascade transition rates across subgroups,
including diagnosis, ART initiation, and retention.
Recent data suggest differential ART cascade by sex, age, and key populations
\cite{Lancaster2016,Schwartz2017,Ma2020,Green2020}.
These differences may stem from the unique needs of population subgroups
and is one reason why differentiated ART services are a core component of HIV programs
\cite{Chikwari2018,Ehrenkranz2019}.
Moreover, barriers to ART may intersect with transmission risk, particularly among key populations,
due to issues of stigma, discrimination, and criminalization \cite{Ortblad2019,Baral2019}.
Thus, further opportunities exist to: incorporate differentiated cascade data,
examine the intersections of intervention and risk heterogeneity, and
to consider the impact of HIV services as they are delivered on the ground.
Similar opportunities were noted regarding modelling of pre-exposure prophylaxis in SSA \cite{Case2019}.
Finally, depending on the research question, the modelled treatment cascade may need expansion
to include more cascade steps and states related to treatment failure/discontinuation.
\par
Third, based on ecological analysis of scenarios, we found that
modelling assumptions about risk and intervention heterogeneity
may influence the projected proportion of infections averted by ART.
We did not find similar evidence for relative incidence reduction due to ART,
but studies reporting both outcomes were largely distinct.
Among studies reporting both, the overall pattern was consistent
\cite{Salomon2005,Abbas2006,Pretorius2010,Nichols2014,Barnighausen2016,Maheu-Giroux2017,Akudibillah2018}.
These findings highlight the limitations of ecological analysis to estimate
the potential influence of modelling assumptions on projected ART prevention benefits,
and motivate additional model comparison studies to better quantify this influence,
such as \cite{Dodd2010,Hontelez2013}.
Our ecological analysis also suggested that the anticipated ART prevention impacts from homogeneous models
may be achievable in the context of risk heterogeneity
if testing and treatment resources are prioritized to higher risk groups.
\par
Limitations of our scoping review include our examination of only a few key populations.
In our conceptual framework for risk heterogeneity, we did not explicitly examine heterogeneity
by type of sex act (i.e. anal sex) which is associated with higher probability of HIV transmission,
nor structural risk factors like violence \cite{Silverman2011,Baggaley2013}.
The large number of differences between scenarios in the scoping review context
also limited our ability to infer the influence of risk heterogeneity across scenarios.
\par
In conclusion, representations of risk heterogeneity vary widely
among models used to project the prevention impacts of ART in SSA.
Such differences may partially explain the large variability in projected impacts.
Opportunities exist to incorporate new and existing data on
the intersections of risk and intervention heterogeneity.
Moving forward, systematic model comparison studies are needed to
estimate and understand the influence of various modelling assumptions on ART prevention impacts.
