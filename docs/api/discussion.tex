Via scoping review, we found that representations of risk heterogeneity varied widely across
transmission modelling studies of ART in SSA, with
% majority of studies examining medium/high prevalence epidemics in East and Southern Afirca, and
stratification by sexual activity and key populations considered in approximately
two thirds and one third of models, respectively.
We also found that the projected proportions of infections averted due to ART scale-up were
larger under assumptions of homogeneous risk or prioritized ART to key populations,
as compared to heterogeneous risk or without prioritized ART to key populations.
Three notable themes emerged from our review.
\par
First, models can go further in keeping pace with epidemiological data on risk heterogeneity.
For example, the majority of models did not consider key populations.
Even in high-prevalence epidemics, especially with declining overall incidence,
core groups continue to influence epidemic dynamics.\cite{Brown2019,Ortblad2019}
Among studies that considered female sex workers,
parameterization of that modelled risk group did not always
reflect epidemiological definitions, such as
representing a smaller population than their clients,
and having much higher sexual activity than other women in the model.
\cite{Watts2010,Scorgie2012}
Many models also defined partnership types based on the risk groups involved,
precluding the formation of longer partnerships with lower condom use between two higher risk individuals.
In fact, female sex workers may form such partnerships with regular clients
and boyfriends/spouses from higher risk groups.\cite{Scorgie2012}
Overall, it is likely important that models reflect established factors of risk heterogeneity
because nested model comparison studies suggest that multiple factors
can each influence the challenge of epidemic control through ART scale-up.
\cite{Dodd2010,Hontelez2013}
\par
Second, emerging evidence of differential ART cascade by sex, age, and key populations
\cite{Mountain2014,Lancaster2016,Green2020}
has not yet been regularly incorporated into modelling assumptions and scenarios.
It is important to incorporate these data because barriers to ART may intersect with transmission risk,
particularly among key populations, due to issues of stigma, discrimination, and criminalization.
\cite{Ortblad2019,Baral2019}
A recent review of PrEP modelling in SSA\cite{Case2019}
identified similar opportunities to better leverage programmatic data.
Context-specific key population cascade data to support these efforts are often lacking.\cite{Mountain2014}
However, sensitivity analyses of modelling assumptions related to cascade equity
would likely be well-justified based on existing data,\cite{Mountain2014,Green2020}
and considering that unmeasured key population cascades may be lowest.\cite{Roberts2020}
Modelled representations of the treatment cascade may also need expansion
to include more cascade steps and states related to treatment failure and discontinuation.
\par
Finally, based on ecological analysis of scenarios, we found evidence that
modelling assumptions about risk and intervention heterogeneity
may influence the projected proportion of future infections averted by ART.
% JK: this feels like repeated results, but I don't know how else to introduce this paragraph...
We did not find similar evidence for relative incidence reduction due to ART,
but studies reporting these two outcomes were largely distinct.
Among studies reporting both, the overall pattern was consistent.
These findings highlight the limitations of ecological analysis to estimate
the potential influence of modelling assumptions on projected ART prevention benefits,
and motivate additional model comparison studies to better answer this question.
Our ecological analysis also suggested that the anticipated ART prevention impacts from homogeneous models
may be achievable in the context of risk heterogeneity
if testing and treatment resources are prioritized to higher risk groups.
% JK: I remember reading a full study that did something like our eSwatini 10-10-10 analysis,
%     and I want to mention & cite it here, but for the life of me I can't find it...
\par
Our review has four main limitations.
First, the key populations we considered did not include
adolescent girls and young women, transgender people, or mobile populations,
even though such populations may experience similar risks of transmission and barriers to care
as other key populations\cite{Tanser2015,Dellar2015}
We also did not document representations of violence, coercion, or anal sex,
which may similarly coincide with transmission risks and barriers to care.
\cite{Silverman2011,Baggaley2013}
Future work should explore representations of
such groups and phenomena in transmission models.
Second, we did not document which (if any) model parameters were fitted
or to which calibration targets.
As shown previously\cite{Eaton2014a,Knight2020}, model fitting can produce
parameter values which compensate for differences in model structure,
and thereby underpin counterintuitive associations between model structure and modelling results.
Third, we did not compare modelled factors of heterogeneity to
context-specific epidemiological data,
which in some cases may justify model assumptions of homogeneity.
However, we did note when authors specifically justified such assumptions.
Finally, we did not estimate the effect size of
individual heterogeneity factors on the projected ART prevention impact.
Such an effect estimate could be biased by confounding factors in univariate analysis,
while exploratory work found challenges in multivariate analysis of our data,
due to the small number of scenarios and high data collinearity.
\par
In conclusion, representations of risk heterogeneity vary widely
among models used to project the prevention impacts of ART in SSA.
Such differences may partially explain the large variability in the projected impacts.
Many opportunities also remain to incorporate new and existing data on
the intersections of risk heterogeneity and intervention heterogeneity.
Step-wise model comparison studies are likely needed to
estimate and understand the relative influence
of various modelling assumptions on the prevention impacts of ART.
