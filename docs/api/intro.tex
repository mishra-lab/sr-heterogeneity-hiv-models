\section{Introduction}
\label{s:intro}
As of 2019, two thirds (25.7 million) of all people living with HIV globally
were in Sub-Saharan Africa (SSA), where
an estimated one million new HIV infections were acquired in 2019 \cite{AIDSinfo}.
HIV treatment to reduce onward transmission remains a key element of combination HIV prevention \cite{WHO2016ART}.
Following empirical evidence of partnership-level efficacy of ART
in preventing HIV transmission \cite{Lundgren2015,Danel2015,Cohen2016},
and model-based evidence of ``treatment as prevention'' \cite{Granich2009,Eaton2012,Cori2014},
several large-scale community-based trials of universal test-and-treat (UTT) were recently completed.
These trials found that over 2-to-4 years,
cumulative incidence under UTT did not significantly differ from
cumulative incidence under ART according to national guidelines \cite{Havlir2019,Hayes2019,Iwuji2018}.
Thus the population-level reductions in incidence anticipated from transmission modelling
were not observed in these trials \cite{Baral2019,Havlir2020}.
\par
One theme in the proposed explanations for limited population-level ART prevention effectiveness
was heterogeneity in intervention coverage and its intersection with
heterogeneity in transmission risks \cite{AbdoolKarim2019,Baral2019}.  %SM: so this review found that the answer to this question was a "YES" correct? if you agree, then lets make sure that is super clear. I revised the abstract to try to make it a bit more clear I hope. For example, how can we get cascade differences by key populations by not modeling key populations? i.e. in abstract and discussion make sure the "intersection" here is the take home point. (where turnover is part of risk heterogeneity).
While viral suppression improved under UTT in all three trials,
21--54\% of study participants remained unsuppressed \cite{Iwuji2018,Havlir2019,Hayes2019}.
Populations experiencing barriers to viral suppression under UTT
may be at highest risk for acquisition and onward transmission, including key populations such as
women and men engaged in sex work, and men who have sex with men \cite{Hakim2018,Nyato2019}.
Data suggest subgroups not formally referred to as key populations, such as youth and men who have sex with women including clients of sex workers, may also
experience barriers to engagement in ART care \cite{Green2020,Quinn2019}.
Indeed, data suggest UTT scale-up in practice has not reach subgroups equally. %SM: include relevant citations here
\par
Risk heterogeneity can be defined by various factors affecting acquisition and onward transmission risk,
and is a well-established determinant of the emergence and persistence of HIV epidemics, including 
what it might take to control local transmission \cite{Anderson1986,Boily1997}.
Systematic model comparison studies by \citet{Hontelez2013} and \citet{Rozhnova2016}
found that projected prevention impacts of ART scale-up were smaller when more heterogeneity was captured in the model.
Given the upstream and complementary role of transmission modelling
in estimating the prevention impacts of ART \cite{Eaton2012,Delva2012},
we sought to examine how heterogeneity in risk and ART uptake has been represented
in mathematical models used to assess the prevention impacts of ART scale-up in SSA. %SM: focus is on ART scale-up as the intervention yes? not ART specifically
We conducted a scoping review and ecological regression with the following objectives.
Among dynamical compartmental models of sexual HIV transmission
that have been used to simulate the prevention impacts of ART in Sub-Saharan Africa:
\begin{enumerate}
\item\label{rq:1}
      In which epidemic contexts (geographies, populations, epidemic phases)
      have these models been applied?
\item\label{rq:2}
      How was the model structured to represent key factors of risk heterogeneity?
\item\label{rq:3}
      What are the potential influences of representations of risk heterogeneity
      on the projected prevention benefits of ART for all?
\end{enumerate}
