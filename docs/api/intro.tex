\section{Introduction}
\label{s:intro}
As of 2019, two thirds (25.7 million) of all people living with HIV globally
were in Sub-Saharan Africa (SSA), where
an estimated one million new HIV infections were acquired in 2019 \cite{AIDSinfo}.
HIV treatment to reduce onward transmission remains a key element of combination HIV prevention \cite{WHO2016ART}.
Following empirical evidence of partnership-level efficacy of ART
in preventing HIV transmission \cite{Lundgren2015,Danel2015,Cohen2016},
and model-based evidence of ``treatment as prevention'' \cite{Granich2009,Eaton2012,Cori2014},
several large-scale community-based trials of universal test-and-treat (UTT) were recently completed.
These trials found that over 2-to-4 years,
cumulative incidence under UTT did not significantly differ from
cumulative incidence under ART according to national guidelines \cite{Havlir2019,Hayes2019,Iwuji2018}.
Thus the population-level reductions in incidence anticipated from transmission modelling
were not observed in these trials \cite{Baral2019,Havlir2020}.
\par
One theme in the proposed explanations for limited population-level ART prevention effectiveness
was heterogeneity in intervention coverage and its intersection with
heterogeneity in transmission risks \cite{AbdoolKarim2019,Baral2019}.
While viral suppression improved under UTT in all three trials,
21--54\% of study participants remained unsuppressed \cite{Iwuji2018,Havlir2019,Hayes2019}.
Populations experiencing barriers to viral suppression under UTT
may be at highest risk for acquisition and onward transmission, including key populations such as
women and men engaged in sex work, and men who have sex with men \cite{Hakim2018,Nyato2019}.
Data suggest other sub-populations, including youth and men who have sex with women, may also
experience barriers to engagement in ART care \cite{Green2020,Quinn2019}
that could undermine treatment as prevention.
While widespread UTT scale-up may fill some coverage gaps,
equitable access to ART for all populations remains an open challenge.
\par
Risk heterogeneity, defined by various factors affecting acquisition and onward transmission risk,
is a well-established determinant of epidemic persistence and controllability
with a basis in the modelling literature \cite{Anderson1986,Boily1997}.
Model comparison studies by \citet{Hontelez2013} and \citet{Rozhnova2016}
found that projected prevention impacts of ART scale-up were smaller with greater heterogeneity.
Given the upstream and complementary role of transmission modelling
in estimating the prevention impacts of ART \cite{Eaton2012,Delva2012},
we sought to examine how heterogeneity in risk and ART uptake has been represented
in mathematical models used to assess the prevention impacts of ART in SSA.
We conducted a scoping review and ecological regression with the following objectives.
Among dynamical compartmental models of HIV transmission
that have been used to simulate ART for prevention in Sub-Saharan Africa:
\begin{enumerate}
\item In which epidemic contexts (geographies, populations, epidemic phases)
      have these models been applied?
\item How was the model structured to represent key factors of risk heterogeneity?
\item What are the potential influences of representations of risk heterogeneity
      on the projected prevention benefits of ART?
\end{enumerate}
