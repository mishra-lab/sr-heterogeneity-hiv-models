The HIV epidemic in Sub-Saharan Africa in 2019 included nearly one million new infections,
and approximately two thirds (25.7 million) of all people living with HIV globally.\cite{AIDSinfo}
Combating the epidemic requires combination prevention, including HIV treatment,
as recommended by the World Health Organization.\cite{WHO2016ART}
Effective HIV treatment with antiretroviral therapy (ART) leads to
viral load suppression and has been shown to prevent HIV 
transmission between sex partners.\cite{Lundgren2015,Danel2015,Cohen2016} % + Cohen2011 ?
\par
Following empirical evidence of partnership-level efficacy of ART
in preventing HIV, \cite{Lundgren2015,Danel2015,Cohen2016}
and model-based evidence of ``treatment as prevention'',\cite{Granich2009,Eaton2012} % + Delva2012 ?
several large-scale community-based trials of universal test-and-treat (UTT)
have recently been completed.\cite{Iwuji2018,Havlir2019,Hayes2019} % + Tanser2013 ?
These trials found that over 2 to 4 years,
cumulative incidence under UTT did not significantly differ from
cumulative incidence under ART according to national guidelines.\cite{Havlir2019,Hayes2019,Iwuji2018}
Thus the population-level reductions in incidence anticipated from transmission modelling were not observed.
\par
One theme in the proposed explanations for limited population-level ART effectiveness
was heterogeneity in intervention coverage and transmission risks.\cite{AbdoolKarim2019,Baral2019}
While viral suppression improved under UTT in all three trials,
21--54\% of study participants remained unsuppressed.\cite{Iwuji2018,Havlir2019,Hayes2019}
Populations experiencing barriers to viral suppression under UTT
may be at highest risk for onward transmission, such as
individuals with acute infection, sex workers, and mobile populations.\cite{Tanser2015,Hakim2018,Nyato2019}
Moreover, non-residents of study communities were excluded from interventions
(including 17\% of enumerated household adults in one trial\cite{Havlir2019})
and all three trials noted substantial migration into/out of study communities.
\cite{Iwuji2018,Havlir2019,Hayes2019}
While widespread UTT scale-up may fill some of these coverage gaps,
equitable access to ART for marginalized and mobile populations remains an open challenge.
\cite{Tanser2015,deGruchy2020}
\par
Given the upstream and complementary role of transmission modelling
to project the impact of ART as prevention,\cite{Eaton2012,Delva2012}
and the critical role of risk heterogeneity in epidemic dynamics,
we sought to critically appraise assumptions and representations of risk heterogeneity
in models assessing ART as prevention in Sub-Saharan Africa via scoping review.
Our objectives were to answer the following research questions.
Among dynamical compartmental models of HIV transmission
that have been used to simulate ART for prevention in Sub-Saharan Africa:
\begin{enumerate}
\item In which epidemic contexts (geographies, populations, epidemic phases)
      have these models been applied?
\item How was the model structured to represent key factors of risk heterogeneity?
\item What are the potential influences of representations of risk heterogeneity
      on the projected prevention benefits of ART?
\end{enumerate}