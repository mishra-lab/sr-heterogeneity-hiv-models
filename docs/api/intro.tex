Sub-Saharan Africa continues to bear the largest burden of HIV.		%SM: terrific intro! comments and edits below. main suggestion is to remove mention of mobile pop (just like no mention of PWID)- and include all that in the discussion section instead. a reader should finish reading intro quickly and land on the objectives we planned and almost anticipate what the primary objectives will be. i pretended not to know the project, and thought it would be focused on mobile pop, FSW, and acute infection as the main hetergoeneity elements...:) 
As of 2019, two thirds (25.7 million) of all people living with HIV globally are in Sub-Saharan Africa, where 
an estimated one million new HIV infections were acquired in 2019.\cite{AIDSinfo}
Data suggest that key populations, such as individuals engaged in sex work and men who have sex with men experience 
disproportioante risks of HIV acquistion and of onward transmission in Sub-Saharan Africa. 
%cite (data papers from reviews led by Baral; and also some modeling papers - see our tPAF paper we just did for published citations)
HIV treatment to reduce onward transmission remains a key element of 
combination HIV prevention.\cite{WHO2016ART}
Effective HIV treatment with antiretroviral therapy (ART) leads to
viral load suppression and has been shown to prevent HIV 
transmission between sex partners.\cite{Lundgren2015,Danel2015,Cohen2016} % + Cohen2011 ?
\par
Following empirical evidence of partnership-level efficacy of ART
in preventing HIV, \cite{Lundgren2015,Danel2015,Cohen2016}
and model-based evidence of ``treatment as prevention'',\cite{Granich2009,Eaton2012} % + Delva2012 ?
several large-scale community-based trials of universal test-and-treat (UTT)
have recently been completed.\cite{Iwuji2018,Havlir2019,Hayes2019} % + Tanser2013 ?
These trials found that over 2 to 4 years,
cumulative incidence under UTT did not significantly differ from
cumulative incidence under ART according to national guidelines.\cite{Havlir2019,Hayes2019,Iwuji2018}
Thus the population-level reductions in incidence anticipated from transmission modelling were not observed in the trials. %SM: cite the transmission modeling statement 
\par
One theme in the proposed explanations for limited population-level ART effectiveness
was heterogeneity in intervention coverage and its intersection with heterogeneity in transmission risks.\cite{AbdoolKarim2019,Baral2019}
While viral suppression improved under UTT in all three trials,
21--54\% of study participants remained unsuppressed.\cite{Iwuji2018,Havlir2019,Hayes2019}
It has been suggested that populations experiencing barriers to viral suppression under UTT		%SM: try to say statements like "it has been suggested , etc. "... rather than definitive statements that sometimes can read as potentially editorial :)
may be at highest risk for onward transmission, such as 
individuals with acute infection, key populations, and mobile populations.\cite{Tanser2015,Hakim2018,Nyato2019}  %not sure I understand the part about how individuals with acute infection face barriers to UTT? seems like makes sense for sex worekrs and mobilie poualtions, but for acute infection - is that not just a function of when during infection testing is done? rather than among whom? unless talking about acute infection in results, I would suggest just focus the introduction on subsets of the population. in the discussion, can talk about acute infection...
Moreover, non-residents of study communities were excluded from interventions
(including 17\% of enumerated household adults in one trial\cite{Havlir2019})
and all three trials noted substantial migration into/out of study communities.		%SM: remove all statements about migration from the introduction since the paepr does not address this. instead, talk about migration in the discussion. otehrwise, ritght now, I am expecting a paper largely focused on mobilie popualtions from reading just the introduction alone...
\cite{Iwuji2018,Havlir2019,Hayes2019}
While widespread UTT scale-up may fill some of these coverage gaps,
equitable access to ART for marginalized and mobile populations remains an open challenge.	%SM: KP are never really mentioned but most of the methods/results is about KP? mobile pop are mentioned at least 4 times so far?
\cite{Tanser2015,deGruchy2020}
\par
Given the upstream and complementary role of transmission modelling
in estimating the impact of ART as prevention,\cite{Eaton2012,Delva2012},	%but have not said anything about critical role of risk heterogenetiy yet to support the 'given' phrase in this statement, so I think just introducing in the first paragraph, and then going directly into this statement here as edited flows without us pre-editorializing (seeming like we are criticizing?)
we sought to critically appraise and examine the type and scope of risk heterogeneity captured
in mathematical models used to assess ART as prevention in Sub-Saharan Africa.
We conducted a scoping review with the following objectives.
Among dynamical compartmental models of sexual HIV transmission
that have been used to simulate the prevention impacts of ART in Sub-Saharan Africa:
\begin{enumerate}
\item\label{rq:1}
      In which epidemic contexts (geographies, populations, epidemic phases)
      have these models been applied?
\item\label{rq:2}
      How was the model structured to represent key factors of risk heterogeneity?
\item\label{rq:3}
      What are the potential influences of representations of risk heterogeneity
      on the projected prevention benefits of ART for all?
\end{enumerate}