Viral load suppression by antiretroviral therapy (ART)
has undisputed individual-level health benefits.
\autocite{}
Partner-based trials have also clearly demonstrated
prevention of HIV transmission via ART (undetectable = untransmittable).
\cite{Lundgren2015,Danel2015,Cohen2016} % Cohen 2011 NEJM?
Such demonstrations have complemented earlier model-based evidence of
``treatment as prevention'',\cite{Granich2009,Eaton2012} motivating
several large-scale community-based trials of universal test-and-treat.%
\cite{Havlir2019,Hayes2019,Iwuji2018}
% JK: add Tanser2013 ? (TODO)
\par
Unfortunately, universal test-and-treat without combination prevention interventions
did not significantly reduce HIV incidence in these trials.
\cite{Havlir2019,Hayes2019,Iwuji2018}
Proposed explanations for non-significance include:
participation bias;
poor linkage to care, possibly associated with stigma and accessibility;
concurrent cascade improvements in the control arm, including
expanded testing and ART eligibility;
mobility and sexual mixing outside study communities;
and possible de-emphasis of primary prevention.%
\cite{Havlir2019,Hayes2019,Iwuji2018,Baral2019}
These explanations highlight the potential challenges of
implementing ART as prevention at scale, particularly if
the unique treatment and prevention needs of at-risk populations
are not explicitly incorporated into ART scale-up;
that is, if heterogeneities in risk and access to care
are not sufficiently considered.
\cite{Baral2019}
% JK: Does the summary of hypothesized challenges (from study discussions)
%     connect clearly enough to the idea of "heterogeneity"?
\par
Given the upstream and complementary role of transmission modelling
to project the impact of ART as prevention,\cite{Granich2009,Eaton2012}
we were motivated to critically appraise
assumptions and representations of risk heterogeneity
in earlier models assessing ART as prevention.
This appraisal might help explain the differences between
projected and observed ART prevention impacts.
For the appraisal, we considered ``factors of risk heterogeneity'',
meaning population stratifications and epidemiological phenomena
which may/not be included in transmission models.
We also defined the following 5 mechanisms
by which factors of risk heterogeneity might influence ART prevention impact.
\begin{itemize}
  \item \textbf{Biological Effects:}
  differential transmission risk within HIV disease course
  that coincide with differential ART coverage
  \cite{Pilcher2004}
  \item \textbf{Network Effects:}
  differential transmission risk within sub-populations
  that increase the challenge of epidemic control through core group dynamics
  \cite{Anderson1986,Boily1997,Watts2010}
  \item \textbf{Coverage Effects:}
  differential transmission risk within sub-populations
  who also experience barriers to achieving viral suppression via ART,
  such as key populations
  \cite{Mountain2014,Lancaster2016,Hakim2018}
  \item \textbf{Behavioural Effects:}
  differential transmission risk due to
  behavioural changes related to engagement in the ART cascade
  \cite{} % JK: TODO
  \item \textbf{Epidemic Effects:}
  changes in the relative prevalence of people in each ART cascade step
  within the sexually active population
  \cite{} % JK: TODO
\end{itemize}
We then compiled a list of key factors of risk heterogeneity,
and their associated mechanisms of influence on ART prevention impact
(Table~\ref{tab:heterogeneity}).
Based on this list, we reviewed the model-based evidence of ART prevention impact
in order to answer the following research questions.
Among dynamical compartmental models of sexual HIV transmission
that have been used to simulate the prevention impacts of ART in Sub-Saharan Africa:
\begin{enumerate}
\item\label{rq:1}
      In which epidemic contexts (geographies, populations, epidemic phases)
      have these models been applied?
\item\label{rq:2}
      How was the model structured to represent key factors of risk heterogeneity?
\item\label{rq:3}
      What are the potential influences of representations of risk heterogeneity
      on the projected prevention benefits of ART for all?
\end{enumerate}
\afterpage{%
  \newgeometry{margin=2cm}
  \begin{landscape}
    \captionof{table}{%
      Factors of heterogeneity in HIV transmission
      and their possible mechanisms of influence on the prevention impact of ART interventions}
    \footnotesize\centering
\begin{tabular}{llp{.35\linewidth}p{.4\linewidth}}
  \toprule
  \textbf{Factor}
& \textbf{MP\tn{a}}
& \textbf{Definition}
& \textbf{Possible mechanism(s) of influence on ART prevention impact}
\\
\midrule
  Acute Infection
& $\beta_i$
& Increased infectiousness immediately following infection \cite{Hollingsworth2008,Boily2009}
& \textbf{Biological}: transmissions during acute infection are unlikely to be prevented by ART
\\
  Late-Stage Infection
& $\beta_i$
& Increased infectiousness during late-stage infection \cite{Hollingsworth2008,Boily2009}
& \textbf{Biological}: transmissions during late-stage are more likely to be prevented by ART
\\
  Drug Resistance
& $\beta_i$
& Transmitted factor that requires regimen switch to achieve viral suppression \cite{DeWaal2018}
& \textbf{Biological}: transmissions during longer delay to achieving viral suppression will not be prevented by ART
\\
\midrule
  HIV Morbidity
& $c$; $\eta$
& Reduced sexual activity during late-stage disease \cite{Myer2010,McGrath2013}
& \textbf{Behaviour Change}: reduced morbidity via ART could increase HIV prevalence among the sexually active population
\\
  HIV Counselling
& $c$; $\eta$; $\kappa$
& Reduced sexual activity and/or increased condom use after HIV diagnosis \cite{Tiwari2020}
& \textbf{Behaviour Change}: increased HIV testing with ART scale up can contribute to prevention even before viral suppression is achieved
\\
\midrule
  Activity Groups
& $c$; $\kappa$
& Any stratification by rate of partnership formation \cite{Anderson1991}
& \textbf{Network}: higher transmission risk among higher activity
\\
  Age Groups
& $c$; $\kappa$
& Any stratification by age
& \textbf{Network \& Cascade}: higher transmission risk and barriers to viral suppression among youth \cite{Birdthistle2019,Green2020}
\\
  Key Populations
& $c$; $\kappa$
& Any epidemiologically defined higher risk groups \cite{WHO2016KP}
& \textbf{Network \& Cascade} higher transmission risk and barriers to viral suppression among key populations \cite{Hakim2018}
\\
  Group Turnover
& $\phi$
& Individuals move between activity groups and/or key populations reflecting sexual lifecourse \cite{Watts2010}
& \textbf{Network \& Cascade}: counteract effect of stratification due to shorter periods in higher risk \cite{Knight2020};
  viral suppression may be achieved only after periods of higher risk
\\
  Assortative Mixing
& $m$
& Any degree of assortative mixing (like-with-like) by age, activity, and/or key populations
& \textbf{Network}: assortative sexual networks compound effect of stratification \cite{Anderson1991}
\\
  Partnership Types
& $\eta$; $\kappa$
& Different partnership types are simulated, with different numbers of sex acts and/or condom usage \cite{Scorgie2012}
& \textbf{Network}: longer duration and lower condom use among main versus casual/sex work partnerships
  counteracts effect of stratification
\\
  ART Cascade Gaps
& $\tau$; $\alpha$
& Slower ART cascade transitions among higher activity groups or key populations \cite{Hakim2018,Green2020}
& \textbf{Cascade}: ART prevention benefits may be allocated differentially among risk groups
\\
\bottomrule
\end{tabular}
\floatfoot{\tnt[a]{MP: Model Parameters ---
  $\beta_i, \beta_s$: transmission probability per act (infectiousness, susceptibility);
$\eta$:      number of sex acts of each type per partnership;
$\kappa$:    proportion of sex acts unprotected by a condom;
$c$:         partnership formation rate;
$m$:         mixing matrix (probability of partnership formation);
$\mu$:       mortality rate;
$\nu$:       entry rate;
$\phi$:      internal turnover between activity groups;
$\tau$:      testing rate;
$\alpha$:    ART initiation rate (and retention-related factors).}
}

    \label{tab:heterogeneity}
  \end{landscape}
  \restoregeometry
  \clearpage}
% JK: [TODO] perhaps we need a list of other factors of heterogeneity
%     that we chose *not* to include because
%     we did not hypothesize any influence on ART prevention impact