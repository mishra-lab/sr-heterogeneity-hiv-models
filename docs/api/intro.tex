Sub-Saharan Africa (SSA) continues to bear the largest burden of HIV.
As of 2019, two thirds (25.7 million) of all people living with HIV globally are in SSA, where
an estimated one million new HIV infections were acquired in 2019 \cite{AIDSinfo}.
Data suggest that key populations, such as individuals engaged in sex work and men who have sex with men experience
disproportionate risks of HIV acquisition and onward transmission in SSA
\cite{Baral2012,Beyrer2012,Mishra2012,Boily2015}.
HIV treatment to reduce onward transmission remains a key element of combination HIV prevention \cite{WHO2016ART}.
Effective HIV treatment with antiretroviral therapy (ART) leads to viral load suppression
and has been shown to prevent HIV transmission between sex partners \cite{Lundgren2015,Danel2015,Cohen2016}.
\par
Following empirical evidence of partnership-level efficacy of ART
in preventing HIV \cite{Lundgren2015,Danel2015,Cohen2016},
and model-based evidence of ``treatment as prevention'' \cite{Granich2009,Eaton2012,Cori2014},
several large-scale community-based trials of universal test-and-treat (UTT)
have recently been completed \cite{Iwuji2018,Havlir2019,Hayes2019}.
These trials found that over 2 to 4 years,
cumulative incidence under UTT did not significantly differ from
cumulative incidence under ART according to national guidelines \cite{Havlir2019,Hayes2019,Iwuji2018}.
Thus the population-level reductions in incidence anticipated from transmission modelling
were not observed in the trials \cite{Baral2019,Havlir2020}.
\par
One theme in the proposed explanations for limited population-level ART effectiveness
was heterogeneity in intervention coverage and its intersection with
heterogeneity in transmission risks \cite{AbdoolKarim2019,Baral2019}.
While viral suppression improved under UTT in all three trials,
21--54\% of study participants remained unsuppressed \cite{Iwuji2018,Havlir2019,Hayes2019}.
It has been suggested that populations experiencing barriers to viral suppression under UTT
may be at highest risk for onward transmission, including key populations like
female sex workers, men who have sex with men, and adolescent girls and young women
\cite{Hakim2018,Nyato2019,Green2020}.
While widespread UTT scale-up may fill some of these coverage gaps,
equitable access to ART for marginalized populations remains an open challenge.
\par
Given the upstream and complementary role of transmission modelling
in estimating the impact of ART as prevention \cite{Eaton2012,Delva2012},
we sought to critically appraise and examine the type and scope of risk heterogeneity captured
in mathematical models used to assess the prevention impacts of ART in SSA.
We conducted a scoping review with the following objectives.
Among dynamical compartmental models of sexual HIV transmission
that have been used to simulate the prevention impacts of ART in Sub-Saharan Africa:
\begin{enumerate}
\item\label{rq:1}
      In which epidemic contexts (geographies, populations, epidemic phases)
      have these models been applied?
\item\label{rq:2}
      How was the model structured to represent key factors of risk heterogeneity?
\item\label{rq:3}
      What are the potential influences of representations of risk heterogeneity
      on the projected prevention benefits of ART for all?
\end{enumerate}