\section{Reviewer \#2}
\begin{comment}[*]
Knight et al. present a systematic scoping review addressing the impact of risk heterogeneity representation on the estimated impact of antiretroviral therapy to reduce HIV transmission in sub-Saharan Africa. They find a range of risk heterogeneity model paramaterisation, in turn leading to substantial variation in the estimated reduction of HIV incidence/new infections attributable to ART. This represents a significant consolidation exercise and represents a timely and valuable addition to the literature as interest in risk heterogeneity in sub-Saharan Africa, including representation of key populations, increases.
\end{comment}
\begin{response}
We thank the reviewer for their comment.
\end{response}
\subsection{Major Comments}
\begin{comment}\label{recommend}
It's not immediately clear to me what the key recommendation is that flows from Figure 3. If I have a model with the base case (no risk heterogeneity), which compartments or dynamics should I add first better to represent the true epidemic?
\end{comment}
\begin{response}
As described above in \ref{reg.2} in response to Reviewer 1 suggestions, the results section 3.3 and discussion have been
substantially revised in light of more precise findings.
The revised discussion now concludes with the following recommendations which we hope are more clear:
\quote{In conclusion, model-based evidence of ART prevention impacts could likely be improved by:
1) consistenly including risk group turnover,
   to reflect challenges to HIV prevention associated with the dynamic nature of sexual risk; %SM: reads bit vague. what do we mean by prevention challenges because of dynamic nature? 
2) integrating data on differences in ART cascade between sexual risk groups, %SM: this was clear
   to reflect services as delivered on the ground; and
3) capturing heterogenetiy in risks in the context of key populations, 
   to reflect intersections of transmission risk and barriers to HIV services 
   that may undermine the prevention benefits of ART.} %SM: avoid using the phrase "treatment as prevention" (has fallen out of favor and is not super precise)
\end{response}
\begin{comment}\label{risk.explain}
Further discussion of the headline finding would be welcome - that the omission of key populations but the inclusion of risk heterogeneity in the generalised population brings about the smallest declines in new HIV infections is notable. Where possible - interrogating which dynamics are most important in the discrepancy between 'Activity (no KP)' and the other three model scenarios would be of interest.
\end{comment}
\begin{response}
In the new regression analysis, there is little difference in estimated effect between ``Activity (no KP)'' and ``KP (same)'',
except for IR for ``KP (same)'', which had wide confidence interval based on only 5 scenario time horizons. This suggests that when the ART coverage is
assumed to  be the same between risk-groups, then the inclusion/exclusion of KP may not have as much influence on the outcome (prevention impact of ART). However, we note this interpretation with caution given the small number of scenarios available for analyses.
\end{response}
\begin{comment}
Please elaborate on and support Figure C.11 - it's not immediately clear to me that 'the pattern of incidence reduction versus modelled heterogeneity was similar to the pattern of infections averted versus modelled heterogeneity". Recognising that these data do not stem from the same studies, it is noted that in Table C.1 the incidence reduction increases ~2fold between no risk heterogeneity and activity (no KP), whilst averted infections decreases \textasciitilde4 fold. This would appear to be a key difference?
\end{comment}
\begin{response}
As noted above (\ref{risk.explain}), the adjusted effect estimates for
the ``Risk Stratification \& Cascade Differences'' variable are now more consistent across both outcomes
(incidence reduction and cumulative infections averted).
We suggest that differences in unadjusted estimates were likely due to confounding by unaccounted factors, %SM: is this noted in discussion section? 
and random chance due to the small number of scenarios time horizons for some factor levels
(e.g. 5 and 1 for ``KP (same)'' and ``KP (priority)'' incidence reduction data).
\end{response}
\begin{comment}
As HIV prevalence is linked to epidemic type, it's interesting that ART prevention impacts were larger with lower HIV prevalence. As the lower prevalence epidemics in West Africa are driven by KPs/more so than the epidemics in ESA, I assume that modelling studies in West Africa are more likely to be KP-disaggregated. However, you have shown that KP-disaggregated models estimate smaller ART prevention impacts. Could this be explored further?
\end{comment}
\begin{response}
As hypothesized by the reviewer,
the regression analysis reverses the estimated influence of HIV prevalence on ART prevention impacts:
from larger impact with lower prevalence (43 vs 22 \%IR, 26 vs 18 \%CIA),
to smaller impact with lower prevalence (-9 \%IR, -9 \%CIA).
\end{response}
\begin{comment}
Whilst recognising this is a scoping review, some more discussion on the impact of these findings for the global HIV response would be welcome. As noted in the discussion, modelled estimates "did not always reflect the available data". The key population estimates (and subsequent estimates of averted HIV transmission) are reliant on weak data and some informed comment on how these results should be used would be good.
\end{comment}
\begin{response}
The key messages of the review and contextualization in the global HIV response
are now hopefully more clear in the revised discussion (see \ref{recommend}).
We additionally note possible priority areas for data collection such as:
\quote{The findings suggest that turnover is important, and as such, data to parameterize turnover could benefit from surveys, cohorts, and repeated population size estimates that
can provide data on individual-level trajectories of sexual risk, such as duration in sex work} \dots %SM: try to give some concrete examples
\quote{Improved modelling and prioritization of sevices designed to reach key populations
will rely on continued investment in community-led data collection for hard-to-reach populations} \dots
\quote{With the growth in data collected by communities about communities most affected by HIV, %SM: i woudl disagree that there is a lack of data given the growth of data in recent years, suggest rephrasing. 
there exists a data-driven opportunity to more consistenly capture heterogenetity in risks and intervention access among key populations in transmission models.}
\end{response}
\subsection{Minor Comments}
\begin{comment}
Figure 2: I would prefer HIV prevalence over PLHIV as the chloropleth.
\end{comment}
\begin{response}
We have revised the figure to use HIV prevalence as the country colour.
\end{response}
\begin{comment}
Do any studies address the age distribution of key populations? This would be useful to include.
\end{comment}
\begin{response}
TBD
\end{response}
\begin{comment}
Were transgender people or prisoners included in any studies?
\end{comment}
\begin{response}
We added transgender people and prisoners to the key populations considered in the review,
with similar definition to MSM and PWID: %SM: huh? how can they have a similar definition? rephrase for clarity 
\quote{Any activity group(s) described by the authors as (transgender, prisoners)}. %SM: Dude!! this reads as transgender persons are prisoners? Please clean up / clarify!
Sections 2.3.1 (Methods), 3.1.1 (Results, including Table 2), and B.2.1 (Appendix: Definitions)
in the manuscript were each updated accordingly.
\end{response}
\begin{comment}
I find the bubble plots difficult to interpret. The bubbles are often similarly sized, I'm not sure the extra information adds to the results and crowds the plot. Perhaps grouped bar charts or grouped box plots would be easier to read - particularly for Figure 3.
\end{comment}
\begin{response}
We have revised to use boxplots,
stratified by 10-year intervals of time since intervention,
and categorizing previously continuous variables
(HIV prevalence, ART coverage target, and relative infectiousness on ART).
One exception is that, due to the small number of data points,
we've left Figure~C.11 as a point plot.
\end{response}
\begin{comment}
Consider inserting Table C.1 into the main text
\end{comment}
\begin{response}
We have moved the table into the main text (see~\ref{tab:c1}).
\end{response}
\begin{comment}
Consider a sensitivity analysis of the main findings without South African data
\end{comment}
\begin{response}
We explored the feasibility of this analysis.
However, only 19 (IR) and 58 (CIA) data points were available without the South African data,
leading to rank defficiency in the model,
and thus the parameters could not be estimated
without changing which parameters were included in the model.
\end{response}