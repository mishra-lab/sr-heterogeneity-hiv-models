\section{Reviewer \#1}
\begin{comment}[*]
This paper by Knight and colleagues reviews mathematical models of the impact of antiretroviral treatment (ART) on HIV incidence. A particular focus is on differences between models in terms of how they represent heterogeneity in HIV risk, and how this heterogeneity influences the predicted impact of ART on HIV incidence. The review is thorough and clearly presented.
\end{comment}
\begin{response}
We thank the reviewer for their comment.
\end{response}
\subsection{Major comments}
\begin{comment}\label{reg.1}
A limitation of this review, which the authors acknowledge, is that it is essentially an ecological analysis. Although there is an association between models allowing for heterogeneity in risk behaviour and models predicting smaller ART impacts on HIV incidence, this finding doesn't necessarily prove that heterogeneity in risk behaviour determines the extent of the ART impact. There are many possible confounders and other variables that influence the extent of the modelled reductions in HIV incidence, as the authors note. Would it not make more sense to use a meta-regression approach to isolate the effect of the heterogeneity assumptions, controlling for the confounding factors? Although this wouldn't completely get around the causality conundrum, it would be better than the current approach, which is effectively relying on univariable rather than multivariable analysis.
\end{comment}
\begin{response}
We agree with that the paper could be strengthened with a regression analysis,
and have revised the paper to include one.
We describe our analysis as a ``multivariate regression'', and not a ``meta-regression'',
as we did not pool uncertainty estimates.
Confidence intervals were only given for
6\% incidence reduction (IR) data points, and
33\% cumulative infections averted (CIA) data points,
each data point representing the outcome (IR or CIA) reported for a unique study, scenario, and time-horizon.
So, we did not attempt to integrate these confidence intervals in the regression as in a typical meta-regression.
\par
The variables included in the regression are given in Table 3 (formerly C.1),
including the following possible confounders of heterogeneity factors:
time horizon for outcome evaluation,
HIV prevalence at time of ART scale-up,
relative rate of HIV transmission of ART, and
CD4 threshold for ART initiation in the scale-up scenario.
Due to missingness (\% missing data points for IR \& CIA, respectively), we omitted the variables:
HIV incidence trend (19\% \& 53\% missing), and ART coverage target (77\% \& 23\% missing).
More details about the regression are given in \ref{reg.2}.
\end{response}
\begin{comment}\label{reg.2}
A related concern is that the tests for statistically significant differences in the univariable analysis appear problematic. When you use the Kruskal-Wallis test you are assuming independence of observations. But here the observations are not statistically independent of one another - in many cases multiple observations are being taken from the same study (and there is likely to be a high degree of within-study correlation). By not taking into account the within-study correlation I think you exaggerate the significance of the differences between model types. That might explain some of the odd results in Table C1 (for example a statistically significant positive relationship between HCT behaviour change and the incidence reduction, but a significant negative relationship between HCT behaviour change and the cumulative \% of infections averted). If you use a meta-regression approach, you should be able to control for the within-study correlation.
\end{comment}
\begin{response}
To account for clustering of the data points by study, we used generalized estimating equations to fit the regression model
(specifically, the \texttt{geeglm} model from the \texttt{geepack} packge in R).
This semi-parametric model allows estimation of population-averaged effects of each factor on the outcome,
while controlling for within-study correlation of outcomes as a nuisance.
We assume independence (\texttt{corstr='independence'}) of outcomes within each cluster (study) because it is more robust for small / imbalanced data such as ours;
this assumption would only affect the standard errors, not the point estimates of effects.
\par
The regression model is described in Section 2.3.3 (Methods),
while the adjusted effect estimates are added to Table 3 (formerly C.1),
and key findings are highlighted in Section 3.3 (Results).
Figures 2 and C.20 also gives a forest plots of effect estimates.
\par
The headline findings of the review have also been revised in light of the regression results:
We found that heterogeneity alone (defined by activity groups with or without key populations)
was not a strong determinant of predicted ART prevention impacts,
but rather risk group turnover and potential cascade differences between groups.
The results in section 3.3 and the discussion have been revised substantially in light of these findings.
\end{response}
\begin{comment}
Although the authors explain why modelling key populations might influence the predicted impact of ART on HIV incidence, I think the introduction would be strengthened if they also mentioned the literature that covers heterogeneity more broadly - and the role that heterogeneity plays in determining intervention impact. For example, Nagelkerke et al (2007, BMC Infectious Diseases, 7:16) showed that the modelled impact of male circumcision on HIV incidence was much greater when assuming no heterogeneity in risk behaviour, Johnson et al (2012, Journal of the Royal Society Interface, 9:1544-54) showed that the modelled impact of condoms and ART was strongly correlated with the heterogeneity in risk behaviour, and Hontelez et al (2013, PLoS Medicine, 10:e1001534) showed that allowing for heterogeneity reduced the predicted impact of ART on HIV incidence. Key populations are one component of the heterogeneity-impact relationship, but the introduction currently reads as if they are the only determinant of the relationship.
\end{comment}
\begin{response}
In addition to a broader description of populations with differential ART cascade (see \ref{hm.gaps}),
the following has been added to the beginning of the last paragraph in the introduction:
\quote{Risk heterogeneity, defined by various factors affecting acquisition and onward transmission risk,
is a well-established determinant of epidemic persistence and controllability
with a basis in the modelling literature [Anderson1986, Boily1997].
Model comparison studies by Hontelez et al (2013) and Rozhnova et al (2016)
found that projected prevention impacts of ART scale-up were smaller with greater heterogeneity.}
\end{response}
\begin{comment}\label{tab:c1}
Given that the results from Table C1 are so central to the overall conclusion of the paper (and are referred to in the abstract), it seems strange to put this table in the supplementary materials. I think it would be more appropriate to include this table in the main text of the article.
\end{comment}
\begin{response}
We have updated the table to include the results of the regression analysis and moved the table into the main text.
We moved the map (formerly Figure 2) to the appendix due to the limit of 5 figures and tables in the main text.
\end{response}
\begin{comment}\label{hm.gaps}
The third paragraph of the Introduction mentions possible inequalities in uptake of HIV testing and ART as an explanation for the lower-than-expected impact of UTT, and mentions a number of sub-populations that might be disadvantaged. But the authors fail to mention heterosexual men. There is much evidence of heterosexual men being at a disadvantage (in terms of HIV testing and ART uptake) and they also contribute more to transmission than heterosexual women, so why are they not mentioned here? Similarly in the second and third paragraphs of the discussion the authors criticize modelling studies that don't consider key population dynamics, but they don't mention the challenges around engaging heterosexual men (and the problem that many models don't consider differences between men and women in ART coverage). The poor uptake of HIV testing and ART in heterosexual men is really the Achilles heel in the 'treatment as prevention' strategy in Africa, yet this issue is frequently ignored in the literature. I feel the authors could have drawn more attention to this issue throughout the paper, rather than focusing narrowly on the traditionally defined key populations.
\end{comment}
\begin{response}
We have updated the introduction to mention heterosexual men:
\quote{Populations experiencing barriers to viral suppression under UTT
may be at highest risk for acquisition and onward transmission, including key populations like
sex workers and men who have sex with men [Hakim2018, Nyato2019].
Other sub-populations, including youth and heterosexual men, also
experience barriers to engagement in ART care [Green2020, Quinn2019]
that could undermine treatment as prevention.}
Additionally, the ``sex'' stratification variable for objective 3
has been updated to distinguish between models that do and do not consider
lower cascade among men,
with findings discussed in section 3.3.
\end{response}
\begin{comment}
It was not clear why the authors excluded individual-based approaches from this review. Although people tend to use the terms 'compartmental models' and 'deterministic models' interchangeably, some would argue the terms mean slightly different things (see Garnett, STIs, 2002, 78:7-12). The point is that it's possible for an individual-based model to be 'compartmental' in the sense that it works with categorical variables rather than continuously-defined variables. In such cases one could argue for including an individual-based model in the review, since its compartments/categories can be classified in the same way as a deterministic model. But even when key variables are defined on a continuous scale it's not clear why you would want to exclude the individual-based model from the review.
\end{comment}
\begin{response}
We agree with the reviewer that individual-based models represent
another important source of model-based evidence for quantifying ART prevention impacts,
for which it is worth examining the various representations of heterogeneity,
and their potential association with projected impacts.
However, as the reviewer suggests, we feel that it would be difficult to define
criteria to include some but not all individual-based models,
for example, on the basis that they include some compartmental-like stratification(s).
Then, or if including all individual-based models, it would become unweildly
to simultaneously compare both continuous and discrete representations of heterogeneity,
such as population stratifications in compartmental models versus
continuous sampling distributions for parameter values in individual-based models.
The additional complexity of modelling concurrent partnerships as well would be
challenging to synthesize well within the available space.
We would look forward to conducting or supporting a complementary review
focused specifically on individual-based models.
\end{response}
\subsection{Minor comments}
\begin{comment}
Page 6: I didn't understand this sentence: "Studies in Dataset B specifically examined scale-up of ART coverage alone (versus combination intervention) for the whole population (versus ART prioritized to subgroups)". The authors seem to be referring to multiple different comparisons in the same sentence. Perhaps it would help to rewrite as two sentences, focusing on the primary comparison of interest in the first sentence.
\end{comment}
\begin{response}
We have clarified the sentence as follows:
\quote{Studies in Dataset B met three additional criteria:
1) examined scale-up of ART coverage alone (versus combination intervention);
2) examined ART intervention for the whole population (versus ART prioritized to subgroups); and
3) reported HIV incidence reduction and/or cumulative HIV infections averted
relative to a base-case scenario reflecting status quo.}
\end{response}
\begin{comment}
Page 17: "Our ecological analysis also suggested that the anticipated ART prevention impacts from homogeneous models may be achievable in the context of risk heterogeneity if testing/treatment resources are prioritized to higher risk groups." I didn't follow this - what is this based on? The comparison of the 21\% vs 10\% on p. 15? If so, aren't the numbers too small to suggest a statistically significant difference? (See my earlier comment on testing for significant differences.)
\end{comment}
\begin{response}
In the revised results and discussion following the regression analysis results,
this sentence has been removed.
The role of cascade differences as a possible determinant of ART prevention impacts is
explored in the third paragraph of the discussion.
\end{response}
\begin{comment}
I felt there could have been more explanation for some of the unexpected findings in Table C1. As noted earlier, some of these odd findings might just be due to inappropriate statistical tests.
\end{comment}
\begin{response}
In the regression analysis, adjusted factor effects appear to be more consistent with expected findings.
Additionally, since the main factors of interest were those related to risk heterogeneity,
we did not seek to explore or explain all covariate effects from the model,
which could lead to ``table 2 fallacy''.
Finally, we added the following to the limitations paragraph:
\quote{Third, we did not extract data on model fitting,
which could explain some counterintuitive effect estimates.
For example, modelling increased infectiousness in late-stage HIV reduced ART prevention impacts.
However, in most studies, newly ART-eligible patients via scale-up had earlier stage HIV;
therefore, such patients would have lower modelled infectiousness than late-stage HIV,
and lower infectiousness than in a model with uniform infectiousness fitted to the same data.
A similar mechanism could explain increased ART prevention impacts when including acute infection.}
\end{response}
\begin{comment}
The references in the supplementary materials are incorrectly formatted. Often they give the second initial of the author but not the first initial. For example, L. Korenromp (reference 62) should be E.L. Korenromp, and J. Abu-Raddad (reference 56) should be L.J. Abu-Raddad.
\end{comment}
\begin{response}
We apologize for these errors.
The references have been thoroughly checked and should now be correct.
\end{response}