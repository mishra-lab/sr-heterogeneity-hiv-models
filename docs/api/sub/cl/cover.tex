\documentclass{article}
\usepackage[margin=3cm]{geometry}
\usepackage[sc,osf]{mathpazo}
\usepackage{ETbb}
\usepackage{graphicx}
\graphicspath{{logo/}}
\setlength{\parindent}{0pt}
\setlength{\parskip}{2ex}
\pagenumbering{gobble}
%%%%%%%%%%%%%%%%%%%%%%%%%%%%%%%%%%%%%%%%%%%%%%%%%%%%%%%%%%%%%%%%%%%%%%%%%%%%%%%%%%%%%%%%%%%%%%%%%%%%
\begin{document}
\begin{minipage}{0.5\linewidth}
  \includegraphics[height=0.8cm]{map-cuhs.eps}\\[1ex]
  \includegraphics[height=0.8cm]{smh.eps}\\[1ex]
  \includegraphics[height=1.2cm]{uoft.eps}
\end{minipage}%
\begin{minipage}{0.5\linewidth}
  \begin{flushright}\small
    Dr. Sharmistha Mishra\\
    MAP Centre for Urban Health Solutions\\
    Li Ka Shing Knowledge Institute\\
    St Michael's Hospital,
    Unity Health Toronto\\
    University of Toronto\\
    sharmistha.mishra@utoronto.ca
  \end{flushright}
\end{minipage}
\vskip\parskip
Dr. Katia Koelle, Professor Steven Riley, Dr. Cecile Viboud\\
Editors-in-Chief of \textit{Epidemics}
\par
\textbf{Re. Submission of a Scoping Review Paper}
\par
Dear Editors,
\par
We are pleased to submit the attached manuscript entitled
\textit{Risk heterogeneity in compartmental HIV transmission models
  applied to assess ART as prevention in Sub-Saharan Africa: A scoping review}
for consideration to publish in \textit{Epidemics}.
\par
Compartmental transmission models have played an important role in projecting
the population-level prevention benefits of ART scale-up.
However, several recent large-scale community-based trials of universal test and treat
have not demonstrated the incidence reductions anticipated from modelling studies.
One hypothesis for why model and trial results have differed suggests that
representations of risk heterogeneity in transmission models
have not sufficiently reflected epidemic contexts.
\par
To explore the evidence for this hypothesis, in this paper we conduct a scoping review on
representations of risk heterogeneity in compartmental transmission models
used to project the prevention benefits of ART scale-up within Sub-Saharan Africa.
First, we develop a conceptual framework to organize key factors of modelled risk heterogeneity,
such as if/how model populations, rates, and probabilities
are stratified along dimensions like age, sex, and sexual activity.
Then, using a set of 94 systematically identified modelling studies,
we summarize how these factors have been modelled, alongside information regarding
the epidemic context and the projected impact of ART scale-up on new infections.
\par
We find that modelled representations of risk heterogeneity vary widely,
with approximately two thirds and two fifths of studies considering
heterogeneous sexual activity and at least one key population, respectively.
Nearly all studies assumed equal rates of
diagnosis, ART initiation, and retention across risk groups.
Barriers to treatment initiation and retention that may be experienced by key populations
were only considered in 5 studies.
We also find preliminary evidence at the ecological level that
homogeneous models project a greater proportion of infections averted by ART,
but that models with key populations project similar ART prevention impacts
when those key populations are prioritized for ART cascade engagement.
\par
Our review summarizes model structures and assumptions underpinning
the current body of compartmental modelling literature on projections of ART prevention impacts.
We highlight the potential influence of different representations of risk heterogeneity
on the projected ART prevention impacts,
and identify key areas for future model developments.
To our knowledge, ours is the first review to focus specifically on
models applied to study ART scale-up,
and with detailed examination of model structure.
As such, we hope you find our review to be of timely interest
to the readership of \textit{Epidemics}.
\par
Thank you for your consideration and we look forward to hearing from you.
\vskip\parskip
Sincerely,\par
Jesse Knight and Sharmistha Mishra
\end{document}
%%%%%%%%%%%%%%%%%%%%%%%%%%%%%%%%%%%%%%%%%%%%%%%%%%%%%%%%%%%%%%%%%%%%%%%%%%%%%%%%%%%%%%%%%%%%%%%%%%%%
