% --------------------------------------------------------------------------------------------------
\subsubsection*{Different Populations need Different \hiv Interventions}
Advances in \hiv antiretroviral treatment (\art)
have produced highly effective drug regimens,
whereby circulating levels of \hiv virus
in adherent patients are reduced to undetectable levels%
~\cite{TBD}.
Viral suppression by \art has clear individual-level benefits
for health and quality of life~\cite{TBD}.
Moreover, recent trials have suggested that
virally suppressed individuals cannot transmit \hiv,
a finding described as:
``undetectable = untransmittable'' (U=U)~\cite{Eisinger2019}.
Inspired by U=U, researchers and policymakers have called for
rapid scale-up of \art coverage
as the main intervention by which to reduce \hiv incidence (``treatment as prevention'')
in the widespread epidemics of Sub-Saharan Africa~\cite{TBD}.
Global ambition to scale up \art coverage is further motivated by
the \textsc{unaids} \mbox{90-90-90} targets, defined as:
90\% of people living with \hiv are diagnosed;
90\% of those diagnosed are on \art; and
90\% of those on \art are virally suppressed.
\par
Unfortunately, several large-scale trials aiming to demonstrate
population-level impact of treatment as prevention in Sub-Saharan Africa
have failed to show a significant reduction in new infections%
~\cite{Iwuji2018,Havlir2018,Hayes2019}.
As suggested by \textcite{Baral2019} and others, these unexpected results
might be attributable to implementation challenges at scale.
Such challenges can emerge at several steps along the treatment cascade, including:
testing for \hiv,
linkage to care after a positive test,
starting \art after linkage to care,
achieving viral suppression after starting \art~\cite{TBD}.
Moreover, individuals who are most likely to experience challenges in \hiv care
are often at the highest risk of \hiv acquisition and onward transmission~\cite{TBD}.
Groups of vulnerable individuals in the epidemic
are sometimes described as ``key populations''.
\par
Several key populations have been identified, including:
adolescent girls and young women;
sex workers;
men who have sex with men;
transgender people;
prisoners;
and people who inject drugs~\cite{WHO2014HIVKP}.
Key populations often experience
several risk factors for \hiv transmission and barriers to care, such as:
violence and coercion into unsafe sex;
criminalization of lifestyle;
stigma related to lifestyle or \hiv status;
housing and financial instability;
and substance abuse~\cite{Mountain2014, WHO2014HIVKP}.
To meet the unique \hiv prevention and treatment needs of key populations,
specific interventions are needed which
address known vulnerabilities~\cite{WHO2014HIVKP}.
For example, risk of \hiv transmission can be reduced through
needle exchange programs~\cite{Kurth2011},
increased access to condoms~\cite{TBD},
and financial support to reduce transactional sex~\cite{Pettifor2012}.
Similarly, outreach and support by community peers
can increase engagement of key populations in \hiv care~\cite{WHO2014HIVKP}.
\par
Despite considerable evidence supporting the need for diversified \hiv interventions,
recent large-scale studies of treatment as prevention
have not considered the unique needs of key populations%
~\cite{Iwuji2018,Havlir2018,Hayes2019}.
Failure to deliver appropriate interventions to key populations
has left these groups far behind global progress toward
the \mbox{90-90-90} targets~\cite{Gupta2017},
threatening to undermine the expected benefits of treatment as prevention.
% --------------------------------------------------------------------------------------------------
\subsubsection*{Mathematical Modelling of \hiv Transmission}
Population-level models of \hiv transmission have long been used to
project \hiv epidemic trajectories
(e.g.\ incidence over time)
and predict intervention impacts
(e.g.\ reduction in incidence after X years)%
~\cite{Eaton2012}.
In popular compartmental models,
overall populations are stratified by disease state and risk group,
while differential equations are used to govern movement of individuals between compartments.
% TODO: a figure could be helpful here.
Many different compartmental model structures have been used,
from a 3-compartment model, representing
3 disease states in a homogeneous population~\cite{Moghadas2003},
to a 294-compartment model, representing
21 disease states and 14 risk groups~\cite{Kerr2015}.
\par
Unfortunately, differences in model structure and assumptions
have been shown to substantially influence projections of
epidemic trajectory and intervention impact~\cite{Eaton2012,Hontelez2013}.
Most importantly, failure to model heterogeneity in risk
results in lower basic reproduction number $R_0$~\cite{Anderson1986},
which could lead to overestimated ease of epidemic control
through universal treatment as prevention~\cite{Hontelez2013}.
And yet, several mathematical models that were used to support treatment as prevention
did not consider heterogeneity in risk of \hiv acquisition or transmission%
~\cite{Granich2009,Eaton2012,TBD}.
Even models that did consider risk heterogeneity
rarely acknowledged known differences in the treatment cascade across risk groups%
~\cite{Eaton2012,TBD},
such as among key populations~\cite{Gupta2017}.
\textcite{Knight2019} showed that
the modelled impact of achieving \mbox{90-90-90} in a population overall
was highly dependent on which risk groups were left behind in the remaining \mbox{``10-10-10''},
emphasizing that differences in treatment cascade cannot be ignored.
Finally, simulated sexual mixing between risk groups
has generally been simpler than observed in reality~\cite{TBD},
with potential implications for validity of modelling results.
For example, \textcite{Wang2019} have shown that
failure to model assortative mixing by \hiv status among men who have sex with men
may result in underestimated impact of pre-exposure prophylaxis.
\par
One major reason why risk groups and mixing may be missing
from \hiv transmission models is lack of data.
Despite best efforts, key populations are often not captured by
large-scale demographic and health surveys, such as those by \textcite{DHS},
due to several barriers:
household-based sampling methodologies,
criminalization of lifestyle,
social desirability bias,
and stigma~\cite{Abdul-Quader2014}.
For example, in the 2006-07 Eswatini demographic and health survey~\cite{DHS-SWZ}
just 0.2\% male respondents reported paying for sex, while estimates of
commercial sex client populations in similar regions were as high as 8\%~\cite{Carael2006}.
% TODO: FSW too/instead?
In many cases, parallel surveys with specific sampling methodologies and community involvement
can overcome these barriers, facilitating data collection on key populations~\cite{SwaziKP2014}.
Moreover, collection of key populations data can and should be integrated with
modelling work and evaluation of tailored interventions.
% --------------------------------------------------------------------------------------------------
\subsubsection*{Future Work}
This review aims to identify parameterizations of risk heterogeneity and mixing
used in previous transmission models of \hiv in Sub-Saharan Africa.
Identified parameterizations will then be considered in a systematic model comparison study,
similar to that by \textcite{Hontelez2013}.
For example, the projected impact of universal treatment as prevention will be compared
in models with versus without female sex workers, or
in models with versus without mixing by risk group.
In comparing parameterizations, potential biases and uncertainties associated with
simpler models can be estimated.
Furthermore, considering the importance of data to inform complex models,
the model comparison study will identify key pieces of information
which are necessary to construct accurate models,
so that these data may be prioritized for collection going forward.
